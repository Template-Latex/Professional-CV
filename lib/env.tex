% Template:     Professional-CV
% Documento:    Definición de entornos
% Versión:      3.0.2 (25/08/2021)
% Codificación: UTF-8
%
% Autor: Pablo Pizarro R.
%        Facultad de Ciencias Físicas y Matemáticas
%        Universidad de Chile
%        pablo@ppizarror.com
%
% Sitio web:    [https://latex.ppizarror.com/professional-cv]
% Licencia MIT: [https://opensource.org/licenses/MIT]

% Crea una sección identada
\newenvironment{indentsection}{%
	\begin{list}{}{%
		\setlength{\leftmargin}{0.75\newparindent}
		\setlength{\parsep}{0pt}
		\setlength{\parskip}{0pt}
		\setlength{\itemsep}{0pt}
		\setlength{\topsep}{0pt}}
	}{
	\end{list}
}

% Crea una sección identada doble
\newenvironment{indentsectiondouble}{%
	\begin{list}{}{%
			\setlength{\leftmargin}{1.5\newparindent}
			\setlength{\parsep}{0pt}
			\setlength{\parskip}{0pt}
			\setlength{\itemsep}{0pt}
			\setlength{\topsep}{0pt}}
	}{
	\end{list}
}

% Crea un bloque de contenido
%	#1	Título
%	#2	Estilo de línea
\newenvironment{cvblock}[2][]{%
	\needspace{1\baselineskip}%
	\textcolor{\cvblocklinecolor}{%
		\ifx\hfuzz#1\hfuzz%
			\hrule%
		\else%
			\ifthenelse{\equal{#1}{..}}{%
				\noindent%
				\hspace*{-1.65em}%
				\makebox[1.01\textwidth][l]{\dotfill}%
				\vspace{-2em}
			}{%
			\ifthenelse{\equal{#1}{none}}{%
				\vspace{-1em}%
			}{%
				\throwwarning{Estilo de linea invalido}%
				\stop%
			}}%
		\fi%
	}%
	\ifx\hfuzz#2\hfuzz%
		\throwwarning{Titulo del bloque no definido}%
		\stop%
	\else%
		\roottitle{#2}%
	\fi%
	\vspace{0.15em}%
	\global\def\GLOBALcvfinishlist {false}%
	\global\def\GLOBALhasinstitution {false}%
	\global\def\GLOBALinstitutionshortend {false}%
	}{%
	\ifthenelse{\equal{\GLOBALhasinstitution}{false}}{}{%
		\ifthenelse{\equal{\GLOBALinstitutionshortend}{false}}{%
			\vspace{-0.05\baselineskip}%
		}{%
			\vspace{-1.15\baselineskip}%
		}%
	}%
	\ifthenelse{\equal{\GLOBALcvfinishlist}{true}}{%
		\vspace{-0.75\baselineskip}%
	}{}
	\par%
	\vspace{0.6em}%
}

% Párrafo de descripción
%	#1	Estilo de línea
\newenvironment{summary}[1][]{%
	\begin{cvblock}[#1]{\nomsummary}%
		\begingroup%
		\vspace{-1em}%
		\begin{multicols}{2}%
			\noindent%
		}{
		\end{multicols}
		\vspace{-0.75\baselineskip}
		\par
		\endgroup
	\end{cvblock}
}

% Párrafo de descripción con foto de perfil
%	#1	Estilo de línea
\newenvironment{photosummary}[1][]{%
	\begin{cvblock}[#1]{\nomsummary}%
		\vspace{0.1em}%
		\noindent%
		\begin{minipage}[t]{\userphotosize cm}%
			\begin{flushleft}%
				\begingroup%
				\setlength{\fboxsep}{0pt}%
				~ \\ \vspace{-0.75em}%
				\ifthenelse{\equal{\showuserphotoborder}{true}}{%
					\noindent \textcolor{\userphotobordercolor}{\fbox{\includegraphics[width=0.99\linewidth]{\fotoautor}}}%
				}{%
					\noindent \includegraphics[width=\userphotosize cm,height=\userphotosize cm]{\fotoautor}%
				}%
				\vspace{-0.7em}%
				\endgroup%
			\end{flushleft}%
		\end{minipage}%
		\hspace*{0.54cm}%
		\begingroup%
		\def\arraystretch{1.3}%
		\begin{minipage}[t]{\dimexpr\linewidth-\userphotosize cm-0.57cm}%
		}{%
		\end{minipage}%
		\endgroup%
		\par%
	\end{cvblock}%
	\vspace{1em}%
}

% Tabla de datos personales
%	#1	Estilo de línea
\newenvironment{personaltabledata}[1][]{%
	\begin{cvblock}[#1]{\nompersonaldata}%
		\vspace{0.1em}%
		\noindent%
		\ifthenelse{\equal{\personaltabledatapic}{true}}{%
			\begin{minipage}[t]{\userphotosize cm}%
				\begin{flushleft}%
					\begingroup%
					\setlength{\fboxsep}{0pt}%
					~ \\ \vspace{-0.95em}%
					\ifthenelse{\equal{\showuserphotoborder}{true}}{%
						\noindent \textcolor{\userphotobordercolor}{\fbox{\includegraphics[width=0.99\linewidth]{\fotoautor}}}
					}{%
						\noindent \includegraphics[width=\userphotosize cm,height=\userphotosize cm]{\fotoautor}%
					}%
					\endgroup%
				\end{flushleft}%
			\end{minipage}%
		}{}%
		\hspace*{0.34cm}%
		\begingroup%
		\def\arraystretch{1.3}%
		\begin{minipage}[t]{\linewidth-\userphotosize cm-0.37cm}%
			\vspace{-0.95cm}%
			\begin{table}[H]%
				\begin{tabular}{ll}%
				}{%
				\end{tabular}%
			\end{table}%
		\end{minipage}%
		\endgroup%
		\par%
	\end{cvblock}%
	\vspace{0.55em}%
}

% Nueva institución
%	#1	Opcional: url o link institución
%	#2	Nombre institución
%	#3	Ubicación institución
\newenvironment{institution}[3][]{%
	\needspace{2\baselineskip}%
	\ifthenelse{\equal{\institutiontitlebold}{true}}{%
		\def\LOCALinstitutiontitle {\textbf{#2}}%
	}{%
		\def\LOCALinstitutiontitle {#2}%
	}
	\nopagebreak[4]%
	\hypersetup{urlcolor=\instnamecolor}%
	\begin{indentsection}%
		\item []%
		\noindent%
		\ifx\hfuzz#3\hfuzz%
			\begin{minipage}{\linewidth}%
				\ifx\hfuzz#1\hfuzz%
					\textscale{1.1}{\textcolor{\instnamecolor}{\LOCALinstitutiontitle}}
				\else
					\textscale{1.1}{\href{#1}{\LOCALinstitutiontitle}}
				\fi
			\end{minipage}
		\else%
			\begin{minipage}[t][][t]{.558\linewidth}%
				\ifx\hfuzz#1\hfuzz
					\textscale{1.1}{\textcolor{\instnamecolor}{\LOCALinstitutiontitle}}
				\else
					\textscale{1.1}{\href{#1}{\LOCALinstitutiontitle}}
				\fi
			\end{minipage}
			\begin{minipage}[t][][t]{.44\linewidth}%
				\begin{flushright}
					\noindent \textcolor{\instregioncolor}{\textsc{#3}}
				\end{flushright}
			\end{minipage}
		\fi%
		\vspace{-\baselineskip}%
		\break%
	\end{indentsection}
	\hypersetup{urlcolor=\urlcolor}%
	\vspace{-1.5\baselineskip}%
	\noindent%
	\global\def\GLOBALinstitutionenabled {true}
	}{%
	\global\def\GLOBALinstitutionenabled {false}%
	\global\def\GLOBALhasinstitution {true}%
	\par%
	\vspace{0.2\baselineskip}%
	\nopagebreak[4]%
}

% Crea la firma del usuario
\newenvironment{signature}{%
	\vfill%
	\begin{flushright}%
		\begin{tabular}{c}%
			\includegraphics[width=\usersignaturesize cm]{\firmaautor} \\%
		}{%
		\end{tabular}%
	\end{flushright}%
	\vspace{-1em}%
}

% Itemize en negrita
%	#1	Parámetros opcionales
\newenvironment{itemizebf}[1][]{%
	\begin{itemize}[font=\bfseries,#1]%
	}{
	\end{itemize}
}

% Enumerate en negrita
%	#1	Parámetros opcionales
\newenvironment{enumeratebf}[1][]{%
	\begin{enumerate}[font=\bfseries,#1]%
	}{
	\end{enumerate}
}
