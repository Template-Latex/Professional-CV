% Template:     Professional-CV
% Documento:    Definición de entornos
% Versión:      1.3.3 (05/04/2020)
% Codificación: UTF-8
%
% Autor: Pablo Pizarro R.
%        Facultad de Ciencias Físicas y Matemáticas
%        Universidad de Chile
%        pablo@ppizarror.com
%
% Sitio web:    [https://latex.ppizarror.com/Professional-CV/]
% Licencia MIT: [https://opensource.org/licenses/MIT/]

% Crea una sección identada
\newenvironment{indentsection}{
	\begin{list}{}{
		\setlength{\leftmargin}{\newparindent}
		\setlength{\parsep}{0pt}
		\setlength{\parskip}{0pt}
		\setlength{\itemsep}{0pt}
		\setlength{\topsep}{0pt}}
	}{
	\end{list}
}

% Crea un bloque de contenido
%	#1	Título
%	#2	Estilo de línea
\newenvironment{cvblock}[2][]{
	\textcolor{\cvblocklinecolor}{
		\ifx\hfuzz#1\hfuzz
			\breakvspace{1em} \hrule \nobreakvspace{0em}
		\else
			\ifthenelse{\equal{#1}{..}}{
				\noindent \makebox[1\linewidth]{\dotfill}
				\vspace{-2em}
			}{
				\ifthenelse{\equal{#1}{none}}{
			}{
				\throwwarning{Estilo de linea invalido}
				\stop
			}}
		\fi
	}
	\ifx\hfuzz#2\hfuzz
		\throwwarning{Titulo del bloque no definido}
		\stop
	\else
		\roottitle{#2}
	\fi
	\vspace{0.5em}
	}{
	\par
}

% Párrafo de descripción
\newenvironment{summary}[1][]{
	\begin{cvblock}[#1]{\nomsummary}
		\begingroup
		\vspace{-1em}
		\begin{multicols}{2}
			\noindent
		}{
		\end{multicols}
		\vspace{-0.1cm}
		\par
		\endgroup
	\end{cvblock}
}

% Párrafo de descripción con foto de perfil
\newenvironment{photosummary}[1][]{
	\begin{cvblock}[#1]{\nomsummary}
		\vspace{0.1em}
		\noindent
		\begin{minipage}[t]{\userphotosize cm}
			\begin{flushleft}
				\begingroup
				\setlength{\fboxsep}{0pt}
				~ \\ \vspace{-1em}
				\ifthenelse{\equal{\showuserphotoborder}{true}}{
					\noindent \textcolor{\userphotobordercolor}{\fbox{\includegraphics[width=0.99\linewidth]{\fotoautor}}}
				}{
					\noindent \includegraphics[width=\userphotosize cm,height=\userphotosize cm]{\fotoautor}
				}
				\vspace{-0.7em}
				\endgroup
			\end{flushleft}
		\end{minipage}
		\hspace*{0.37cm}
		\begingroup
		\def\arraystretch{1.3}
		\begin{minipage}[t]{0.79\linewidth}
		}{
		\end{minipage}
		\endgroup
		\par
	\end{cvblock}
}

% Tabla de datos personales
\newenvironment{personaltabledata}[1][]{
	\begin{cvblock}[#1]{\nompersonaldata}
		\vspace{0.1em}
		\noindent
		\ifthenelse{\equal{\personaltabledatapic}{true}}{
			\begin{minipage}[t]{\userphotosize cm}
				\begin{flushleft}
					\begingroup
					\setlength{\fboxsep}{0pt}
					~ \\ \vspace{-1em}
					\ifthenelse{\equal{\showuserphotoborder}{true}}{
						\noindent \textcolor{\userphotobordercolor}{\fbox{\includegraphics[width=0.99\linewidth]{\fotoautor}}}
					}{
						\noindent \includegraphics[width=\userphotosize cm,height=\userphotosize cm]{\fotoautor}
					}
					\vspace{-0.7em}
					\endgroup
				\end{flushleft}
			\end{minipage}
		}{}
		\hspace*{0cm}
		\begingroup
		\def\arraystretch{1.3}
		\begin{minipage}[t]{0.79\linewidth}
			\vspace*{-0.945cm}
			\begin{table}[H]
				\begin{tabular}{ll}
				}{
				\end{tabular}
			\end{table}
		\end{minipage}
		\endgroup
		\par
	\end{cvblock}
}

% Nueva institución
%	#1	Opcional: url o link institución
%	#2	Nombre institución
%	#3	Ubicación institución
\newenvironment{institution}[3][]{
	\nopagebreak[4]
	\begin{indentsection}
		\item []
		\noindent
		\ifx\hfuzz#3\hfuzz
			\begin{minipage}{0.995\linewidth}
				\ifx\hfuzz#1\hfuzz
				\textscale{1.1}{\textcolor{\linkcolor}{#2}}
				\else
				\textscale{1.1}{\href{#1}{#2}}
				\fi
			\end{minipage}
		\else
			\begin{minipage}{.55\linewidth}
				\ifx\hfuzz#1\hfuzz
				\textscale{1.1}{\textcolor{\linkcolor}{#2}}
				\else
				\textscale{1.1}{\href{#1}{#2}}
				\fi
			\end{minipage}
			\begin{minipage}{.44\linewidth}
				\begin{flushright}
					\noindent \textcolor{\instregioncolor}{\textsc{#3}}
				\end{flushright}
			\end{minipage}
		\fi
		\break
		\begin{minipage}{1.0\linewidth}
			\noindent
			}{
		\end{minipage}
	\end{indentsection}
	\par
	\nopagebreak[4]
}

% Crea la firma del usuario
%	#1	Título
\newenvironment{signature}{
	\vfill
	\begin{flushright}
		\begin{tabular}{c}
			\includegraphics[width=\usersignaturesize cm]{\firmaautor} \\
		}{
		\end{tabular}
	\end{flushright}
}
