% Template:     Professional-CV
% Documento:    Funciones utilitarias
% Versión:      2.2.0 (24/07/2021)
% Codificación: UTF-8
%
% Autor: Pablo Pizarro R.
%        Facultad de Ciencias Físicas y Matemáticas
%        Universidad de Chile
%        pablo@ppizarror.com
%
% Sitio web:    [https://latex.ppizarror.com/professional-cv]
% Licencia MIT: [https://opensource.org/licenses/MIT]

% Lanza un mensaje de error
% 	#1	Función del error
%	#2	Mensaje
\newcommand{\throwerror}[2]{
	\errmessage{LaTeX Error: \noexpand#1 #2 (linea \the\inputlineno)}
	\stop
}

% Lanza un mensaje de advertencia
%	#1	Mensaje
\newcommand{\throwwarning}[1]{
	\errmessage{LaTeX Warning: #1 (linea \the\inputlineno)}
}

% Lanza un mensaje de error indicando mala configuración
%	#1	Mensaje de error
% 	#2	Configuración usada
%	#3	Valores esperados
\newcommand{\throwbadconfig}[3]{
	\errmessage{LaTeX Warning: #1 \noexpand #2=#2. Valores esperados: #3}
	\stop
}

% Lanza un mensaje de error indicando mala configuración dentro de begin{document}
%	#1	Mensaje de error
% 	#2	Configuración usada
%	#3	Valores esperados
\newcommand{\throwbadconfigondoc}[3]{
	\errmessage{#1 \noexpand #2=#2. Valores esperados: #3}
	\stop
}

% Comprueba si una variable está definida
%	#1	Variable
\newcommand{\checkvardefined}[1]{%
	\ifthenelse{\isundefined{#1}}{%
		\errmessage{LaTeX Warning: Variable \noexpand#1 no definida}
		\stop
	}{}
}

% Comprueba si una variable está definida
%	#1	Variable
%	#2	Mensaje
\newcommand{\checkextravarexist}[2]{%
	\ifthenelse{\isundefined{#1}}{%
		\errmessage{LaTeX Warning: Variable \noexpand#1 no definida}
		\ifx\hfuzz#2\hfuzz%
			\errmessage{LaTeX Warning: Defina la variable en el bloque de INFORMACION DEL USUARIO al comienzo del archivo principal del Template}
		\else%
			\errmessage{LaTeX Warning: #2}
		\fi%
	}{}
}

% Lanza un mensaje de error si una variable no ha sido definida
% 	#1	Función del error
%	#2	Variable
%	#3	Mensaje
\newcommand{\emptyvarerr}[3]{
	\ifx\hfuzz#2\hfuzz%
		\errmessage{LaTeX Warning: \noexpand#1 #3 (linea \the\inputlineno)}
	\fi
}

% Cambia márgenes de las páginas [cm]
% 	#1	Margen izquierdo
%	#2	Margen superior
%	#3	Margen derecho
%	#4	Margen inferior
\newcommand{\setpagemargincm}[4]{%
	\newgeometry{left=#1cm, top=#2cm, right=#3cm, bottom=#4cm}%
}

% Cambia los márgenes del documento
%	#1	Margen izquierdo
%	#2	Margen derecho
\newcommand{\changemargin}[2]{%
	\emptyvarerr{\changemargin}{#1}{Margen izquierdo no definido}%
	\emptyvarerr{\changemargin}{#2}{Margen derecho no definido}%
	\list{}{\rightmargin#2\leftmargin#1}\item[]%
}
\let\endchangemargin=\endlist

% Título principal con subtítulo y último cambio
%	#1	Nombre autor
%	#2	Edad corta autor
%	#3	Último cambio
\newcommand{\maintitle}[3]{%
	\emptyvarerr{\maintitle}{#1}{Nombre autor no definido}%
	\emptyvarerr{\maintitle}{#2}{Edad corta autor no definido}%
	%\emptyvarerr{\maintitle}{#3}{Fecha de ultimo cambio no definido}
	\noindent {\fontsizemaintitle \stylemaintitle #1} \quad \emph{#2} \hfill \quad {\scriptsize #3} \\ \vspace{-1em}%
}

% Escribe el encabezado del CV
\newcommand{\writeheader}{%
	% Escribe título
	\ifthenelse{\equal{\writetitleheader}{true}}{%
		\begin{center}%
			\fontsize{16}{12} \selectfont \vspace{0.3cm} \textcolor{\titlecolor}{\headertitle}
		\end{center}
		\vspace{0.1cm}
	}{}
	\ifthenelse{\equal{\writelastchangeheader}{true}}{%
		\maintitle{\textcolor{\titlecolor}{\nombreautor}}{\textcolor{\headertextcolor}{\mesnacimientoautor \dianacimientoautor, \anonacimientoautor}}{\textcolor{\lastchangeheadercolor}{\nomlastchange\ \today}}
	}{%
		\maintitle{\textcolor{\titlecolor}{\nombreautor}}{\textcolor{\headertextcolor}{\mesnacimientoautor \dianacimientoautor, \anonacimientoautor}}{}
	}
	
	% Escribe datos autor
	{%
		\noindent
		\ifthenelse{\equal{\correoautor}{}}{}{%
			\href{mailto:\correoautor}{\correoautor}%
		}
		\textcolor{\headertextcolor}{%
		\ifthenelse{\equal{\telefonoautor}{}}{}{%
			\headerseparator\telefonoautor}%
		\ifthenelse{\equal{\ciautor}{\xspace}}{}{%
			\headerseparator\nomciheader: \ciautor }%
		\ifthenelse{\equal{\linkredsocial}{}}{}{%
			\headerseparator\href{\linkredsocial}{\linkredsocial}} \\
		\ifthenelse{\equal{\direccionautor}{\xspace}}{%
			\ifthenelse{\equal{\regionautor}{\xspace}}{}{\regionautor\headerseparator}
			\ifthenelse{\equal{\paisautor}{\xspace}}{}{\paisautor}
		}{%
			\direccionautor%
			\ifthenelse{\equal{\regionautor}{\xspace}}{}{\headerseparator\regionautor}
			\ifthenelse{\equal{\paisautor}{\xspace}}{}{\headerseparator\paisautor}
		}}%
	}
}

% Título de las secciones principales
%	#1	Título
\newcommand*{\roottitle}[1]{%
	\emptyvarerr{\roottitle}{#1}{Titulo no definido}
	\section*{#1}\vspace{-0.3em}\nopagebreak[4]
	\addcontentsline{toc}{section}{#1}
}

% Crea un acrónimo de forma fácil
%	#1	Acrónimo
\newcommand*{\acr}[1]{%
	\textscale{.85}{#1}
}

% Crea un campo de texto
% 	#1	Texto
\newcommand{\bodytext}[1]{%
	\nopagebreak[4]%
	\begin{indentsection}%
		\item[] #1
	\end{indentsection}
	\pagebreak[2]
}

% Crea una entrada en la tabla de datos personales
%	#1	Nombre entrada
%	#2	Datos entrada
\newcommand{\personaltableentry}[2]{%
	\emptyvarerr{\personaltableentry}{#1}{Nombre entrada no definido}%
	\emptyvarerr{\personaltableentry}{#2}{Datos entrada no definidos}%
	\textcolor{\personaltblentcolor}{\textbf{#1:}} & #2 \\
}

% Inserta un objeto en un elemento institución
%	#1	Cargo
%	#2	Indica si escribe fechas o no
%	#3	Fecha inicial
%	#4	Fecha final
%	#5	Ancho título
%	#6	Ancho fecha
%	#7	Descripción
\newcommand{\newinstitutionentry}[7]{%
	\emptyvarerr{\newinstitutionentry}{#1}{Cargo o posición no definido}
	\emptyvarerr{\newinstitutionentry}{#5}{Dimensiones de titulo no definido}
	\emptyvarerr{\newinstitutionentry}{#6}{Dimensiones de fecha no definido}
	\nopagebreak[4]
	\begin{indentsectiondouble}
		\item []
		\noindent
		\begin{minipage}[t][][t]{#5}%
			{\textcolor{\instentrytitlecolor}{\textbf{#1}}}
		\end{minipage}
		\begin{minipage}[t][][t]{#6}%
			\begin{flushright}
				\noindent \textcolor{\datecolor}{\emph{#2#4#3}}
			\end{flushright}
		\end{minipage}
		\break
		\ifx\hfuzz#7\hfuzz
			\vspace*{-0.25em}
		\else
			\begin{minipage}{1.0\linewidth}%
				\vspace{0.2cm}
				\bodytext{#7}
				\par
			\end{minipage}
		\fi
	\end{indentsectiondouble}
}

% Entrada en institución
%	#1	Cargo
%	#2	Fecha inicial
%	#3	Fecha final
%	#4	Descripción
\newcommand{\institutionentry}[4]{%
	\newinstitutionentry{#1}{#2 }{ #3}{\dateseparator}{.68\linewidth}{.31\linewidth}{#4}%
}

% Entrada en institución con fecha corta
%	#1	Cargo
%	#2	Fecha inicial
%	#3	Fecha final
%	#4	Descripción
\newcommand{\institutionentryshort}[4]{%
	\newinstitutionentry{#1}{#2 }{ #3}{\dateseparator}{.85\linewidth}{.14\linewidth}{#4}%
}

% Entrada en institución sin fecha
%	#1	Cargo
%	#2	Descripción
\newcommand{\institutionentrynodate}[2]{%
	\newinstitutionentry{#1}{}{}{}{.99\linewidth}{.0\linewidth}{#2}%
}

% Otro tipo de entrada en cvblock
%	#1	Título
%	#2	Contenido
\newcommand{\otherentry}[2]{%
	\begin{basedescript}%
		{\setlength{\leftmargin}{\doubleparindent}}%
		\item[\hspace{\newparindent}\textcolor{\otherentrytitlecolor}{\textbf{#1}}] #2%
	\end{basedescript}
	\vspace{-0.6cm}
}

% Inserta un elemento en el bloque de firma
%	#1	Elemento en la firma
\newcommand{\signatureentry}[1]{
	\texttt{\MakeUppercase{#1}} \\
}

% Crea un bull
\newcommand{\sbullet}{\ \ \raisebox{0.11em}[-1em][-1em]{\tiny $\bullet$} \ \ }

% Variaciones de vspace
\newcommand{\breakvspace}[1]{\pagebreak[2] \vspace{#1} \pagebreak[2]}
\newcommand{\nobreakvspace}[1]{\nopagebreak[4] \vspace{#1} \nopagebreak[4]}

% Apóstrofe
\newcommand{\apo}{\raisebox{-.18ex}{'}{\hspace{-.1em}}}

% Inserta un texto entre comillas
\newcommand{\quotes}[1]{\enquote*{#1}}

% Inserta un email con un link cliqueable
\newcommand{\insertemail}[1]{\href{mailto:#1}{\texttt{#1}}}

% Inserta un teléfono celular
\newcommand{\insertphone}[1]{\href{tel:#1}{\texttt{#1}}}

% Actualiza el padding de las celdas de las tablas
%	#1	Padding horizontal (em)
%	#2	Padding vertical (em)
\newcommand{\settablecellpadding}[2]{
	\emptyvarerr{\settablecellpadding}{#1}{Padding horizontal no definido}
	\emptyvarerr{\settablecellpadding}{#2}{Padding vertical no definido}
	\setlength{\tabcolsep}{#1 em} % Horizontal
	\def\arraystretch {#2} % Vertical
}
