% Template:     Professional-CV
% Documento:    Configuraciones iniciales del documento
% Versión:      2.1.2 (01/01/2021)
% Codificación: UTF-8
%
% Autor: Pablo Pizarro R.
%        Facultad de Ciencias Físicas y Matemáticas
%        Universidad de Chile
%        pablo@ppizarror.com
%
% Sitio web:    [https://latex.ppizarror.com/Professional-CV/]
% Licencia MIT: [https://opensource.org/licenses/MIT/]

% Se revisa si las variables no han sido borradas
\checkvardefined{\anonacimientoautor}
\checkvardefined{\ciautor}
\checkvardefined{\correoautor}
\checkvardefined{\dianacimientoautor}
\checkvardefined{\direccionautor}
\checkvardefined{\edadautor}
\checkvardefined{\linkredsocial}
\checkvardefined{\mesnacimientoautor}
\checkvardefined{\nombreautor}
\checkvardefined{\regionautor}
\checkvardefined{\telefonoautor}

% Se añade \xspace a las variables
\makeatletter
\g@addto@macro\anonacimientoautor\xspace
\g@addto@macro\ciautor\xspace
\g@addto@macro\dianacimientoautor\xspace
\g@addto@macro\direccionautor\xspace
\g@addto@macro\edadautor\xspace
\g@addto@macro\mesnacimientoautor\xspace
\g@addto@macro\nombreautor\xspace
\g@addto@macro\paisautor\xspace
\g@addto@macro\regionautor\xspace
\g@addto@macro\telefonoautor\xspace
\makeatother

% Se define metadata del pdf
\ifthenelse{\equal{\cfgshowbookmarkmenu}{true}}{
	\def\cdfpagemodepdf {UseOutlines}
}{
	\def\cdfpagemodepdf {UseNone}
}
\hypersetup{
	bookmarksopen={\cfgpdfbookmarkopen},
	bookmarksopenlevel={\cfgbookmarksopenlevel},
	bookmarkstype={toc},
	pdfauthor={\nombreautor},
	pdfcenterwindow={\cfgpdfcenterwindow},
	pdfcopyright={Currículum Vítae, \nombreautor. Email: \correoautor. Teléfono: \telefonoautor},
	pdfcreator={LaTeX},
	pdfdisplaydoctitle={\cfgpdfdisplaydoctitle},
	pdffitwindow={\cfgpdffitwindow},
	pdfinfo={
		Autor.Email={\correoautor},
		Autor.Nombre={\nombreautor},
		Autor.Telefono={\telefonoautor},
		Autor.Ubicacion.Direccion={\direccionautor},
		Autor.Ubicacion.Pais={\paisautor},
		Autor.Ubicacion.Region={\regionautor},
		Template.Autor.Alias={ppizarror},
		Template.Autor.Email={pablo@ppizarror.com},
		Template.Autor.Nombre={Pablo Pizarro R.},
		Template.Autor.Web={https://ppizarror.com/},
		Template.Codificacion={UTF-8},
		Template.Fecha={01/01/2021},
		Template.Licencia.Tipo={MIT},
		Template.Licencia.Web={https://opensource.org/licenses/MIT/},
		Template.Nombre={Professional-CV},
		Template.Tipo={Normal},
		Template.Version.Dev={2.1.2},
		Template.Version.Hash={5C4216A515560668F1067BE0A098580F},
		Template.Version.Release={2.1.2},
		Template.Web.Dev={https://github.com/Template-Latex/Professional-CV/},
		Template.Web.Manual={https://latex.ppizarror.com/Professional-CV/}
	},
	pdfkeywords={CV, \nombreautor, Currículum, Vítae},
	pdflang={\documentlang},
	pdfmenubar={\cfgpdfmenubar},
	pdfpagelayout={\cfgpdfpagemode},
	pdfpagemode={\cdfpagemodepdf},
	pdfproducer={Professional-CV v2.1.2 | (Pablo Pizarro R.) ppizarror.com},
	pdfremotestartview={Fit},
	pdfstartpage={1},
	pdfstartview={\cfgpdfpageview},
	pdfsubject={C.V. \nombreautor},
	pdftitle={Currículum Vítae \nombreautor},
	pdftoolbar={\cfgpdftoolbar},
	pdftype={Text}
}

% Establece la carpeta de imágenes por defecto
\graphicspath{{./img}}

% Ajuste del entrelineado
\renewcommand{\baselinestretch}{\documentinterline}

% Configuración de los colores
\def\linkcolor{\urlcolor}
\color{\maintextcolor} % Color principal
\arrayrulecolor{\tablelinecolor} % Color de las líneas de las tablas
\sethlcolor{\highlightcolor} % Color del subrayado por defecto
\ifthenelse{\equal{\showborderonlinks}{true}}{
	\hypersetup{
		% Color de links con borde
		citebordercolor=\numcitecolor,
		linkbordercolor=\linkcolor,
		urlbordercolor=\urlcolor
	}
}{
	\hypersetup{
		% Color de links sin borde
		hidelinks,
		colorlinks=true,
		citecolor=\numcitecolor,
		linkcolor=\linkcolor,
		urlcolor=\urlcolor
	}
}
\ifthenelse{\equal{\colorpage}{white}}{}{
	\pagecolor{\colorpage}
}

% Definición de la fuente
\usepackage{ifxetex}
\ifxetex
	\usepackage{fontspec}
	\setmainfont
	[ExternalLocation,
	Mapping=tex-text,
	Numbers=OldStyle,
	Ligatures={Common,Contextual},
	BoldFont=texgyrepagella-bold.otf,
	ItalicFont=texgyrepagella-italic.otf,
	BoldItalicFont=texgyrepagella-bolditalic.otf]
	{texgyrepagella-regular.otf}
	\usepackage[protrusion]{microtype}
\else
	\usepackage{tgpagella}
	\usepackage[expansion,protrusion]{microtype}
\fi

% Configuración de la identación
\newlength{\newparindent}
\addtolength{\newparindent}{\parindent}
\newlength{\doubleparindent}
\addtolength{\doubleparindent}{\parindent}
\addtolength{\doubleparindent}{\parindent}

% Configuración de hbox y vbox
\hfuzz=200pt \vfuzz=200pt
\hbadness=\maxdimen \vbadness=\maxdimen

% Se activa el word-wrap para textos con \texttt{}
\ttfamily \hyphenchar\the\font=`\-

% Se define el tipo de texto de los url
\urlstyle{tt}

% Se define la versión menor a compilar
\pdfminorversion=\pdfcompileversion
