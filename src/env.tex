% Template:     Professional-CV
% Documento:    Definición de entornos
% Versión:      4.1.0 (18/08/2024)
% Codificación: UTF-8
%
% Autor: Pablo Pizarro R.
%        pablo@ppizarror.com
%
% Manual template: [https://latex.ppizarror.com/professional-cv]
% Licencia MIT:    [https://opensource.org/licenses/MIT]

% Definición de variables globales
\global\def\GLOBALinstitutionenabled {false}
\global\def\GLOBALinstitutionicon {}
\global\def\GLOBALinstitutioniconmargin {}
\global\def\GLOBALpersonaltableenabled {false}
\newcounter{institutionindentdepth}

% Crea un bloque de contenido
%	#1	Estilo de línea
%	#2	Título
\newenvironment{cvblock}[2][]{%
	\ifx\hfuzz#1\hfuzz%
		\printifspace[i]{8}{%
			\textcolor{\cvblocklinecolor}{\hrule}%
			\vspace{0.15em}%
		}{}%
	\else%
		\ifthenelse{\equal{#1}{..}}{%
			\printifspace[i]{8}{%
				\textcolor{\cvblocklinecolor}{%
					\noindent%
					\hspace*{-1.65em}%
					\makebox[1.01\textwidth][l]{\dotfill}%
					\vspace{-0.8em}%
				}%
			}{}%
		}{%
		\ifthenelse{\equal{#1}{none}}{%
			\printifspace[s]{3}{\vspace{3\baselineskip}}{}%
		}{%
			\throwwarning{Estilo de linea invalido}%
			\stop%
		}}%
	\fi%
	\ifx\hfuzz#2\hfuzz%
		\throwwarning{Titulo del cvblock no definido}%
		\stop%
	\else%
		\ifthenelse{\equal{#2}{}}{}{%
			\section*{#2}\nopagebreak[4]%
			\addcontentsline{toc}{section}{#2}%
		}%
	\fi%
	\begin{sloppypar}%
	}{%
	\end{sloppypar}%
}

% Bloque itemizado
\newenvironment{cvblocki}[2][]{%
	\begin{cvblock}[#1]{#2}%
		\begin{itemizebf}%
		}{%
		\end{itemizebf}%
	\end{cvblock}%
}
\newenvironment{cvblocke}[2][]{%
	\begin{cvblock}[#1]{#2}%
		\begin{enumeratebf}%
		}{%
		\end{enumeratebf}%
	\end{cvblock}%
}

% Crea un bloque de contenido estricto a los saltos de línea
%	#1	Estilo de línea
%	#2	Título
\newenvironment{cvsblock}[2][]{%
	\begin{ignorelinebreaks}%
		\begin{cvblock}[#1]{#2}%
		}{%
		\end{cvblock}%
	\end{ignorelinebreaks}%
}

% Bloque estricto itemizado
\newenvironment{cvsblocki}[2][]{%
	\begin{cvsblock}[#1]{#2}%
		\begin{itemizebf}%
		}{%
		\end{itemizebf}%
	\end{cvsblock}%
}
\newenvironment{cvsblocke}[2][]{%
	\begin{cvsblock}[#1]{#2}%
		\begin{enumeratebf}%
		}{%
		\end{enumeratebf}%
	\end{cvsblock}%
}

% Párrafo de descripción
%	#1	Estilo de línea
\newenvironment{summary}[1][]{%
	\summarystylemarginprefix{#1}%
	\begin{cvblock}[#1]{\namesummary}%
		\begingroup%
		\setlength\columnsep{20pt}%
		\summarystylemarginpostfix{#1}%
		\ifthenelse{\equal{\namesummary}{}}{%
			\vspace{-\baselineskip}%
		}{%
			\vspace{-0.75\baselineskip}%
		}%
		\begin{multicols}{2}%
			\noindent%
			\summaryfontsize%
		}{%
		\end{multicols}%
		\vspace{-0.75\baselineskip}%
		\endgroup%
	\end{cvblock}%
	\vspace{0.375\baselineskip}%
}

% Párrafo de descripción
%	#1	Estilo de línea
\newenvironment{summarynocol}[1][]{%
	\summarystylemarginprefix{#1}%
	\begin{cvblock}[#1]{\namesummary}%
		\summarystylemarginpostfix{#1}%
		\begingroup%
		\summaryfontsize%
		}{%
		\endgroup%
	\end{cvblock}%
	\vspace{0.375\baselineskip}%
}

% Párrafo de descripción con foto de perfil
%	#1	Estilo de línea
\newenvironment{photosummary}[1][]{%
	\summarystylemarginprefix{#1}%
	\begin{cvblock}[#1]{\namesummary}%
		\summarystylemarginpostfix{#1}%
		\vspace{0.1em}%
		\noindent%
		\begin{minipage}[t]{\userphotosize cm}%
			\begin{flushleft}%
				\begingroup%
				\setlength{\fboxsep}{0pt}%
				~ \\ \vspace{-0.75em}%
				\ifthenelse{\equal{\showuserphotoborder}{true}}{%
					\noindent \textcolor{\userphotobordercolor}{\fbox{\includegraphics[width=0.99\linewidth]{\photo}}}%
				}{%
					\noindent \includegraphics[width=\userphotosize cm,height=\userphotosize cm]{\photo}%
				}%
				\vspace{-0.7em}%
				\endgroup%
			\end{flushleft}%
		\end{minipage}%
		\hspace*{0.54cm}%
		\begingroup%
		\def\arraystretch {1.3}%
		\begin{minipage}[t]{\dimexpr\linewidth-\userphotosize cm-0.57cm}%
		\summaryfontsize%
		}{%
		\end{minipage}%
		\endgroup%
	\end{cvblock}%
	\vspace{1.6em}%
}

% Tabla de datos personales
%	#1	Estilo de línea
\newenvironment{personaltabledata}[1][]{%
	\global\def\GLOBALpersonaltableenabled {true}%
	\begin{cvblock}[#1]{\namepersonaldata}%
		\vspace{0.1em}%
		\noindent%
		\ifthenelse{\equal{\personaltabledatapic}{true}}{%
			\begin{minipage}[t]{\userphotosize cm}%
				\begin{flushleft}%
					\begingroup%
					\setlength{\fboxsep}{0pt}%
					~ \\ \vspace{-0.95em}%
					\ifthenelse{\equal{\showuserphotoborder}{true}}{%
						\noindent \textcolor{\userphotobordercolor}{\fbox{\includegraphics[width=0.99\linewidth]{\photo}}}%
					}{%
						\noindent \includegraphics[width=\userphotosize cm,height=\userphotosize cm]{\photo}%
					}%
					\endgroup%
				\end{flushleft}%
			\end{minipage}%
		}{}%
		\def\personaltabledataimgspace {0.5cm}%
		\hspace*{\personaltabledataimgspace}%
		\begingroup%
		\def\arraystretch {1.185}%
		\begin{minipage}[t]{\linewidth-\userphotosize cm-\personaltabledataimgspace}%
			\vspace{-0.95cm}%
			\begin{table}[H]%
				\begin{tabular}{ll}%
				}{%
				\end{tabular}%
			\end{table}%
		\end{minipage}%
		\endgroup%
	\end{cvblock}%
	\vspace{1.15em}%
	\global\def\GLOBALpersonaltableenabled {false}%
}

% Nueva institución
%	#1	Opcional: url o link institución
%	#2	Nombre institución
%	#3	Ubicación institución
\newenvironment{institution}[3][]{%
	\ifthenelse{\equal{\institutiontitlebold}{true}}{%
		\def\LOCALinstitutiontitle {\textbf{#2}}%
	}{%
		\def\LOCALinstitutiontitle {#2}%
	}%
	\nopagebreak[4]%
	\hypersetup{urlcolor=\instnamecolor}%
	\hfuzz=4pt%
	\begin{institutionindent}%
		\noindent%
		\def\LOCALinstitutioniconl {%
			\ifthenelse{\equal{\institutioniconleft}{true}}{%
				\ifthenelse{\equal{\GLOBALinstitutionicon}{}}{}{%
					\GLOBALinstitutionicon\hspace{\GLOBALinstitutioniconmargin}%
				}%
			}{}%
		}%
		\def\LOCALinstitutioniconr {%
			\ifthenelse{\equal{\institutioniconleft}{false}}{%
				\ifthenelse{\equal{\GLOBALinstitutionicon}{}}{}{%
					\hspace{\GLOBALinstitutioniconmargin}\GLOBALinstitutionicon%
				}%
			}{}%
		}%
		\ifx\hfuzz#3\hfuzz%
			\printifspace{3}{%
				\begin{minipage}{\linewidth}%
					\LOCALinstitutioniconl%
					\ifx\hfuzz#1\hfuzz%
						\textscale{1.1}{\textcolor{\instnamecolor}{\LOCALinstitutiontitle}}%
					\else%
						\textscale{1.1}{\href{#1}{\LOCALinstitutiontitle}}%
					\fi%
					\LOCALinstitutioniconr%
				\end{minipage}%
			}{}%
		\else%
			\printifspace{3}{%
				\begin{minipage}[t][][t]{0.56\linewidth}%
					\LOCALinstitutioniconl%
					\ifx\hfuzz#1\hfuzz%
						\textscale{1.1}{\textcolor{\instnamecolor}{\LOCALinstitutiontitle}}%
					\else%
						\textscale{1.1}{\href{#1}{\LOCALinstitutiontitle}}%
					\fi%
					\LOCALinstitutioniconr%
				\end{minipage}%
				\begin{minipage}[t][][t]{0.44\linewidth}%
					\begin{flushright}%
						\noindent \textcolor{\instregioncolor}{\textsc{#3}}%
					\end{flushright}%
				\end{minipage}%
			}{}%
		\fi%
		\break%
	\end{institutionindent}%
	\hypersetup{urlcolor=\urlcolor}%
	\vspace{-1.75\baselineskip}%
	\noindent%
	\global\def\GLOBALinstitutionicon {}%
	\global\def\GLOBALinstitutionenabled {true}~%
	}{%
	\vspace{0.25\baselineskip}%
	\global\def\GLOBALinstitutionenabled {false}%
	\nopagebreak[4]%
}

% Crea una sección identada para las instituciones
\newenvironment{institutionindent}{%
	\stepcounter{institutionindentdepth}%
	\begin{list}{}{%
		\ifnum\value{institutionindentdepth}=1%
			\setlength{\leftmargin}{\institutionoutermargin\newparindent}%
		\else%
			\setlength{\leftmargin}{\institutioninnermargin\newparindent}%
		\fi%
		\setlength{\parsep}{0pt}%
		\setlength{\parskip}{0pt}%
		\setlength{\itemsep}{0pt}%
		\setlength{\topsep}{0pt}%
	}%
		\item []%
		}{%
	\end{list}%
	\addtocounter{institutionindentdepth}{-1}%
}

% Nueva institución con ícono
%	#1	Opcional: url o link institución
%	#2	Nombre institución
%	#3	Ubicación institución
%	#4	Ícono de la institución
\newenvironment{institutionicon}[4][]{%
	\ifthenelse{\equal{\institutioniconsize}{small}}{%
		\global\def\GLOBALinstitutioniconmargin {0.55em}%
		\global\def\GLOBALinstitutionicon {\raisebox{-0.01\baselineskip}{%
			\includegraphics[height=0.65\baselineskip,width=0.65\baselineskip]{#4}}%
		}%
	}{%
	\ifthenelse{\equal{\institutioniconsize}{normal}}{%
		\global\def\GLOBALinstitutioniconmargin {0.65em}%
		\global\def\GLOBALinstitutionicon {\raisebox{-0.15\baselineskip}{%
			\includegraphics[height=1\baselineskip,width=1\baselineskip]{#4}}%
		}%
	}{%
	\ifthenelse{\equal{\institutioniconsize}{large}}{%
		\global\def\GLOBALinstitutioniconmargin {0.65em}%
		\global\def\GLOBALinstitutionicon {\raisebox{-0.25\baselineskip}{%
			\includegraphics[height=1.25\baselineskip,width=1.25\baselineskip]{#4}}%
		}%
	}{%
	\ifthenelse{\equal{\institutioniconsize}{huge}}{%
		\global\def\GLOBALinstitutioniconmargin {0.7em}%
		\global\def\GLOBALinstitutionicon {\raisebox{-0.5\baselineskip}{%
			\includegraphics[height=1.5\baselineskip,width=1.5\baselineskip]{#4}}%
		}%
	}{%
		\throwbadconfigondoc{Configuracion tamano icono incorrecto}{\institutioniconsize}{small,normal,large,huge}}}}%
	}%
	\begin{institution}[#1]{#2}{#3}%
}{%
	\end{institution}%
}

% Crea la firma del usuario
\newenvironment{signature}{%
	\vfill%
	\begin{flushright}%
		\begin{tabular}{c}%
			\includegraphics[width=\usersignaturesize cm]{\sign} \\%
		}{%
		\end{tabular}%
	\end{flushright}%
	\vspace{-1\baselineskip}%
}

% Itemize en negrita
%	#1	Parámetros opcionales
\newenvironment{itemizebf}[1][]{%
	\begin{itemize}[font=\bfseries,#1]%
	}{%
	\end{itemize}%
}

% Enumerate en negrita
%	#1	Parámetros opcionales
\newenvironment{enumeratebf}[1][]{%
	\begin{enumerate}[font=\bfseries,#1]%
	}{%
	\end{enumerate}%
}
