% Template:     Professional-CV
% Documento:    Configuraciones iniciales del documento
% Versión:      3.0.4 (08/09/2021)
% Codificación: UTF-8
%
% Autor: Pablo Pizarro R.
%        pablo@ppizarror.com
%
% Manual template: [https://latex.ppizarror.com/professional-cv]
% Licencia MIT:    [https://opensource.org/licenses/MIT]

% Se revisa si las variables no han sido borradas
\checkvardefined{\anonacimientoautor}
\checkvardefined{\correoautor}
\checkvardefined{\dianacimientoautor}
\checkvardefined{\mesnacimientoautor}
\checkvardefined{\nombreautor}
\checkvardefined{\telefonoautor}

% Se añade \xspace a las variables
\makeatletter
\g@addto@macro\anonacimientoautor\xspace
\g@addto@macro\dianacimientoautor\xspace
\g@addto@macro\mesnacimientoautor\xspace
\g@addto@macro\nombreautor\xspace
\g@addto@macro\telefonoautor\xspace
\makeatother

% Se define metadata del pdf
\ifthenelse{\equal{\cfgshowbookmarkmenu}{true}}{
	\def\cdfpagemodepdf {UseOutlines}
}{
	\def\cdfpagemodepdf {UseNone}
}
\hypersetup{
	keeppdfinfo,
	bookmarksopen={\cfgpdfbookmarkopen},
	bookmarksopenlevel={\cfgbookmarksopenlevel},
	bookmarkstype={toc},
	pdfauthor={\nombreautor},
	pdfcenterwindow={\cfgpdfcenterwindow},
	pdfcopyright={\headertitle, \nombreautor. Email: \correoautor. Tel: \telefonoautor},
	pdfcreator={LaTeX},
	pdfdisplaydoctitle={\cfgpdfdisplaydoctitle},
	pdffitwindow={\cfgpdffitwindow},
	pdfinfo={
		Autor.Email={\correoautor},
		Autor.Nombre={\nombreautor},
		Autor.Telefono={\telefonoautor},
		Template.Autor.Alias={ppizarror},
		Template.Autor.Email={pablo@ppizarror.com},
		Template.Autor.Nombre={Pablo Pizarro R.},
		Template.Autor.Web={https://ppizarror.com/},
		Template.Codificacion={UTF-8},
		Template.Fecha={01/09/2021},
		Template.Licencia.Tipo={MIT},
		Template.Licencia.Web={https://opensource.org/licenses/MIT/},
		Template.Nombre={Professional-CV},
		Template.Tipo={Normal},
		Template.Version.Dev={3.0.4},
		Template.Version.Hash={E6710708D3DCE33A6810CB7142AB59F3},
		Template.Version.Release={3.0.4},
		Template.Web.Dev={https://github.com/Template-Latex/Professional-CV/},
		Template.Web.Manual={https://latex.ppizarror.com/Professional-CV/}
	},
	pdfkeywords={CV, \nombreautor, \headertitle, \correoautor},
	pdfmenubar={\cfgpdfmenubar},
	pdfpagelayout={\cfgpdfpagemode},
	pdfpagemode={\cdfpagemodepdf},
	pdfproducer={Professional-CV v3.0.4 | (Pablo Pizarro R.) ppizarror.com},
	pdfremotestartview={Fit},
	pdfstartpage={1},
	pdfstartview={\cfgpdfpageview},
	pdfsubject={CV \nombreautor},
	pdftitle={\headertitle~ \nombreautor},
	pdftoolbar={\cfgpdftoolbar},
	pdftype={Text}
}

% Establece la carpeta de imágenes por defecto
\graphicspath{{./img}}

% Ajuste del entrelineado
\renewcommand{\baselinestretch}{\documentinterline}

% Ajuste de tablas
\settablecellpadding{0.5}{1}

% Configuración de los colores
\color{\maintextcolor} % Color principal
\arrayrulecolor{\tablelinecolor} % Color de las líneas de las tablas
\sethlcolor{\highlightcolor} % Color del subrayado por defecto
\ifthenelse{\equal{\showborderonlinks}{true}}{
	\hypersetup{
		% Color de links con borde
		citebordercolor=\numcitecolor,
		linkbordercolor=\urlcolor,
		urlbordercolor=\urlcolor
	}
}{
	\hypersetup{
		% Color de links sin borde
		hidelinks,
		colorlinks=true,
		citecolor=\numcitecolor,
		linkcolor=\urlcolor,
		urlcolor=\urlcolor
	}
}
\ifthenelse{\equal{\colorpage}{white}}{}{
	\pagecolor{\colorpage}
}

% Configuración de la identación
\newlength{\newparindent}
\addtolength{\newparindent}{\parindent}
\newlength{\doubleparindent}
\addtolength{\doubleparindent}{\parindent}
\addtolength{\doubleparindent}{\parindent}

% Configuración de hbox y vbox
\hfuzz=200pt
\vfuzz=200pt
\hbadness=\maxdimen
\vbadness=\maxdimen

% Se activa el word-wrap para textos con \texttt{}
\ttfamily \hyphenchar\the\font=`\-
\expandafter\def\expandafter\UrlBreaks\expandafter{\UrlBreaks%
	\do\a\do\b\do\c\do\d\do\e\do\f\do\g\do\h\do\i\do\j%
	\do\k\do\l\do\m\do\n\do\o\do\p\do\q\do\r\do\s\do\t%
	\do\u\do\v\do\w\do\x\do\y\do\z\do\A\do\B\do\C\do\D%
	\do\E\do\F\do\G\do\H\do\I\do\J\do\K\do\L\do\M\do\N%
	\do\O\do\P\do\Q\do\R\do\S\do\T\do\U\do\V\do\W\do\X%
	\do\Y\do\Z%
}

% Se define el tipo de texto de los url
\urlstyle{\fonturl}

% Se define la versión menor a compilar
\pdfminorversion=\pdfcompileversion

% Configura items
\AfterEndEnvironment{itemize}{\vspace{-0.4\baselineskip}}
\AfterEndEnvironment{enumerate}{\vspace{-0.4\baselineskip}}
