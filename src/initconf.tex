% Template:     Professional-CV
% Documento:    Configuraciones iniciales del documento
% Versión:      4.0.2 (05/11/2023)
% Codificación: UTF-8
%
% Autor: Pablo Pizarro R.
%        pablo@ppizarror.com
%
% Manual template: [https://latex.ppizarror.com/professional-cv]
% Licencia MIT:    [https://opensource.org/licenses/MIT]

% Se revisa si las variables no han sido borradas
\checkvardefined{\birthday}
\checkvardefined{\birthmonth}
\checkvardefined{\birthyear}
\checkvardefined{\email}
\checkvardefined{\name}
\checkvardefined{\phonenumber}

% Se añade \xspace a las variables
\makeatletter
	\g@addto@macro\birthday\xspace
	\g@addto@macro\birthmonth\xspace
	\g@addto@macro\birthyear\xspace
	\g@addto@macro\name\xspace
	\g@addto@macro\phonenumber\xspace
\makeatother

% Crea el título del documento
\ifthenelse{\equal{\writetitleheader}{true}}{%
	\def\cvdoctitle {\headertitle~ \name}
}{
	\def\cvdoctitle {\name}
}%

% Se define metadata del pdf
\ifthenelse{\equal{\cfgshowbookmarkmenu}{true}}{
	\def\cdfpagemodepdf {UseOutlines}
}{
	\def\cdfpagemodepdf {UseNone}
}
\hypersetup{
	keeppdfinfo,
	bookmarksopen={\cfgpdfbookmarkopen},
	bookmarksopenlevel={\cfgbookmarksopenlevel},
	bookmarkstype={toc},
	pdfauthor={\name},
	pdfcenterwindow={\cfgpdfcenterwindow},
	pdfcopyright={\headertitle, \name. Email: \email. Phone: \phonenumber},
	pdfcreator={LaTeX},
	pdfdisplaydoctitle={\cfgpdfdisplaydoctitle},
	pdffitwindow={\cfgpdffitwindow},
	pdfinfo={
		Author.Email={\email},
		Author.Name={\name},
		Author.Phone={\phonenumber},
		Template.Author.Alias={ppizarror},
		Template.Author.Email={pablo@ppizarror.com},
		Template.Author.Name={Pablo Pizarro R.},
		Template.Author.Web={https://ppizarror.com/},
		Template.Date={05/11/2023},
		Template.Encoding={UTF-8},
		Template.License.Type={MIT},
		Template.License.Web={https://opensource.org/licenses/MIT/},
		Template.Name={Professional-CV},
		Template.Type={Normal},
		Template.Version.Dev={4.0.2-9},
		Template.Version.Hash={7F46C0E51C0CD882764DA42EC4F0DEA9},
		Template.Version.Release={4.0.2},
		Template.Web.Dev={https://github.com/Template-Latex/Professional-CV/},
		Template.Web.Manual={https://latex.ppizarror.com/Professional-CV/}
	},
	pdfkeywords={CV, \name, \headertitle, \email},
	pdfmenubar={\cfgpdfmenubar},
	pdfpagelayout={\cfgpdfpagemode},
	pdfpagemode={\cdfpagemodepdf},
	pdfproducer={Professional-CV v4.0.2 | (Pablo Pizarro R.) ppizarror.com},
	pdfremotestartview={Fit},
	pdfstartpage={1},
	pdfstartview={\cfgpdfpageview},
	pdfsubject={CV \name},
	pdftitle={\cvdoctitle},
	pdftoolbar={\cfgpdftoolbar},
	pdftype={Text}
}

% Establece la carpeta de imágenes por defecto
\graphicspath{{./img}}

% Ajuste del entrelineado
\renewcommand{\baselinestretch}{\documentinterline}

% Ajuste de tablas
\setlength{\tabcolsep}{0.5em} % Horizontal
\def\arraystretch {1} % Vertical

% Configuración de los colores
\color{\maintextcolor} % Color principal
\arrayrulecolor{\tablelinecolor} % Color de las líneas de las tablas
\sethlcolor{\highlightcolor} % Color del subrayado por defecto
\ifthenelse{\equal{\showborderonlinks}{true}}{
	\hypersetup{
		% Color de links con borde
		citebordercolor=\numcitecolor,
		linkbordercolor=\urlcolor,
		urlbordercolor=\urlcolor
	}
}{
	\hypersetup{
		% Color de links sin borde
		hidelinks,
		colorlinks=true,
		citecolor=\numcitecolor,
		filecolor=\urlcolor,
		linkcolor=\urlcolor,
		urlcolor=\urlcolor
	}
}
\ifthenelse{\equal{\colorpage}{white}}{}{
	\pagecolor{\colorpage}
}

% Configuración de la identación
\newlength{\newparindent}
\addtolength{\newparindent}{\parindent}
\newlength{\doubleparindent}
\addtolength{\doubleparindent}{\parindent}
\addtolength{\doubleparindent}{\parindent}
\setenumerate{itemsep=0pt}
\setitemize{itemsep=0pt}

% Configuración de hbox y vbox
\hfuzz=200pt
\vfuzz=200pt
\hbadness=\maxdimen
\vbadness=\maxdimen

% Se activa el word-wrap para textos con \texttt{}
\ttfamily \hyphenchar\the\font=`\-

% Se define el tipo de texto de los url
\urlstyle{\fonturl}

% Se define la versión menor a compilar
\ifthenelse{\equal{\compilertype}{pdf2latex}}{
	% Nivel de compresión
	\pdfcompresslevel=9
	
	% El óptimo es 2, según
	% https://texdoc.org/serve/pdftex-a.pdf/0 p.20
	\pdfdecimaldigits=2
	
	% Inclusión de PDF
	\pdfinclusionerrorlevel=0
	
	% Versión
	\pdfminorversion=\pdfcompileversion
	
	% Compresión de objetos
	\pdfobjcompresslevel=2
}{}

% Configura items
\AfterEndEnvironment{itemize}{\vspace{-0.4\baselineskip}}
\AfterEndEnvironment{enumerate}{\vspace{-0.4\baselineskip}}
