% Template:     Professional-CV
% Documento:    Núcleo del template
% Versión:      3.0.0 (22/08/2021)
% Codificación: UTF-8
%
% Autor: Pablo Pizarro R.
%        Facultad de Ciencias Físicas y Matemáticas
%        Universidad de Chile
%        pablo@ppizarror.com
%
% Sitio web:    [https://latex.ppizarror.com/professional-cv]
% Licencia MIT: [https://opensource.org/licenses/MIT]

% -----------------------------------------------------------------------------
% CONFIGURACIONES
% -----------------------------------------------------------------------------
% Template:     Professional-CV
% Documento:    Configuraciones del template
% Versión:      3.3.9 (30/10/2023)
% Codificación: UTF-8
%
% Autor: Pablo Pizarro R.
%        pablo@ppizarror.com
%
% Manual template: [https://latex.ppizarror.com/professional-cv]
% Licencia MIT:    [https://opensource.org/licenses/MIT]

% CONFIGURACIONES GENERALES
\def\documentfontsize {11}        % Tamaño de la fuente del documento [pt]
\def\documentinterline {1.1}      % Interlineado del documento defecto [factor]
\def\fontdocument {default}       % Tipografía base, ver soportadas en manual
\def\fonttypewriter {cmodern}     % Tipografía de \texttt, ver manual
\def\fonturl {same}               % Tipo de fuente url {tt,sf,rm,same}
\def\pointdecimal {false}         % Decimales con punto en vez de coma

% CONFIGURACIÓN DEL ESTILO DEL TEMPLATE
\def\dateseparator {--}           % Carácter que separa las fechas
\def\headeritemsmargin {0.3}      % Margen vertical de los ítems del header [cm]
\def\headeritemslinespace {0.1}   % Distancia entre líneas del header [cm]
\def\headerseparator {\sbullet}   % Separador entre elementos del encabezado
\def\headertitlecentered {false}  % Header centrado
\def\headertitlemargin {0.25}     % Margen de \headertitle [cm]
\def\hfstyle {style5}             % Estilos de encabezado y pie de pag (11 disp.)
\def\institutioniconleft {false}  % Íconos a la izquierda o derecha
\def\institutioniconsize {normal} % Tamaño ícono {small,normal,large,huge}
\def\institutiontitlebold {false} % Título institución en negrita
\def\personaltabledatapic {true}  % Muestra/Oculta la foto en tabla antecedentes
\def\showuserphotoborder {true}   % Muestra un recuadro en la foto de perfil
\def\userphotosize {3.5}          % Dimensiones de la foto del usuario [cm]
\def\usersignaturesize {5.0}      % Ancho de la foto de la firma [cm]
\def\writebirthdayheader {true}   % Escribe año nacimiento encabezado
\def\writelastchangeheader{false} % Escribe fecha último cambio en el encabezado
\def\writetitleheader {false}     % Escribe el \headertitle en el encabezado

% CONFIGURACIÓN DE LOS COLORES DEL DOCUMENTO
\def\colorpage {white}            % Color de la página
\def\cvblocklinecolor {gray!30}   % Color de las líneas divisor. de los bloques
\def\datecolor {gray!90}          % Color de fechas en entradas de instituciones
\def\headerseparatorcolor {black} % Color del separador del header
\def\headertextcolor {black}      % Color del texto en el header
\def\headerurlcolor {dark-blue}   % Color de los enlaces en el header
\def\hftextcolor {lgray}          % Color del texto en header-footer
\def\highlightcolor {yellow}      % Color del subrayado con \hl
\def\instentrytitlecolor {black}  % Color del título en entrada de instituciones
\def\instnamecolor {dark-blue}    % Color del nombre de las instituciones (+url)
\def\instregioncolor {black}      % Color de la región en las instituciones
\def\lastchangeheadercolor {gray} % Color de fecha último cambio en encabezado
\def\maintextcolor {black}        % Color principal del texto
\def\numcitecolor {black}         % Color del número de las referencias o citas
\def\otherentrytitlecolor {black} % Color título en entradas <otherentry>
\def\personaltblentcolor {black}  % Color de los títulos tabla de antecedentes
\def\showborderonlinks {false}    % Encierra los links por un recuadro de color
\def\tablelinecolor {black}       % Color de las líneas de las tablas
\def\titlecolor {black}           % Color de los títulos
\def\urlcolor {blue}              % Color enlaces web en el cuerpo del documento
\def\userphotobordercolor {black} % Color del borde de la foto del usuario

% ESTILO DE LOS TÍTULOS
\def\fontsizemaintitle {\huge}    % Tamaño título principal
\def\fontsizetitle {\Large}       % Tamaño título secciones
\def\stylemaintitle {\bfseries}   % Estilo título principal
\def\styletitle {\bfseries}       % Estilo título secciones

% MÁRGENES DE PÁGINA
\def\pagemarginbottom {1.9}       % Margen inferior página [cm]
\def\pagemarginleft {1.9}         % Margen izquierdo página [cm]
\def\pagemarginright {1.9}        % Margen derecho página [cm]
\def\pagemargintop {1.5}          % Margen superior página [cm]

% OPCIONES DEL PDF COMPILADO
\def\cfgbookmarksopenlevel {1}    % Nivel de los marcadores (1: títulos)
\def\cfgpdfbookmarkopen {false}   % Expande marcadores hasta el nivel config.
\def\cfgpdfcenterwindow {true}    % Centra la ventana del lector al abrir el pdf
\def\cfgpdfdisplaydoctitle {true} % Muestra nombre autor como título del pdf
\def\cfgpdffitwindow {false}      % Ajusta ventana del lector al tamaño del pdf
\def\cfgpdfmenubar {true}         % Muestra el menú del lector al abrir el pdf
\def\cfgpdfpagemode {OneColumn}   % Modo de página pdf (OneColumn,SinglePage)
\def\cfgpdfpageview {FitH}        % Fit,FitH,FitV,FitR,FitB,FitBH,FitBV
\def\cfgpdftoolbar {true}         % Muestra la barra de herramientas en lector
\def\cfgshowbookmarkmenu {false}  % Muestra menú de marcadores al abrir el pdf
\def\pdfcompileversion {7}        % Versión mínima del pdf a compilar (1.x)

% NOMBRES DE LOS OBJETOS
\def\headertitle {Curriculum Vitae}   % Título principal en el header
\def\nomlastchange {Último cambio el} % Nombre apartado último cambio
\def\nompersonaldata {Antecedentes Personales} % Título antecedentes personales
\def\nomsummary {Descripción}         % Título sección descripción
\def\present {presente}               % Fecha presente en <institutionitem>


% -----------------------------------------------------------------------------
% IMPORTACIÓN DE FUNCIONES Y LIBRERÍAS
% -----------------------------------------------------------------------------
% Parches de librerías
\let\counterwithout\relax
\let\counterwithin\relax
\let\underbar\relax
\let\underline\relax

% Si se desactiva el idioma
\def\unaccentedoperators {}
\def\decimalpoint {}
\def\bibname {}

% Parche de sectsty.sty
\makeatletter
\def\underline#1{\relax\ifmmode\@@underline{#1}\else $\@@underline{\hbox{#1}}\m@th$\relax\fi}
\def\underbar#1{\underline{\sbox\tw@{#1}\dp\tw@\z@\box\tw@}}
\makeatother

% Tamaño de la fuente del documento
\usepackage{scrextend}
\usepackage{anyfontsize}
\changefontsizes{\documentfontsize pt}

% -----------------------------------------------------------------------------
% Librerías del núcleo
% -----------------------------------------------------------------------------
% Manejo de condicionales
\usepackage{ifthen}

% Idioma español, borrar esta línea si se desea usar otro idioma
\ifthenelse{\equal{\usespanishbabel}{true}}{
	\usepackage[spanish,es-nosectiondot,es-lcroman,es-noquoting]{babel}}{
\ifthenelse{\equal{\useenglishbabel}{true}}{
	\usepackage[english]{babel}}{}
}
\usepackage[utf8]{inputenc} % Codificación UTF-8

% Evita error "Too many alphabets used in version normal"
\newcommand\hmmax {0}
\newcommand\bmmax {0}

% -----------------------------------------------------------------------------
% Librerías independientes
% -----------------------------------------------------------------------------
\usepackage{amssymb}       % Librerías matemáticas
\usepackage{array}         % Nuevas características a las tablas
\usepackage{booktabs}      % Permite manejar elementos visuales en tablas
\usepackage{color}         % Colores
\usepackage{colortbl}      % Administración de color en tablas
\usepackage{csquotes}      % Citas y comillas
\usepackage{enumitem}      % Permite enumerar ítems
\usepackage{ifxetex}       % Detecta compilador
\usepackage{geometry}      % Dimensiones y geometría del documento
\usepackage{graphicx}      % Propiedades extra para los gráficos
\usepackage{lipsum}        % Permite crear textos dummy
\usepackage{mdwlist}       % Listas con encabezado
\usepackage{multicol}      % Múltiples columnas
\usepackage{needspace}     % Maneja los espacios en página
\usepackage{ragged2e}      % Mejora posicionado
\usepackage{relsize}       % Escalado avanzado
\usepackage{sectsty}       % Cambia el estilo de los títulos
\usepackage{selinput}      % Compatibilidad con acentos
\usepackage{setspace}      % Cambia el espacio entre líneas
\usepackage{soul}          % Permite subrayar texto
\usepackage{textcomp}      % Simbología común
\usepackage{url}           % Permite añadir enlaces
\usepackage{wasysym}       % Contiene caracteres misceláneos
\usepackage{wrapfig}       % Permite comprimir imágenes
\usepackage{xcolor}        % Administración de color avanzado
\usepackage{xspace}        % Administra espacios en párrafos y líneas

% -----------------------------------------------------------------------------
% Librerías con parámetros
% -----------------------------------------------------------------------------
\usepackage[pdfencoding=auto,psdextra]{hyperref} % Enlaces, referencias
\usepackage[final]{pdfpages} % Permite administrar páginas en pdf

% -----------------------------------------------------------------------------
% Librerías dependientes
% -----------------------------------------------------------------------------
\usepackage{bookmark}      % Administración de marcadores en pdf
\usepackage{fancyhdr}      % Encabezados y pie de páginas
\usepackage{float}         % Administrador de posiciones de objetos
\usepackage{hyperxmp}      % Etiquetas opcionales para el pdf compilado
\usepackage{multirow}      % Agrega nuevas opciones a las tablas

% -----------------------------------------------------------------------------
% Tipografía del documento
% -----------------------------------------------------------------------------
% Tipografías clásicas
\ifthenelse{\equal{\fontdocument}{default}}{
	\ifxetex
		\usepackage{fontspec}
		\setmainfont
		[ExternalLocation,
		Mapping=tex-text,
		Numbers=OldStyle,
		Ligatures={Common,Contextual},
		BoldFont=texgyrepagella-bold.otf,
		ItalicFont=texgyrepagella-italic.otf,
		BoldItalicFont=texgyrepagella-bolditalic.otf]
		{texgyrepagella-regular.otf}
		\usepackage[protrusion]{microtype}
	\else
		\usepackage{tgpagella}
		\usepackage[expansion,protrusion]{microtype}
	\fi
}{
\ifthenelse{\equal{\fontdocument}{lmodern}}{
	\usepackage{lmodern}
}{
\ifthenelse{\equal{\fontdocument}{arial}}{
	\usepackage{helvet}
	\renewcommand{\familydefault}{\sfdefault}
}{
\ifthenelse{\equal{\fontdocument}{arial2}}{
	\usepackage{arial}
}{
\ifthenelse{\equal{\fontdocument}{times}}{
	\usepackage{mathptmx}
}{
\ifthenelse{\equal{\fontdocument}{mathptmx}}{
	\usepackage{mathptmx}
}{
\ifthenelse{\equal{\fontdocument}{helvet}}{
	\renewcommand{\familydefault}{\sfdefault}
	\usepackage[scaled=0.95]{helvet}
	\usepackage[helvet]{sfmath}
}{
\ifthenelse{\equal{\fontdocument}{opensans}}{
	\usepackage[default,scale=0.95]{opensans}
}{
\ifthenelse{\equal{\fontdocument}{mathpazo}}{
	\usepackage{mathpazo}
}{
\ifthenelse{\equal{\fontdocument}{cambria}}{
	\usepackage{caladea}
}{
\ifthenelse{\equal{\fontdocument}{libertine}}{
	\usepackage[libertine]{newtxmath}
	\usepackage[tt=false]{libertine}
}{
\ifthenelse{\equal{\fontdocument}{custom}}{
}{

% Otros (último: fbb el 08/08/2021 - https://tug.org/FontCatalogue/seriffonts.html)
\ifthenelse{\equal{\fontdocument}{accanthis}}{
	\usepackage{accanthis}
}{
\ifthenelse{\equal{\fontdocument}{alegreya}}{
	\usepackage{Alegreya}
	\renewcommand*\oldstylenums[1]{{\AlegreyaOsF #1}}
}{
\ifthenelse{\equal{\fontdocument}{alegreyasans}}{
	\usepackage[sfdefault]{AlegreyaSans}
	\renewcommand*\oldstylenums[1]{{\AlegreyaSansOsF #1}}
}{
\ifthenelse{\equal{\fontdocument}{algolrevived}}{
	\usepackage{algolrevived}
}{
\ifthenelse{\equal{\fontdocument}{almendra}}{
	\usepackage{almendra}
}{
\ifthenelse{\equal{\fontdocument}{antpolt}}{
	\usepackage{antpolt}
}{
\ifthenelse{\equal{\fontdocument}{antpoltlight}}{
	\usepackage[light]{antpolt}
}{
\ifthenelse{\equal{\fontdocument}{anttor}}{
	\usepackage[math]{anttor}
}{
\ifthenelse{\equal{\fontdocument}{anttorcondensed}}{
	\usepackage[condensed,math]{anttor}
}{
\ifthenelse{\equal{\fontdocument}{anttorlight}}{
	\usepackage[light,math]{anttor}
}{
\ifthenelse{\equal{\fontdocument}{anttorlightcondensed}}{
	\usepackage[light,condensed,math]{anttor}
}{
\ifthenelse{\equal{\fontdocument}{arev}}{
	\let\quarternote\relax
	\let\eighthnote\relax
	\usepackage{arev}
}{
\ifthenelse{\equal{\fontdocument}{arimo}}{
	\usepackage[sfdefault]{arimo}
	\renewcommand*\familydefault{\sfdefault}
}{
\ifthenelse{\equal{\fontdocument}{arvo}}{
	\usepackage{Arvo}
}{
\ifthenelse{\equal{\fontdocument}{baskervald}}{
	\usepackage{baskervald}
}{
\ifthenelse{\equal{\fontdocument}{baskervaldx}}{
	\usepackage[lf]{Baskervaldx}
	\usepackage[bigdelims,vvarbb]{newtxmath}
	\usepackage[cal=boondoxo]{mathalfa}
	\renewcommand*\oldstylenums[1]{\textosf{#1}}
}{
\ifthenelse{\equal{\fontdocument}{berasans}}{
	\usepackage[scaled]{berasans}
	\renewcommand*\familydefault{\sfdefault}
}{
\ifthenelse{\equal{\fontdocument}{beraserif}}{
	\usepackage{bera}
}{
\ifthenelse{\equal{\fontdocument}{biolinum}}{
	\usepackage{libertine}
	\renewcommand*\familydefault{\sfdefault}
}{
\ifthenelse{\equal{\fontdocument}{bitter}}{
	\usepackage{bitter}
}{
\ifthenelse{\equal{\fontdocument}{boisik}}{
	\let\div\relax
	\usepackage{boisik}
}{
\ifthenelse{\equal{\fontdocument}{bookman}}{
	\usepackage{bookman}
}{
\ifthenelse{\equal{\fontdocument}{cabin}}{
	\usepackage[sfdefault]{cabin}
	\renewcommand*\familydefault{\sfdefault}
}{
\ifthenelse{\equal{\fontdocument}{cabincondensed}}{
	\usepackage[sfdefault,condensed]{cabin}
	\renewcommand*\familydefault{\sfdefault}
}{
\ifthenelse{\equal{\fontdocument}{caladea}}{
	\usepackage{caladea}
}{
\ifthenelse{\equal{\fontdocument}{cantarell}}{
	\usepackage[default]{cantarell}
}{
\ifthenelse{\equal{\fontdocument}{carlito}}{
	\usepackage[sfdefault]{carlito}
	\renewcommand*\familydefault{\sfdefault}
}{
\ifthenelse{\equal{\fontdocument}{charterbt}}{
	\usepackage[bitstream-charter]{mathdesign}
}{
\ifthenelse{\equal{\fontdocument}{chivolight}}{
	\usepackage[familydefault,light]{Chivo}
}{
\ifthenelse{\equal{\fontdocument}{chivoregular}}{
	\usepackage[familydefault,regular]{Chivo}
}{
\ifthenelse{\equal{\fontdocument}{clara}}{
	\usepackage{clara}
}{
\ifthenelse{\equal{\fontdocument}{clearsans}}{
	\usepackage[sfdefault]{ClearSans}
	\renewcommand*\familydefault{\sfdefault}
}{
\ifthenelse{\equal{\fontdocument}{cochineal}}{
	\usepackage{cochineal}
}{
\ifthenelse{\equal{\fontdocument}{coelacanth}}{
	\usepackage[nf]{coelacanth}
	\let\oldnormalfont\normalfont
	\def\normalfont{\oldnormalfont\mdseries}
}{
\ifthenelse{\equal{\fontdocument}{coelacanthextralight}}{
	\usepackage[el,nf]{coelacanth}
	\let\oldnormalfont\normalfont
	\def\normalfont{\oldnormalfont\mdseries}
}{
\ifthenelse{\equal{\fontdocument}{coelacanthlight}}{
	\usepackage[l,nf]{coelacanth}
	\let\oldnormalfont\normalfont
	\def\normalfont{\oldnormalfont\mdseries}
}{
\ifthenelse{\equal{\fontdocument}{comfortaa}}{
	\usepackage[default]{comfortaa}
}{
\ifthenelse{\equal{\fontdocument}{comicneue}}{
	\usepackage[default]{comicneue}
}{
\ifthenelse{\equal{\fontdocument}{comicneueangular}}{
	\usepackage[default,angular]{comicneue}
}{
\ifthenelse{\equal{\fontdocument}{computerconcrete}}{
	\usepackage{concmath}
}{
\ifthenelse{\equal{\fontdocument}{computerconcreteeuler}}{
	\let\Re\relax
	\let\Im\relax
	\usepackage{beton}
	\usepackage{euler}
}{
\ifthenelse{\equal{\fontdocument}{computermodern}}{
}{
\ifthenelse{\equal{\fontdocument}{computermodernbright}}{
	\usepackage{cmbright}
}{
\ifthenelse{\equal{\fontdocument}{crimson}}{
	\usepackage{crimson}
}{
\ifthenelse{\equal{\fontdocument}{crimsonpro}}{
	\usepackage{CrimsonPro}
	\let\oldnormalfont\normalfont
	\def\normalfont{\oldnormalfont\mdseries}
}{
\ifthenelse{\equal{\fontdocument}{crimsonproextralight}}{
	\usepackage[extralight]{CrimsonPro}
	\let\oldnormalfont\normalfont
	\def\normalfont{\oldnormalfont\mdseries}
}{
\ifthenelse{\equal{\fontdocument}{crimsonprolight}}{
	\usepackage[light]{CrimsonPro}
	\let\oldnormalfont\normalfont
	\def\normalfont{\oldnormalfont\mdseries}
}{
\ifthenelse{\equal{\fontdocument}{crimsonpromedium}}{
	\usepackage[medium]{CrimsonPro}
	\let\oldnormalfont\normalfont
	\def\normalfont{\oldnormalfont\mdseries}
}{
\ifthenelse{\equal{\fontdocument}{cyklop}}{
	\usepackage{cyklop}
}{
\ifthenelse{\equal{\fontdocument}{dejavusans}}{
	\usepackage{DejaVuSans}
	\renewcommand*\familydefault{\sfdefault}
}{
\ifthenelse{\equal{\fontdocument}{dejavusanscondensed}}{
	\usepackage{DejaVuSansCondensed}
	\renewcommand*\familydefault{\sfdefault}
}{
\ifthenelse{\equal{\fontdocument}{domitian}}{
	\usepackage{mathpazo}
	\usepackage{domitian}
	\let\oldstylenums\oldstyle
}{
\ifthenelse{\equal{\fontdocument}{droidsans}}{
	\usepackage[defaultsans]{droidsans}
	\renewcommand*\familydefault{\sfdefault}
}{
\ifthenelse{\equal{\fontdocument}{electrum}}{
	\usepackage[lf]{electrum}
}{	
\ifthenelse{\equal{\fontdocument}{erewhon}}{
	\usepackage[proportional,scaled=1.064]{erewhon}
	\usepackage[erewhon,vvarbb,bigdelims]{newtxmath}
	\renewcommand*\oldstylenums[1]{\textosf{#1}}
}{
\ifthenelse{\equal{\fontdocument}{fbb}}{
	\usepackage{fbb}
}{
\ifthenelse{\equal{\fontdocument}{fetamont}}{
	\usepackage{fetamont}
	\renewcommand*\familydefault{\sfdefault}
}{
\ifthenelse{\equal{\fontdocument}{firasans}}{
	\usepackage[sfdefault]{FiraSans}
	\renewcommand*\familydefault{\sfdefault}
}{
\ifthenelse{\equal{\fontdocument}{firasansnewtxsf}}{
	\usepackage[sfdefault]{FiraSans}
	\usepackage{newtxsf}
}{
\ifthenelse{\equal{\fontdocument}{fourier}}{
	\usepackage{fourier}
}{
\ifthenelse{\equal{\fontdocument}{fouriernc}}{
	\usepackage{fouriernc}
}{
\ifthenelse{\equal{\fontdocument}{gfsartemisia}}{
	\let\textlozenge\relax
	\usepackage{gfsartemisia}
}{
\ifthenelse{\equal{\fontdocument}{gfsartemisiaeuler}}{
	\let\textlozenge\relax
	\let\Re\relax
	\let\Im\relax
	\usepackage{gfsartemisia-euler}
}{
\ifthenelse{\equal{\fontdocument}{heuristica}}{
	\usepackage{heuristica}
	\usepackage[heuristica,vvarbb,bigdelims]{newtxmath}
	\renewcommand*\oldstylenums[1]{\textosf{#1}}
}{
\ifthenelse{\equal{\fontdocument}{iwona}}{
	\usepackage[math]{iwona}
}{
\ifthenelse{\equal{\fontdocument}{iwonacondensed}}{
	\usepackage[condensed,math]{iwona}
}{
\ifthenelse{\equal{\fontdocument}{iwonalight}}{
	\usepackage[light,math]{iwona}
}{
\ifthenelse{\equal{\fontdocument}{iwonalightcondensed}}{
	\usepackage[light,condensed,math]{iwona}
}{
\ifthenelse{\equal{\fontdocument}{kerkis}}{
	\usepackage{kmath,kerkis}
}{
\ifthenelse{\equal{\fontdocument}{kurier}}{
	\usepackage[math]{kurier}
}{
\ifthenelse{\equal{\fontdocument}{kuriercondensed}}{
	\usepackage[condensed,math]{kurier}
}{
\ifthenelse{\equal{\fontdocument}{kurierlight}}{
	\usepackage[light,math]{kurier}
}{
\ifthenelse{\equal{\fontdocument}{kurierlightcondensed}}{
	\usepackage[light,condensed,math]{kurier}
}{
\ifthenelse{\equal{\fontdocument}{lato}}{
	\usepackage[default]{lato}
}{
\ifthenelse{\equal{\fontdocument}{libertinus}}{
	\usepackage{libertinus}
}{
\ifthenelse{\equal{\fontdocument}{librebaskerville}}{
	\usepackage{librebaskerville}
}{
\ifthenelse{\equal{\fontdocument}{librebodoni}}{
	\usepackage{LibreBodoni}
}{
\ifthenelse{\equal{\fontdocument}{librecaslon}}{
	\usepackage{librecaslon}
}{
\ifthenelse{\equal{\fontdocument}{libris}}{
	\usepackage{libris}
	\renewcommand*\familydefault{\sfdefault}
}{
\ifthenelse{\equal{\fontdocument}{lxfonts}}{
	\usepackage{lxfonts}
}{
\ifthenelse{\equal{\fontdocument}{merriweather}}{
	\usepackage[sfdefault]{merriweather}
}{
\ifthenelse{\equal{\fontdocument}{merriweatherlight}}{
	\usepackage[sfdefault,light]{merriweather}
}{
\ifthenelse{\equal{\fontdocument}{mintspirit}}{
	\usepackage[default]{mintspirit}
}{
\ifthenelse{\equal{\fontdocument}{mlmodern}}{
	\usepackage{mlmodern}
}{
\ifthenelse{\equal{\fontdocument}{montserratalternatesextralight}}{
	\usepackage[defaultfam,extralight,tabular,lining,alternates]{montserrat}
	\renewcommand*\oldstylenums[1]{{\fontfamily{Montserrat-TOsF}\selectfont #1}}
}{
\ifthenelse{\equal{\fontdocument}{montserratalternatesregular}}{
	\usepackage[defaultfam,tabular,lining,alternates]{montserrat}
	\renewcommand*\oldstylenums[1]{{\fontfamily{Montserrat-TOsF}\selectfont #1}}
}{
\ifthenelse{\equal{\fontdocument}{montserratalternatesthin}}{
	\usepackage[defaultfam,thin,tabular,lining,alternates]{montserrat}
	\renewcommand*\oldstylenums[1]{{\fontfamily{Montserrat-TOsF}\selectfont #1}}
}{
\ifthenelse{\equal{\fontdocument}{montserratextralight}}{
	\usepackage[defaultfam,extralight,tabular,lining]{montserrat}
	\renewcommand*\oldstylenums[1]{{\fontfamily{Montserrat-TOsF}\selectfont #1}}
}{
\ifthenelse{\equal{\fontdocument}{montserratlight}}{
	\usepackage[defaultfam,light,tabular,lining]{montserrat}
	\renewcommand*\oldstylenums[1]{{\fontfamily{Montserrat-TOsF}\selectfont #1}}
}{
\ifthenelse{\equal{\fontdocument}{montserratregular}}{
	\usepackage[defaultfam,tabular,lining]{montserrat}
	\renewcommand*\oldstylenums[1]{{\fontfamily{Montserrat-TOsF}\selectfont #1}}
}{
\ifthenelse{\equal{\fontdocument}{montserratthin}}{
	\usepackage[defaultfam,thin,tabular,lining]{montserrat}
	\renewcommand*\oldstylenums[1]{{\fontfamily{Montserrat-TOsF}\selectfont #1}}
}{
\ifthenelse{\equal{\fontdocument}{newpx}}{
	\usepackage{newpxtext,newpxmath}
}{
\ifthenelse{\equal{\fontdocument}{nimbussans}}{
	\usepackage{nimbussans}
	\renewcommand*\familydefault{\sfdefault}
}{
\ifthenelse{\equal{\fontdocument}{noto}}{
	\usepackage[sfdefault]{noto}
	\renewcommand*\familydefault{\sfdefault}
}{
\ifthenelse{\equal{\fontdocument}{notoserif}}{
	\usepackage{notomath}
}{
\ifthenelse{\equal{\fontdocument}{opensansserif}}{
	\usepackage[default,oldstyle,scale=0.95]{opensans}
}{
\ifthenelse{\equal{\fontdocument}{overlock}}{
	\usepackage[sfdefault]{overlock}
	\renewcommand*\familydefault{\sfdefault}
}{
\ifthenelse{\equal{\fontdocument}{paratype}}{
	\usepackage{paratype}
	\renewcommand*\familydefault{\sfdefault}
}{
\ifthenelse{\equal{\fontdocument}{paratypesanscaption}}{
	\usepackage{PTSansCaption}
	\renewcommand*\familydefault{\sfdefault}
}{
\ifthenelse{\equal{\fontdocument}{paratypesansnarrow}}{
	\usepackage{PTSansNarrow}
	\renewcommand*\familydefault{\sfdefault}
}{
\ifthenelse{\equal{\fontdocument}{pxfonts}}{
	\usepackage{pxfonts}
}{
\ifthenelse{\equal{\fontdocument}{quattrocento}}{
	\usepackage[sfdefault]{quattrocento}
}{
\ifthenelse{\equal{\fontdocument}{raleway}}{
	\usepackage[default]{raleway}
}{
\ifthenelse{\equal{\fontdocument}{roboto}}{
	\usepackage[sfdefault]{roboto}
}{
\ifthenelse{\equal{\fontdocument}{robotocondensed}}{
	\usepackage[sfdefault,condensed]{roboto}
}{
\ifthenelse{\equal{\fontdocument}{robotolight}}{
	\usepackage[sfdefault,light]{roboto}
}{
\ifthenelse{\equal{\fontdocument}{robotolightcondensed}}{
	\usepackage[sfdefault,light,condensed]{roboto}
}{
\ifthenelse{\equal{\fontdocument}{robotothin}}{
	\usepackage[sfdefault,thin]{roboto}
}{
\ifthenelse{\equal{\fontdocument}{rosario}}{
	\usepackage[familydefault]{Rosario}
}{
\ifthenelse{\equal{\fontdocument}{sourcesanspro}}{
	\usepackage[default]{sourcesanspro}
}{
\ifthenelse{\equal{\fontdocument}{step}}{
	\usepackage[notext]{stix}
	\usepackage{step}
}{
\ifthenelse{\equal{\fontdocument}{stickstoo}}{
	\usepackage{stickstootext}
	\usepackage[stickstoo,vvarbb]{newtxmath}
}{
\ifthenelse{\equal{\fontdocument}{texgyrebonum}}{
	\usepackage{tgbonum}
}{
\ifthenelse{\equal{\fontdocument}{txfonts}}{
	\usepackage{txfonts}
}{
\ifthenelse{\equal{\fontdocument}{uarial}}{
	\usepackage{uarial}
	\renewcommand*\familydefault{\sfdefault}
}{
\ifthenelse{\equal{\fontdocument}{ugq}}{
	\renewcommand*\sfdefault{ugq}
	\renewcommand*\familydefault{\sfdefault}
}{
\ifthenelse{\equal{\fontdocument}{universalis}}{
	\usepackage[sfdefault]{universalis}
}{
\ifthenelse{\equal{\fontdocument}{universaliscondensed}}{
	\usepackage[condensed,sfdefault]{universalis}
}{
\ifthenelse{\equal{\fontdocument}{venturis}}{
	\usepackage[lf]{venturis}
	\renewcommand*\familydefault{\sfdefault}
}{
	\throwbadconfig[nostop]{Fuente desconocida}{\fontdocument}{(Fuentes recomendadas) default,lmodern,carial,arial2,times,mathptmx,helvet,opensans,mathpazo,cambria,libertine,custom}
	\throwbadconfig[noheader-nostop]{Fuente desconocida}{\fontdocument}{(Fuentes adicionales) accanthis,alegreya,alegreyasans,algolrevived,almendra,antpolt,antpoltlight,anttor,anttorcondensed,anttorlight,anttorlightcondensed,arev,arimo,arvo,baskervald,baskervaldx,berasans,beraserif,biolinum,bitter,boisik,bookman,cabin,cabincondensed,cantarell,caladea,carlito,charterbt,chivolight,chivoregular,clara,clearsans,cochineal,coelacanth,coelacanthextralight,coelacanthlight,comfortaa,comicneue,comicneueangular,computerconcrete,computerconcreteeuler,computermodern,computermodernbright,crimson,crimsonpro,crimsonproextralight,crimsonprolight,crimsonpromedium,cyklop}
	\throwbadconfig[noheader-nostop]{Fuente desconocida}{\fontdocument}{dejavusans,dejavusanscondensed,domitian,droidsans,electrum,erewhon,fbb,fetamont,firasans,firasansnewtxsf,fourier,fouriernc,gfsartemisia,gfsartemisiaeuler,heuristica,iwona,iwonacondensed,iwonalight,iwonalightcondensed,kerkis,kurier,kuriercondensed,kurierlight,kurierlightcondensed,lato,libertinus,librebaskerville,librebodoni,librecaslon,libris,lxfonts}
	\throwbadconfig[noheader]{Fuente desconocida}{\fontdocument}{merriweather,merriweatherlight,mintspirit,mlmodern,montserratalternatesextralight,montserratalternatesregular,montserratalternatesthin,montserratextralight,montserratlight,montserratregular,montserratthin,newpx,nimbussans,noto,notoserif,opensansserif,overlock,paratype,paratypesanscaption,paratypesansnarrow,pxfonts,quattrocento,raleway,roboto,robotolight,robotolightcondensed,robotothin,rosario,sourcesanspro,step,stickstoo,uarial,texgyrebonum,txfonts,ugq,universalis,universaliscondensed,venturis}
	}}}}}}}}}}}}}}}}}}}}}}}}}}}}}}}}}}}}}}}}}}}}}}}}}}}}}}}}}}}}}}}}}}}}}}}}}}}}}}}}}}}}}}}}}}}}}}}}}}}}}}}}}}}}}}}}}}}}}}}}}}}}}}}}}}
}

% -----------------------------------------------------------------------------
% Tipografía typewriter
% -----------------------------------------------------------------------------
% https://tug.org/FontCatalogue/typewriterfonts.html
\ifthenelse{\equal{\fonttypewriter}{custom}}{
}{
\ifthenelse{\equal{\fonttypewriter}{tmodern}}{
	\renewcommand*\ttdefault{lmvtt}
}{
\ifthenelse{\equal{\fonttypewriter}{anonymouspro}}{
	\usepackage[ttdefault=true]{AnonymousPro}
}{
\ifthenelse{\equal{\fonttypewriter}{ascii}}{
	\usepackage{ascii}
	\let\SI\relax
}{
\ifthenelse{\equal{\fonttypewriter}{beramono}}{
	\usepackage[scaled]{beramono}
}{
\ifthenelse{\equal{\fonttypewriter}{cascadiacode}}{
	\usepackage{cascadia-code}
}{
\ifthenelse{\equal{\fonttypewriter}{cmpica}}{
	\usepackage{addfont}
	\addfont{OT1}{cmpica}{\pica}
	\addfont{OT1}{cmpicab}{\picab}
	\addfont{OT1}{cmpicati}{\picati}
	\renewcommand*\ttdefault{pica}
}{
\ifthenelse{\equal{\fonttypewriter}{cmodern}}{
}{
\ifthenelse{\equal{\fonttypewriter}{courier}}{
	\usepackage{courier}
}{
\ifthenelse{\equal{\fonttypewriter}{courier10}}{
	\usepackage{courierten}
}{
\ifthenelse{\equal{\fonttypewriter}{cmvtt}}{
	\renewcommand*\ttdefault{cmvtt}
}{
\ifthenelse{\equal{\fonttypewriter}{dejavusansmono}}{
	\usepackage[scaled]{DejaVuSansMono}
}{
\ifthenelse{\equal{\fonttypewriter}{droidsansmono}}{
	\usepackage[defaultmono]{droidsansmono}
}{
\ifthenelse{\equal{\fonttypewriter}{firamono}}{
	\usepackage[scale=0.85]{FiraMono}
}{
\ifthenelse{\equal{\fonttypewriter}{gomono}}{
	\usepackage[scale=0.85]{GoMono}
}{
\ifthenelse{\equal{\fonttypewriter}{inconsolata}}{
	\usepackage{inconsolata}
}{
\ifthenelse{\equal{\fonttypewriter}{nimbusmono}}{
	\usepackage{nimbusmono}
}{
\ifthenelse{\equal{\fonttypewriter}{newtxtt}}{
	\usepackage[zerostyle=d]{newtxtt}
}{
\ifthenelse{\equal{\fonttypewriter}{nimbusmono}}{
	\usepackage{nimbusmono}
}{
\ifthenelse{\equal{\fonttypewriter}{nimbusmononarrow}}{
	\usepackage{nimbusmononarrow}
}{
\ifthenelse{\equal{\fonttypewriter}{lcmtt}}{
	\renewcommand*\ttdefault{lcmtt}
}{
\ifthenelse{\equal{\fonttypewriter}{sourcecodepro}}{
	\usepackage[ttdefault=true,scale=0.85]{sourcecodepro}
}{
\ifthenelse{\equal{\fonttypewriter}{texgyrecursor}}{
	\usepackage{tgcursor}
}{
\ifthenelse{\equal{\fonttypewriter}{txtt}}{
	\renewcommand*\ttdefault{txtt}
}{
	\throwbadconfig{Fuente desconocida}{\fonttypewriter}{custom,anonymouspro,ascii,beramono,cascadiacode,cmpica,cmodern,courier,courier10,cvmtt,dejavusansmono,droidsansmono,firamono,gomono,inconsolata,kpmonospaced,lcmtt,newtxtt,nimbusmono,nimbusmononarrow,texgyrecursor,tmodern,txtt}
	}}}}}}}}}}}}}}}}}}}}}}}
}

% -----------------------------------------------------------------------------
% Finales
% -----------------------------------------------------------------------------
\usepackage[T1]{fontenc} % Caracteres acentuados

% Definición de variables globales
\global\def\GLOBALhasinstitution {false}
\global\def\GLOBALheaderfaimport {false}
\global\def\GLOBALheaderlineitem {false}
\global\def\GLOBALheaderseparatorsticky {false}
\global\def\GLOBALinstitutionenabled {false}
\global\def\GLOBALinstitutionshortend {false}

\def\LOCALpercentchar#1{}
\edef\LOCALpercentchar{\expandafter\LOCALpercentchar\string\%}

% Archivo que guarda el código del header
\newwrite\fileheaderitems
\immediate\openout\fileheaderitems=\jobname.hitems
\AtBeginDocument{\immediate\closeout\fileheaderitems}                               

% Lanza un mensaje de error
% 	#1	Función del error
%	#2	Mensaje
\newcommand{\throwerror}[2]{%
	\errmessage{LaTeX Error: \noexpand#1 #2 (linea \the\inputlineno)}
	\stop
}

% Lanza un mensaje de advertencia
%	#1	Mensaje
\newcommand{\throwwarning}[1]{%
	\errmessage{LaTeX Warning: #1 (linea \the\inputlineno)}
}

% Lanza un mensaje de error indicando mala configuración
%	#1	Mensaje de error
% 	#2	Configuración usada
%	#3	Valores esperados
\newcommand{\throwbadconfig}[3]{%
	\errmessage{LaTeX Warning: #1 \noexpand #2=#2. Valores esperados: #3}
	\stop
}

% Lanza un mensaje de error indicando mala configuración dentro de begin{document}
%	#1	Mensaje de error
% 	#2	Configuración usada
%	#3	Valores esperados
\newcommand{\throwbadconfigondoc}[3]{%
	\errmessage{#1 \noexpand #2=#2. Valores esperados: #3}
	\stop
}

% Comprueba si una variable está definida
%	#1	Variable
\newcommand{\checkvardefined}[1]{%
	\ifthenelse{\isundefined{#1}}{%
		\errmessage{LaTeX Warning: Variable \noexpand#1 no definida}
		\stop
	}{}
}

% Comprueba si una variable está definida
%	#1	Variable
%	#2	Mensaje
\newcommand{\checkextravarexist}[2]{%
	\ifthenelse{\isundefined{#1}}{%
		\errmessage{LaTeX Warning: Variable \noexpand#1 no definida}
		\ifx\hfuzz#2\hfuzz%
			\errmessage{LaTeX Warning: Defina la variable en el bloque de INFORMACION DEL USUARIO al comienzo del archivo principal del Template}
		\else%
			\errmessage{LaTeX Warning: #2}
		\fi%
	}{}
}

% Lanza un mensaje de error si una variable no ha sido definida
% 	#1	Función del error
%	#2	Variable
%	#3	Mensaje
\newcommand{\emptyvarerr}[3]{
	\ifx\hfuzz#2\hfuzz%
		\errmessage{LaTeX Warning: \noexpand#1 #3 (linea \the\inputlineno)}
	\fi
}

% Cambia márgenes de las páginas [cm]
% 	#1	Margen izquierdo
%	#2	Margen superior
%	#3	Margen derecho
%	#4	Margen inferior
\newcommand{\setpagemargincm}[4]{%
	\newgeometry{left=#1cm, top=#2cm, right=#3cm, bottom=#4cm}%
}

% Cambia los márgenes del documento
%	#1	Margen izquierdo
%	#2	Margen derecho
\newcommand{\changemargin}[2]{%
	\emptyvarerr{\changemargin}{#1}{Margen izquierdo no definido}%
	\emptyvarerr{\changemargin}{#2}{Margen derecho no definido}%
	\list{}{\rightmargin#2\leftmargin#1}\item[]%
}
\let\endchangemargin=\endlist

% Agrega el separador entre items del header
\newcommand{\insertheaderseparator}{%
	\ifthenelse{\equal{\GLOBALheaderlineitem}{true}}{%
		\ifthenelse{\equal{\GLOBALheaderseparatorsticky}{false}}{%
			\immediate\write\fileheaderitems{\unexpanded{\textcolor{\headerseparatorcolor}{\headerseparator}}\LOCALpercentchar}
		}{}
	}{}
	\global\def\GLOBALheaderlineitem {true}
}

% Agrega un item de texto al header
%	#1	Texto
\newcommand{\addheadertext}[1]{%
	\global\def\GLOBALheaderseparatorsticky {false}
	\insertheaderseparator
	\immediate\write\fileheaderitems{\unexpanded{\textcolor{\headertextcolor}{#1}}\LOCALpercentchar}
}

% Agrega un correo al header
%	#1	Correo
\newcommand{\addheadermail}[1]{%
	\global\def\GLOBALheaderseparatorsticky {false}
	\insertheaderseparator
	\immediate\write\fileheaderitems{\unexpanded{\href{mailto:#1}{#1}}\LOCALpercentchar}
}

% Agrega un enlace al header
%	#1	Ícono de fontawesome, si está vacío se usa url normal
%	#2	Enlace
\newcommand{\addheaderlink}[2][]{%
	\ifthenelse{\equal{#1}{}}{ % Sin ícono
		\global\def\GLOBALheaderseparatorsticky {false}
		\insertheaderseparator
		\immediate\write\fileheaderitems{\unexpanded{\url{#2}}\LOCALpercentchar}
	}{
		\insertheaderseparator
		\ifthenelse{\equal{\GLOBALheaderfaimport}{false}}{%
			\global\def\GLOBALheaderfaimport {true}
			\usepackage{fontawesome5}
		}{}
		\ifthenelse{\equal{\GLOBALheaderseparatorsticky}{true}}{%
			\immediate\write\fileheaderitems{\unexpanded{\hspace{0.2cm}}\LOCALpercentchar}
		}{}
		\immediate\write\fileheaderitems{\unexpanded{\href{#2}{\faIcon{#1}}}\LOCALpercentchar}
		\global\def\GLOBALheaderseparatorsticky {true}
	}
}

% Agrega una nueva línea al header
\newcommand{\addheadernewline}{%
	\global\def\GLOBALheaderlineitem {false}
	\global\def\GLOBALheaderseparatorsticky {false}
	\immediate\write\fileheaderitems{\unexpanded{\\}\LOCALpercentchar}
}

% Escribe el encabezado del CV
\newcommand{\writeheader}{%
	% Escribe título
	\ifthenelse{\equal{\writetitleheader}{true}}{%
		\begin{center}%
			\fontsize{16}{12} \selectfont \vspace{0.3cm} \textcolor{\titlecolor}{\headertitle}
		\end{center}
		\vspace{0.1cm}
	}{}
	
	% Nombre
	\noindent {\fontsizemaintitle \stylemaintitle \textcolor{\titlecolor}{\nombreautor}}
	\ifthenelse{\equal{\writebirthdayheader}{true}}{%
		\quad \emph{\textcolor{\headertextcolor}{\mesnacimientoautor \dianacimientoautor, \anonacimientoautor}}%
	}{}%
	\ifthenelse{\equal{\writelastchangeheader}{true}}{%
		\hfill \quad {\scriptsize \textcolor{\lastchangeheadercolor}{\nomlastchange\ \today}}%
	}{}%
	\\ \vspace{-1em}%
	
	% Escribe items del header
	\vspace{\headeritemsmargin cm}
	{%
		\noindent%
		\input{\jobname.hitems}%
	}
}

% Título de las secciones principales
%	#1	Título
\newcommand*{\roottitle}[1]{%
	\emptyvarerr{\roottitle}{#1}{Titulo no definido}
	\section*{#1}\vspace{-0.3em}\nopagebreak[4]
	\addcontentsline{toc}{section}{#1}
}

% Crea un acrónimo de forma fácil
%	#1	Acrónimo
\newcommand*{\acr}[1]{%
	\textscale{.85}{#1}
}

% Crea un campo de texto
% 	#1	Texto
\newcommand{\bodytext}[1]{%
	\nopagebreak[4]%
	\begin{indentsection}%
		\item[] #1
	\end{indentsection}
	\pagebreak[2]
}

% Crea una entrada en la tabla de datos personales
%	#1	Nombre entrada
%	#2	Datos entrada
\newcommand{\personaltableentry}[2]{%
	\emptyvarerr{\personaltableentry}{#1}{Nombre entrada no definido}%
	\emptyvarerr{\personaltableentry}{#2}{Datos entrada no definidos}%
	\textcolor{\personaltblentcolor}{\textbf{#1:}} & #2 \\
}

% Inserta un objeto en un elemento institución
%	#1	Cargo
%	#2	Indica si escribe fechas o no
%	#3	Fecha inicial
%	#4	Fecha final
%	#5	Ancho título
%	#6	Ancho fecha
%	#7	Descripción
\newcommand{\newinstitutionentry}[7]{%
	\ifthenelse{\equal{\GLOBALinstitutionenabled}{true}}{}{%
		\throwwarning{Funciones \noexpand\newinstitutionentry, \noexpand\institutionentry o \noexpand\institutionentryshort no pueden usarse fuera del entorno \noexpand\institution}\stop%
	}%
	\emptyvarerr{\newinstitutionentry}{#1}{Cargo o posición no definido}
	\emptyvarerr{\newinstitutionentry}{#5}{Dimensiones de titulo no definido}
	\emptyvarerr{\newinstitutionentry}{#6}{Dimensiones de fecha no definido}
	\nopagebreak[4]
	\begin{indentsectiondouble}
		\item []
		\noindent
		\begin{minipage}{1\linewidth}
			\begin{minipage}[t][][t]{#5}%
				{\textcolor{\instentrytitlecolor}{\textbf{#1}} ~ \\ \vspace{0.7em}}
			\end{minipage}
			\hfill
			\begin{minipage}[t][][t]{#6}%
				\begin{flushright}
					\noindent \textcolor{\datecolor}{\emph{#2#4#3}}
				\end{flushright}
			\end{minipage}
		\end{minipage}
		\vspace{-1.1\baselineskip}
		\break
		\ifx\hfuzz#7\hfuzz
			\vspace*{-0.75\baselineskip}
			\global\def\GLOBALinstitutionshortend {true}%
		\else
			\begin{minipage}{1.0\linewidth}%
				\bodytext{#7}
				\par
			\end{minipage}\vspace{0.1\baselineskip}%
			\global\def\GLOBALinstitutionshortend {false}%
		\fi
	\end{indentsectiondouble}
}

% Entrada en institución
%	#1	Cargo
%	#2	Fecha inicial
%	#3	Fecha final
%	#4	Descripción
\newcommand{\institutionentry}[4]{%
	\newinstitutionentry{#1}{#2 }{ #3}{\dateseparator}{.687\linewidth}{.31\linewidth}{#4}%
}

% Entrada en institución con fecha corta
%	#1	Cargo
%	#2	Fecha inicial
%	#3	Fecha final
%	#4	Descripción
\newcommand{\institutionentryshort}[4]{%
	\newinstitutionentry{#1}{#2 }{ #3}{\dateseparator}{.833\linewidth}{.163\linewidth}{#4}%
}

% Entrada en institución sin fecha
%	#1	Cargo
%	#2	Descripción
\newcommand{\institutionentrynodate}[2]{%
	\newinstitutionentry{#1}{}{}{}{\linewidth}{0\linewidth}{#2}%
}

% Otro tipo de entrada en cvblock
%	#1	Título
%	#2	Contenido
\newcommand{\otherentry}[2]{%
	\begin{basedescript}%
		{\setlength{\leftmargin}{\doubleparindent}}%
		\item[\hspace{\newparindent}\textcolor{\otherentrytitlecolor}{\textbf{#1}}] #2%
	\end{basedescript}
	\vspace{-0.6cm}
}

% Inserta un elemento en el bloque de firma
%	#1	Elemento en la firma
\newcommand{\signatureentry}[1]{%
	\texttt{\MakeUppercase{#1}} \\
}

% Crea un bull
\newcommand{\sbullet}{%
	\ \ \raisebox{0.11em}[-1em][-1em]{\tiny $\bullet$} \ \ %
}

% Variaciones de vspace
\newcommand{\breakvspace}[1]{%
	\pagebreak[2] \vspace{#1} \pagebreak[2]%
}
\newcommand{\nobreakvspace}[1]{%
	\nopagebreak[4] \vspace{#1} \nopagebreak[4]%
}

% Apóstrofe
\newcommand{\apo}{%
	\raisebox{-.18ex}{'}{\hspace{-.1em}}%
}

% Inserta un texto entre comillas
\newcommand{\quotes}[1]{%
	\enquote*{#1}%
}

% Inserta un texto con el formato de enlace
% 	#1 	Enlace
\newcommand{\hreftext}[1]{%
	\ifthenelse{\equal{\fonturl}{same}}{%
		#1%
	}{%
	\ifthenelse{\equal{\fonturl}{tt}}{%
		\texttt{#1}%
	}{%
	\ifthenelse{\equal{\fonturl}{rm}}{%
		\textrm{#1}%
	}{%
	\ifthenelse{\equal{\fonturl}{sf}}{%
		\textsf{#1}%
	}{}}}}%
}

% Inserta un email con un link cliqueable
\newcommand{\insertemail}[1]{%
	\href{mailto:#1}{\hreftext{#1}}%
}

% Inserta un teléfono celular
\newcommand{\insertphone}[1]{%
	\href{tel:#1}{\hreftext{#1}}%
}

% Actualiza el padding de las celdas de las tablas
%	#1	Padding horizontal (em)
%	#2	Padding vertical (em)
\newcommand{\settablecellpadding}[2]{%
	\emptyvarerr{\settablecellpadding}{#1}{Padding horizontal no definido}
	\emptyvarerr{\settablecellpadding}{#2}{Padding vertical no definido}
	\setlength{\tabcolsep}{#1 em} % Horizontal
	\def\arraystretch {#2} % Vertical
}

% Definición de colores
\definecolor{backcolour}{rgb}{0.95, 0.95, 0.92}
\definecolor{codegray}{rgb}{0.5, 0.5, 0.5}
\definecolor{codegreen}{rgb}{0, 0.6, 0}
\definecolor{codepurple}{rgb}{0.58, 0, 0.82}
\definecolor{dark-blue}{rgb}{0.15, 0.15, 0.40}
\definecolor{dgray}{RGB}{104, 108, 113}
\definecolor{dkgreen}{rgb}{0, 0.6, 0}
\definecolor{gray}{rgb}{0.5, 0.5, 0.5}
\definecolor{lbrown}{RGB}{255, 252, 249}
\definecolor{lgray}{RGB}{180, 180, 180}
\definecolor{lyellow}{rgb}{1.0, 1.0, 0.88}
\definecolor{mauve}{rgb}{0.58, 0, 0.82}
\definecolor{mygray}{rgb}{0.5, 0.5, 0.5}
\definecolor{mygreen}{rgb}{0, 0.6, 0}
\definecolor{mylilas}{RGB}{170, 55, 241}
\definecolor{ocre}{RGB}{243, 102, 25}

% Crea una sección identada
\newenvironment{indentsection}{%
	\begin{list}{}{%
		\setlength{\leftmargin}{0.75\newparindent}
		\setlength{\parsep}{0pt}
		\setlength{\parskip}{0pt}
		\setlength{\itemsep}{0pt}
		\setlength{\topsep}{0pt}}
	}{
	\end{list}
}

% Crea una sección identada doble
\newenvironment{indentsectiondouble}{%
	\begin{list}{}{%
			\setlength{\leftmargin}{1.5\newparindent}
			\setlength{\parsep}{0pt}
			\setlength{\parskip}{0pt}
			\setlength{\itemsep}{0pt}
			\setlength{\topsep}{0pt}}
	}{
	\end{list}
}

% Crea un bloque de contenido
%	#1	Título
%	#2	Estilo de línea
\newenvironment{cvblock}[2][]{%
	\textcolor{\cvblocklinecolor}{
		\ifx\hfuzz#1\hfuzz
			\breakvspace{1em} \hrule \nobreakvspace{0em}
		\else
			\ifthenelse{\equal{#1}{..}}{
				\noindent \makebox[1\linewidth]{\dotfill}
				\vspace{-2em}
			}{
				\ifthenelse{\equal{#1}{none}}{
			}{
				\throwwarning{Estilo de linea invalido}
				\stop
			}}
		\fi
	}
	\ifx\hfuzz#2\hfuzz
		\throwwarning{Titulo del bloque no definido}
		\stop
	\else
		\roottitle{#2}
	\fi
	\vspace{0.15em}
	}{
	\ifthenelse{\equal{\GLOBALhasinstitution}{false}}{}{%
		\ifthenelse{\equal{\GLOBALinstitutionshortend}{false}}{
			\vspace{-0.1\baselineskip}%
		}{%
			\vspace{-\baselineskip}%
		}%
	}
	\global\def\GLOBALinstitutionshortend {false}
	\global\def\GLOBALhasinstitution {false}
	\par
}

% Párrafo de descripción
%	#1	Estilo de línea
\newenvironment{summary}[1][]{%
	\begin{cvblock}[#1]{\nomsummary}%
		\begingroup%
		\vspace{-1em}%
		\begin{multicols}{2}%
			\noindent%
		}{
		\end{multicols}
		\vspace{-0.75\baselineskip}
		\par
		\endgroup
	\end{cvblock}
}

% Párrafo de descripción con foto de perfil
%	#1	Estilo de línea
\newenvironment{photosummary}[1][]{%
	\begin{cvblock}[#1]{\nomsummary}%
		\vspace{0.1em}%
		\noindent%
		\begin{minipage}[t]{\userphotosize cm}%
			\begin{flushleft}%
				\begingroup%
				\setlength{\fboxsep}{0pt}
				~ \\ \vspace{-0.75em}
				\ifthenelse{\equal{\showuserphotoborder}{true}}{
					\noindent \textcolor{\userphotobordercolor}{\fbox{\includegraphics[width=0.99\linewidth]{\fotoautor}}}
				}{
					\noindent \includegraphics[width=\userphotosize cm,height=\userphotosize cm]{\fotoautor}
				}
				\vspace{-0.7em}
				\endgroup
			\end{flushleft}
		\end{minipage}
		\hspace*{0.37cm}
		\begingroup
		\def\arraystretch{1.3}
		\begin{minipage}[t]{\dimexpr\linewidth-\userphotosize cm-0.57cm}
		}{
		\end{minipage}
		\endgroup
		\par
	\end{cvblock}
}

% Tabla de datos personales
%	#1	Estilo de línea
\newenvironment{personaltabledata}[1][]{%
	\begin{cvblock}[#1]{\nompersonaldata}%
		\vspace{0.1em}
		\noindent
		\ifthenelse{\equal{\personaltabledatapic}{true}}{
			\begin{minipage}[t]{\userphotosize cm}
				\begin{flushleft}
					\begingroup
					\setlength{\fboxsep}{0pt}
					~ \\ \vspace{-0.95em}
					\ifthenelse{\equal{\showuserphotoborder}{true}}{
						\noindent \textcolor{\userphotobordercolor}{\fbox{\includegraphics[width=0.99\linewidth]{\fotoautor}}}
					}{
						\noindent \includegraphics[width=\userphotosize cm,height=\userphotosize cm]{\fotoautor}
					}
					\endgroup
				\end{flushleft}
			\end{minipage}
		}{}
		\hspace*{0cm}
		\begingroup
		\def\arraystretch{1.3}
		\begin{minipage}[t]{0.79\linewidth}
			\vspace*{-0.945cm}
			\begin{table}[H]
				\begin{tabular}{ll}
				}{
				\end{tabular}
			\end{table}
		\end{minipage}
		\endgroup
		\par
	\end{cvblock}
}

% Nueva institución
%	#1	Opcional: url o link institución
%	#2	Nombre institución
%	#3	Ubicación institución
\newenvironment{institution}[3][]{%
	\needspace{2\baselineskip}%
	\ifthenelse{\equal{\institutiontitlebold}{true}}{%
		\def\LOCALinstitutiontitle {\textbf{#2}}%
	}{%
		\def\LOCALinstitutiontitle {#2}%
	}
	\nopagebreak[4]%
	\begin{indentsection}%
		\item []%
		\noindent%
		\ifx\hfuzz#3\hfuzz
			\begin{minipage}{\linewidth}
				\ifx\hfuzz#1\hfuzz
					\textscale{1.1}{\textcolor{\linkcolor}{\LOCALinstitutiontitle}}
				\else
					\textscale{1.1}{\href{#1}{\LOCALinstitutiontitle}}
				\fi
			\end{minipage}
		\else
			\begin{minipage}[t][][t]{.558\linewidth}
				\ifx\hfuzz#1\hfuzz
					\textscale{1.1}{\textcolor{\linkcolor}{\LOCALinstitutiontitle}}
				\else
					\textscale{1.1}{\href{#1}{\LOCALinstitutiontitle}}
				\fi
			\end{minipage}
			\begin{minipage}[t][][t]{.44\linewidth}
				\begin{flushright}
					\noindent \textcolor{\instregioncolor}{\textsc{#3}}
				\end{flushright}
			\end{minipage}
		\fi
		\vspace*{-\baselineskip}
		\break
	\end{indentsection}
	\vspace{-1.5\baselineskip}
	\noindent
	\global\def\GLOBALinstitutionenabled {true}
	}{
	\global\def\GLOBALinstitutionenabled {false}
	\global\def\GLOBALhasinstitution {true}
	\par
	\vspace{0.2\baselineskip}
	\nopagebreak[4]
}

% Crea la firma del usuario
\newenvironment{signature}{%
	\vfill%
	\begin{flushright}%
		\begin{tabular}{c}%
			\includegraphics[width=\usersignaturesize cm]{\firmaautor} \\
		}{
		\end{tabular}
	\end{flushright}
}

% Itemize en negrita
%	#1	Parámetros opcionales
\newenvironment{itemizebf}[1][]{%
	\begin{itemize}[font=\bfseries,#1]%
	}{
	\end{itemize}
}

% Enumerate en negrita
%	#1	Parámetros opcionales
\newenvironment{enumeratebf}[1][]{%
	\begin{enumerate}[font=\bfseries,#1]%
	}{
	\end{enumerate}
}

% -----------------------------------------------------------------------------
% CONFIGURACIÓN INICIAL DEL DOCUMENTO
% -----------------------------------------------------------------------------
% Se revisa si las variables no han sido borradas
\checkvardefined{\anonacimientoautor}
\checkvardefined{\ciautor}
\checkvardefined{\correoautor}
\checkvardefined{\dianacimientoautor}
\checkvardefined{\mesnacimientoautor}
\checkvardefined{\nombreautor}
\checkvardefined{\telefonoautor}

% Se añade \xspace a las variables
\makeatletter
\g@addto@macro\anonacimientoautor\xspace
\g@addto@macro\ciautor\xspace
\g@addto@macro\dianacimientoautor\xspace
\g@addto@macro\mesnacimientoautor\xspace
\g@addto@macro\nombreautor\xspace
\g@addto@macro\telefonoautor\xspace
\makeatother

% Se define metadata del pdf
\ifthenelse{\equal{\cfgshowbookmarkmenu}{true}}{
	\def\cdfpagemodepdf {UseOutlines}
}{
	\def\cdfpagemodepdf {UseNone}
}
\hypersetup{
	keeppdfinfo,
	bookmarksopen={\cfgpdfbookmarkopen},
	bookmarksopenlevel={\cfgbookmarksopenlevel},
	bookmarkstype={toc},
	pdfauthor={\nombreautor},
	pdfcenterwindow={\cfgpdfcenterwindow},
	pdfcopyright={\headertitle, \nombreautor. Email: \correoautor. Tel: \telefonoautor},
	pdfcreator={LaTeX},
	pdfdisplaydoctitle={\cfgpdfdisplaydoctitle},
	pdffitwindow={\cfgpdffitwindow},
	pdfinfo={
		Autor.Email={\correoautor},
		Autor.Nombre={\nombreautor},
		Autor.Telefono={\telefonoautor},
		Template.Autor.Alias={ppizarror},
		Template.Autor.Email={pablo@ppizarror.com},
		Template.Autor.Nombre={Pablo Pizarro R.},
		Template.Autor.Web={https://ppizarror.com/},
		Template.Codificacion={UTF-8},
		Template.Fecha={22/08/2021},
		Template.Licencia.Tipo={MIT},
		Template.Licencia.Web={https://opensource.org/licenses/MIT/},
		Template.Nombre={Professional-CV},
		Template.Tipo={Normal},
		Template.Version.Dev={3.0.0-1},
		Template.Version.Hash={9638B1329538B9762A8ED28E93CEFA3D},
		Template.Version.Release={3.0.0},
		Template.Web.Dev={https://github.com/Template-Latex/Professional-CV/},
		Template.Web.Manual={https://latex.ppizarror.com/Professional-CV/}
	},
	pdfkeywords={CV, \nombreautor, \headertitle, \correoautor},
	pdfmenubar={\cfgpdfmenubar},
	pdfpagelayout={\cfgpdfpagemode},
	pdfpagemode={\cdfpagemodepdf},
	pdfproducer={Professional-CV v3.0.0 | (Pablo Pizarro R.) ppizarror.com},
	pdfremotestartview={Fit},
	pdfstartpage={1},
	pdfstartview={\cfgpdfpageview},
	pdfsubject={CV \nombreautor},
	pdftitle={\headertitle~ \nombreautor},
	pdftoolbar={\cfgpdftoolbar},
	pdftype={Text}
}

% Establece la carpeta de imágenes por defecto
\graphicspath{{./img}}

% Ajuste del entrelineado
\renewcommand{\baselinestretch}{\documentinterline}

% Ajuste de tablas
\settablecellpadding{0.5}{1}

% Configuración de los colores
\def\linkcolor {\urlcolor}
\color{\maintextcolor} % Color principal
\arrayrulecolor{\tablelinecolor} % Color de las líneas de las tablas
\sethlcolor{\highlightcolor} % Color del subrayado por defecto
\ifthenelse{\equal{\showborderonlinks}{true}}{
	\hypersetup{
		% Color de links con borde
		citebordercolor=\numcitecolor,
		linkbordercolor=\linkcolor,
		urlbordercolor=\urlcolor
	}
}{
	\hypersetup{
		% Color de links sin borde
		hidelinks,
		colorlinks=true,
		citecolor=\numcitecolor,
		linkcolor=\linkcolor,
		urlcolor=\urlcolor
	}
}
\ifthenelse{\equal{\colorpage}{white}}{}{
	\pagecolor{\colorpage}
}

% Configuración de la identación
\newlength{\newparindent}
\addtolength{\newparindent}{\parindent}
\newlength{\doubleparindent}
\addtolength{\doubleparindent}{\parindent}
\addtolength{\doubleparindent}{\parindent}

% Configuración de hbox y vbox
\hfuzz=200pt
\vfuzz=200pt
\hbadness=\maxdimen
\vbadness=\maxdimen

% Se activa el word-wrap para textos con \texttt{}
\ttfamily \hyphenchar\the\font=`\-
\expandafter\def\expandafter\UrlBreaks\expandafter{\UrlBreaks%
	\do\a\do\b\do\c\do\d\do\e\do\f\do\g\do\h\do\i\do\j%
	\do\k\do\l\do\m\do\n\do\o\do\p\do\q\do\r\do\s\do\t%
	\do\u\do\v\do\w\do\x\do\y\do\z\do\A\do\B\do\C\do\D%
	\do\E\do\F\do\G\do\H\do\I\do\J\do\K\do\L\do\M\do\N%
	\do\O\do\P\do\Q\do\R\do\S\do\T\do\U\do\V\do\W\do\X%
	\do\Y\do\Z%
}

% Se define el tipo de texto de los url
\urlstyle{\fonturl}

% Se define la versión menor a compilar
\pdfminorversion=\pdfcompileversion

% -----------------------------------------------------------------------------
% CONFIGURACIONES DE PÁGINA
% -----------------------------------------------------------------------------
\newcommand{\templatePagecfg}{
	
	% Se define el punto decimal
	\ifthenelse{\equal{\pointdecimal}{true}}{
		\decimalpoint}{
	}
	
	% Configuración de estilo de página
	\setpagemargincm{\pagemarginleft}{\pagemargintop}{\pagemarginright}{\pagemarginbottom}
	\ifthenelse{\equal{\hfstyle}{style1}}{
		\pagestyle{empty}
	}{
	\ifthenelse{\equal{\hfstyle}{style2}}{
		\pagestyle{fancy} \fancyhf{}
		\fancyhead[L]{}
		\fancyhead[C]{}
		\fancyhead[R]{}
		\fancyfoot[L]{\textcolor{\hftextcolor}{\fontsize{10}{12} \headertitle}}
		\fancyfoot[C]{\textcolor{\hftextcolor}{\small \nombreautor}}
		\fancyfoot[R]{\textcolor{\hftextcolor}{\small \thepage}}
		\renewcommand{\headrulewidth}{0pt}
		\renewcommand{\footrulewidth}{0pt}
	}{
	\ifthenelse{\equal{\hfstyle}{style3}}{
		\pagestyle{fancy} \fancyhf{}
		\fancyfoot[L]{\textcolor{\hftextcolor}{\small \thepage}}
		\renewcommand{\headrulewidth}{0pt}
		\renewcommand{\footrulewidth}{0pt}
	}{
	\ifthenelse{\equal{\hfstyle}{style4}}{
		\pagestyle{fancy} \fancyhf{}
		\fancyfoot[C]{\textcolor{\hftextcolor}{\small \thepage}}
		\renewcommand{\headrulewidth}{0pt}
		\renewcommand{\footrulewidth}{0pt}
	}{
	\ifthenelse{\equal{\hfstyle}{style5}}{
		\pagestyle{fancy} \fancyhf{}
		\fancyfoot[R]{\textcolor{\hftextcolor}{\small \thepage}}
		\renewcommand{\headrulewidth}{0pt}
		\renewcommand{\footrulewidth}{0pt}
	}{
	\ifthenelse{\equal{\hfstyle}{style6}}{
		\pagestyle{fancy} \fancyhf{}
		\fancyfoot[L]{\textcolor{\hftextcolor}{\small \thepage}}
		\renewcommand{\headrulewidth}{0pt}
		\renewcommand{\footrulewidth}{0.5pt}
	}{
	\ifthenelse{\equal{\hfstyle}{style7}}{
		\pagestyle{fancy} \fancyhf{}
		\fancyfoot[C]{\textcolor{\hftextcolor}{\small \thepage}}
		\renewcommand{\headrulewidth}{0pt}
		\renewcommand{\footrulewidth}{0.5pt}
	}{
	\ifthenelse{\equal{\hfstyle}{style8}}{
		\pagestyle{fancy} \fancyhf{}
		\fancyfoot[R]{\textcolor{\hftextcolor}{\small \thepage}}
		\renewcommand{\headrulewidth}{0pt}
		\renewcommand{\footrulewidth}{0.5pt}
	}{
	\ifthenelse{\equal{\hfstyle}{style9}}{
		\pagestyle{fancy} \fancyhf{}
		\fancyfoot[L]{\textcolor{\hftextcolor}{\small \nombreautor}}
		\fancyfoot[R]{\textcolor{\hftextcolor}{\small \thepage}}
		\renewcommand{\headrulewidth}{0pt}
		\renewcommand{\footrulewidth}{0.5pt}
	}{
		\throwbadconfigondoc{Estilo de header-footer incorrecto}{\hfstyle}{style1..style9}}}}}}}}}
	}
	
	% Escribe el encabezado
	\writeheader
	
	% Estilo de título de secciones
	\sectionfont{\color{\titlecolor} \fontsizetitle \styletitle \selectfont}
	
}
