% Template:     Professional-CV
% Documento:    Núcleo del template
% Versión:      3.2.1 (09/11/2021)
% Codificación: UTF-8
%
% Autor: Pablo Pizarro R.
%        pablo@ppizarror.com
%
% Manual template: [https://latex.ppizarror.com/professional-cv]
% Licencia MIT:    [https://opensource.org/licenses/MIT]

% -----------------------------------------------------------------------------
% CONFIGURACIONES
% -----------------------------------------------------------------------------
% Template:     Professional-CV
% Documento:    Configuraciones del template
% Versión:      3.3.9 (30/10/2023)
% Codificación: UTF-8
%
% Autor: Pablo Pizarro R.
%        pablo@ppizarror.com
%
% Manual template: [https://latex.ppizarror.com/professional-cv]
% Licencia MIT:    [https://opensource.org/licenses/MIT]

% CONFIGURACIONES GENERALES
\def\documentfontsize {11}        % Tamaño de la fuente del documento [pt]
\def\documentinterline {1.1}      % Interlineado del documento defecto [factor]
\def\fontdocument {default}       % Tipografía base, ver soportadas en manual
\def\fonttypewriter {cmodern}     % Tipografía de \texttt, ver manual
\def\fonturl {same}               % Tipo de fuente url {tt,sf,rm,same}
\def\pointdecimal {false}         % Decimales con punto en vez de coma

% CONFIGURACIÓN DEL ESTILO DEL TEMPLATE
\def\dateseparator {--}           % Carácter que separa las fechas
\def\headeritemsmargin {0.3}      % Margen vertical de los ítems del header [cm]
\def\headeritemslinespace {0.1}   % Distancia entre líneas del header [cm]
\def\headerseparator {\sbullet}   % Separador entre elementos del encabezado
\def\headertitlecentered {false}  % Header centrado
\def\headertitlemargin {0.25}     % Margen de \headertitle [cm]
\def\hfstyle {style5}             % Estilos de encabezado y pie de pag (11 disp.)
\def\institutioniconleft {false}  % Íconos a la izquierda o derecha
\def\institutioniconsize {normal} % Tamaño ícono {small,normal,large,huge}
\def\institutiontitlebold {false} % Título institución en negrita
\def\personaltabledatapic {true}  % Muestra/Oculta la foto en tabla antecedentes
\def\showuserphotoborder {true}   % Muestra un recuadro en la foto de perfil
\def\userphotosize {3.5}          % Dimensiones de la foto del usuario [cm]
\def\usersignaturesize {5.0}      % Ancho de la foto de la firma [cm]
\def\writebirthdayheader {true}   % Escribe año nacimiento encabezado
\def\writelastchangeheader{false} % Escribe fecha último cambio en el encabezado
\def\writetitleheader {false}     % Escribe el \headertitle en el encabezado

% CONFIGURACIÓN DE LOS COLORES DEL DOCUMENTO
\def\colorpage {white}            % Color de la página
\def\cvblocklinecolor {gray!30}   % Color de las líneas divisor. de los bloques
\def\datecolor {gray!90}          % Color de fechas en entradas de instituciones
\def\headerseparatorcolor {black} % Color del separador del header
\def\headertextcolor {black}      % Color del texto en el header
\def\headerurlcolor {dark-blue}   % Color de los enlaces en el header
\def\hftextcolor {lgray}          % Color del texto en header-footer
\def\highlightcolor {yellow}      % Color del subrayado con \hl
\def\instentrytitlecolor {black}  % Color del título en entrada de instituciones
\def\instnamecolor {dark-blue}    % Color del nombre de las instituciones (+url)
\def\instregioncolor {black}      % Color de la región en las instituciones
\def\lastchangeheadercolor {gray} % Color de fecha último cambio en encabezado
\def\maintextcolor {black}        % Color principal del texto
\def\numcitecolor {black}         % Color del número de las referencias o citas
\def\otherentrytitlecolor {black} % Color título en entradas <otherentry>
\def\personaltblentcolor {black}  % Color de los títulos tabla de antecedentes
\def\showborderonlinks {false}    % Encierra los links por un recuadro de color
\def\tablelinecolor {black}       % Color de las líneas de las tablas
\def\titlecolor {black}           % Color de los títulos
\def\urlcolor {blue}              % Color enlaces web en el cuerpo del documento
\def\userphotobordercolor {black} % Color del borde de la foto del usuario

% ESTILO DE LOS TÍTULOS
\def\fontsizemaintitle {\huge}    % Tamaño título principal
\def\fontsizetitle {\Large}       % Tamaño título secciones
\def\stylemaintitle {\bfseries}   % Estilo título principal
\def\styletitle {\bfseries}       % Estilo título secciones

% MÁRGENES DE PÁGINA
\def\pagemarginbottom {1.9}       % Margen inferior página [cm]
\def\pagemarginleft {1.9}         % Margen izquierdo página [cm]
\def\pagemarginright {1.9}        % Margen derecho página [cm]
\def\pagemargintop {1.5}          % Margen superior página [cm]

% OPCIONES DEL PDF COMPILADO
\def\cfgbookmarksopenlevel {1}    % Nivel de los marcadores (1: títulos)
\def\cfgpdfbookmarkopen {false}   % Expande marcadores hasta el nivel config.
\def\cfgpdfcenterwindow {true}    % Centra la ventana del lector al abrir el pdf
\def\cfgpdfdisplaydoctitle {true} % Muestra nombre autor como título del pdf
\def\cfgpdffitwindow {false}      % Ajusta ventana del lector al tamaño del pdf
\def\cfgpdfmenubar {true}         % Muestra el menú del lector al abrir el pdf
\def\cfgpdfpagemode {OneColumn}   % Modo de página pdf (OneColumn,SinglePage)
\def\cfgpdfpageview {FitH}        % Fit,FitH,FitV,FitR,FitB,FitBH,FitBV
\def\cfgpdftoolbar {true}         % Muestra la barra de herramientas en lector
\def\cfgshowbookmarkmenu {false}  % Muestra menú de marcadores al abrir el pdf
\def\pdfcompileversion {7}        % Versión mínima del pdf a compilar (1.x)

% NOMBRES DE LOS OBJETOS
\def\headertitle {Curriculum Vitae}   % Título principal en el header
\def\nomlastchange {Último cambio el} % Nombre apartado último cambio
\def\nompersonaldata {Antecedentes Personales} % Título antecedentes personales
\def\nomsummary {Descripción}         % Título sección descripción
\def\present {presente}               % Fecha presente en <institutionitem>


% -----------------------------------------------------------------------------
% IMPORTACIÓN DE FUNCIONES Y LIBRERÍAS
% -----------------------------------------------------------------------------
% Parches de librerías
\let\counterwithout\relax
\let\counterwithin\relax
\let\underbar\relax
\let\underline\relax

% Si se desactiva el idioma
\def\unaccentedoperators {}
\def\decimalpoint {}
\def\bibname {}

% Parche de sectsty.sty
\makeatletter
\def\underline#1{\relax\ifmmode\@@underline{#1}\else $\@@underline{\hbox{#1}}\m@th$\relax\fi}
\def\underbar#1{\underline{\sbox\tw@{#1}\dp\tw@\z@\box\tw@}}
\makeatother

% Tamaño de la fuente del documento
\usepackage{scrextend}
\usepackage{anyfontsize}
\changefontsizes{\documentfontsize pt}

% -----------------------------------------------------------------------------
% Librerías del núcleo
% -----------------------------------------------------------------------------
% Manejo de condicionales
\usepackage{iftex}
\usepackage{ifthen}

% Verifica el tipo de compilador
\ifPDFTeX
	\def\compilertype {pdf2latex}
\else\ifXeTeX
	\def\compilertype {xelatex}
\else\ifLuaTeX
	\def\compilertype {lualatex}
\else
	\errmessage{Compilador no soportado}
	\stop
	\fi\fi
\fi

% Carga el idioma
\usepackage{tracklang}
\IfTrackedLanguage{spanish}{
	\usepackage[es-nosectiondot,es-lcroman,es-noquoting]{babel}
}{ % english, otros
	\usepackage{babel}
}

% Codificación
\ifthenelse{\equal{\compilertype}{pdf2latex}}{
	\usepackage[utf8]{inputenc}}{
}

% Evita error "Too many alphabets used in version normal"
\newcommand\hmmax {0}
\newcommand\bmmax {0}

% -----------------------------------------------------------------------------
% Librerías independientes
% -----------------------------------------------------------------------------
\usepackage{amssymb}       % Librerías matemáticas
\usepackage{array}         % Nuevas características a las tablas
\usepackage{booktabs}      % Permite manejar elementos visuales en tablas
\usepackage{color}         % Colores
\usepackage{colortbl}      % Administración de color en tablas
\usepackage{enumitem}      % Permite enumerar ítems
\usepackage{ifxetex}       % Detecta compilador
\usepackage{graphicx}      % Propiedades extra para los gráficos
\usepackage{lipsum}        % Permite crear textos dummy
\usepackage{mdwlist}       % Listas con encabezado
\usepackage{multicol}      % Múltiples columnas
\usepackage{needspace}     % Maneja los espacios en página
\usepackage{ragged2e}      % Mejora posicionado
\usepackage{relsize}       % Escalado avanzado
\usepackage{sectsty}       % Cambia el estilo de los títulos
\usepackage{selinput}      % Compatibilidad con acentos
\usepackage{setspace}      % Cambia el espacio entre líneas
\usepackage{soul}          % Permite subrayar texto
\usepackage{textcomp}      % Simbología común
\usepackage{wasysym}       % Contiene caracteres misceláneos
\usepackage{wrapfig}       % Permite comprimir imágenes
\usepackage{xcolor}        % Administración de color avanzado
\usepackage{xspace}        % Administra espacios en párrafos y líneas
\usepackage{xurl}          % Permite añadir enlaces

% Dimensiones y geometría del documento
\ifthenelse{\equal{\compilertype}{lualatex}}{ % En lualatex sólo se puede cambiar 1 vez el margen
	\usepackage[top=\pagemargintop cm,bottom=\pagemarginbottom cm,left=\pagemarginleft cm,right=\pagemarginright cm]{geometry}
}{ % pdf2latex, xelatex
	\usepackage{geometry}
}

% -----------------------------------------------------------------------------
% Librerías con parámetros
% -----------------------------------------------------------------------------
\usepackage[pdfencoding=auto,psdextra]{hyperref} % Enlaces, referencias
\usepackage[final]{pdfpages} % Permite administrar páginas en pdf

% -----------------------------------------------------------------------------
% Librerías dependientes
% -----------------------------------------------------------------------------
\usepackage{bookmark}      % Administración de marcadores en pdf
\usepackage{fancyhdr}      % Encabezados y pie de páginas
\usepackage{float}         % Administrador de posiciones de objetos
\usepackage{hyperxmp}      % Etiquetas opcionales para el pdf compilado
\usepackage{multirow}      % Agrega nuevas opciones a las tablas

% -----------------------------------------------------------------------------
% Tipografía del documento
% -----------------------------------------------------------------------------
% Tipografías clásicas
\ifthenelse{\equal{\fontdocument}{default}}{
	\ifxetex
		\usepackage{fontspec}
		\setmainfont
		[ExternalLocation,
		Mapping=tex-text,
		Numbers=OldStyle,
		Ligatures={Common,Contextual},
		BoldFont=texgyrepagella-bold.otf,
		ItalicFont=texgyrepagella-italic.otf,
		BoldItalicFont=texgyrepagella-bolditalic.otf]
		{texgyrepagella-regular.otf}
		\usepackage[protrusion]{microtype}
	\else
		\usepackage{tgpagella}
		\usepackage[expansion,protrusion]{microtype}
	\fi
}{
\ifthenelse{\equal{\fontdocument}{lmodern}}{
	\usepackage{lmodern}
}{
\ifthenelse{\equal{\fontdocument}{arial}}{
	\usepackage{helvet}
	\renewcommand{\familydefault}{\sfdefault}
}{
\ifthenelse{\equal{\fontdocument}{arial2}}{
	\usepackage{arial}
}{
\ifthenelse{\equal{\fontdocument}{times}}{
	\usepackage{mathptmx}
}{
\ifthenelse{\equal{\fontdocument}{mathptmx}}{
	\usepackage{mathptmx}
}{
\ifthenelse{\equal{\fontdocument}{helvet}}{
	\renewcommand{\familydefault}{\sfdefault}
	\usepackage[scaled=0.95]{helvet}
	\usepackage[helvet]{sfmath}
}{
\ifthenelse{\equal{\fontdocument}{opensans}}{
	\usepackage[default,scale=0.95]{opensans}
}{
\ifthenelse{\equal{\fontdocument}{mathpazo}}{
	\usepackage{mathpazo}
}{
\ifthenelse{\equal{\fontdocument}{cambria}}{
	\usepackage{caladea}
}{
\ifthenelse{\equal{\fontdocument}{libertine}}{
	\usepackage[libertine]{newtxmath}
	\usepackage[tt=false]{libertine}
}{
\ifthenelse{\equal{\fontdocument}{custom}}{
}{

% Otros (último: fbb el 08/08/2021 - https://tug.org/FontCatalogue/seriffonts.html)
\ifthenelse{\equal{\fontdocument}{accanthis}}{
	\usepackage{accanthis}
}{
\ifthenelse{\equal{\fontdocument}{alegreya}}{
	\usepackage{Alegreya}
	\renewcommand*\oldstylenums[1]{{\AlegreyaOsF #1}}
}{
\ifthenelse{\equal{\fontdocument}{alegreyasans}}{
	\usepackage[sfdefault]{AlegreyaSans}
	\renewcommand*\oldstylenums[1]{{\AlegreyaSansOsF #1}}
}{
\ifthenelse{\equal{\fontdocument}{algolrevived}}{
	\usepackage{algolrevived}
}{
\ifthenelse{\equal{\fontdocument}{almendra}}{
	\usepackage{almendra}
}{
\ifthenelse{\equal{\fontdocument}{antpolt}}{
	\usepackage{antpolt}
}{
\ifthenelse{\equal{\fontdocument}{antpoltlight}}{
	\usepackage[light]{antpolt}
}{
\ifthenelse{\equal{\fontdocument}{anttor}}{
	\usepackage[math]{anttor}
}{
\ifthenelse{\equal{\fontdocument}{anttorcondensed}}{
	\usepackage[condensed,math]{anttor}
}{
\ifthenelse{\equal{\fontdocument}{anttorlight}}{
	\usepackage[light,math]{anttor}
}{
\ifthenelse{\equal{\fontdocument}{anttorlightcondensed}}{
	\usepackage[light,condensed,math]{anttor}
}{
\ifthenelse{\equal{\fontdocument}{arev}}{
	\let\quarternote\relax
	\let\eighthnote\relax
	\usepackage{arev}
}{
\ifthenelse{\equal{\fontdocument}{arimo}}{
	\usepackage[sfdefault]{arimo}
	\renewcommand*\familydefault{\sfdefault}
}{
\ifthenelse{\equal{\fontdocument}{arvo}}{
	\usepackage{Arvo}
}{
\ifthenelse{\equal{\fontdocument}{baskervald}}{
	\usepackage{baskervald}
}{
\ifthenelse{\equal{\fontdocument}{baskervaldx}}{
	\usepackage[lf]{Baskervaldx}
	\usepackage[bigdelims,vvarbb]{newtxmath}
	\usepackage[cal=boondoxo]{mathalfa}
	\renewcommand*\oldstylenums[1]{\textosf{#1}}
}{
\ifthenelse{\equal{\fontdocument}{berasans}}{
	\usepackage[scaled]{berasans}
	\renewcommand*\familydefault{\sfdefault}
}{
\ifthenelse{\equal{\fontdocument}{beraserif}}{
	\usepackage{bera}
}{
\ifthenelse{\equal{\fontdocument}{biolinum}}{
	\usepackage{libertine}
	\renewcommand*\familydefault{\sfdefault}
}{
\ifthenelse{\equal{\fontdocument}{bitter}}{
	\usepackage{bitter}
}{
\ifthenelse{\equal{\fontdocument}{boisik}}{
	\let\div\relax
	\usepackage{boisik}
}{
\ifthenelse{\equal{\fontdocument}{bookman}}{
	\usepackage{bookman}
}{
\ifthenelse{\equal{\fontdocument}{cabin}}{
	\usepackage[sfdefault]{cabin}
	\renewcommand*\familydefault{\sfdefault}
}{
\ifthenelse{\equal{\fontdocument}{cabincondensed}}{
	\usepackage[sfdefault,condensed]{cabin}
	\renewcommand*\familydefault{\sfdefault}
}{
\ifthenelse{\equal{\fontdocument}{caladea}}{
	\usepackage{caladea}
}{
\ifthenelse{\equal{\fontdocument}{cantarell}}{
	\usepackage[default]{cantarell}
}{
\ifthenelse{\equal{\fontdocument}{carlito}}{
	\usepackage[sfdefault]{carlito}
	\renewcommand*\familydefault{\sfdefault}
}{
\ifthenelse{\equal{\fontdocument}{charterbt}}{
	\usepackage[bitstream-charter]{mathdesign}
}{
\ifthenelse{\equal{\fontdocument}{chivolight}}{
	\usepackage[familydefault,light]{Chivo}
}{
\ifthenelse{\equal{\fontdocument}{chivoregular}}{
	\usepackage[familydefault,regular]{Chivo}
}{
\ifthenelse{\equal{\fontdocument}{clara}}{
	\usepackage{clara}
}{
\ifthenelse{\equal{\fontdocument}{clearsans}}{
	\usepackage[sfdefault]{ClearSans}
	\renewcommand*\familydefault{\sfdefault}
}{
\ifthenelse{\equal{\fontdocument}{cochineal}}{
	\usepackage{cochineal}
}{
\ifthenelse{\equal{\fontdocument}{coelacanth}}{
	\usepackage[nf]{coelacanth}
	\let\oldnormalfont\normalfont
	\def\normalfont{\oldnormalfont\mdseries}
}{
\ifthenelse{\equal{\fontdocument}{coelacanthextralight}}{
	\usepackage[el,nf]{coelacanth}
	\let\oldnormalfont\normalfont
	\def\normalfont{\oldnormalfont\mdseries}
}{
\ifthenelse{\equal{\fontdocument}{coelacanthlight}}{
	\usepackage[l,nf]{coelacanth}
	\let\oldnormalfont\normalfont
	\def\normalfont{\oldnormalfont\mdseries}
}{
\ifthenelse{\equal{\fontdocument}{comfortaa}}{
	\usepackage[default]{comfortaa}
}{
\ifthenelse{\equal{\fontdocument}{comicneue}}{
	\usepackage[default]{comicneue}
}{
\ifthenelse{\equal{\fontdocument}{comicneueangular}}{
	\usepackage[default,angular]{comicneue}
}{
\ifthenelse{\equal{\fontdocument}{computerconcrete}}{
	\usepackage{concmath}
}{
\ifthenelse{\equal{\fontdocument}{computerconcreteeuler}}{
	\let\Re\relax
	\let\Im\relax
	\usepackage{beton}
	\usepackage{euler}
}{
\ifthenelse{\equal{\fontdocument}{computermodern}}{
}{
\ifthenelse{\equal{\fontdocument}{computermodernbright}}{
	\usepackage{cmbright}
}{
\ifthenelse{\equal{\fontdocument}{crimson}}{
	\usepackage{crimson}
}{
\ifthenelse{\equal{\fontdocument}{crimsonpro}}{
	\usepackage{CrimsonPro}
	\let\oldnormalfont\normalfont
	\def\normalfont{\oldnormalfont\mdseries}
}{
\ifthenelse{\equal{\fontdocument}{crimsonproextralight}}{
	\usepackage[extralight]{CrimsonPro}
	\let\oldnormalfont\normalfont
	\def\normalfont{\oldnormalfont\mdseries}
}{
\ifthenelse{\equal{\fontdocument}{crimsonprolight}}{
	\usepackage[light]{CrimsonPro}
	\let\oldnormalfont\normalfont
	\def\normalfont{\oldnormalfont\mdseries}
}{
\ifthenelse{\equal{\fontdocument}{crimsonpromedium}}{
	\usepackage[medium]{CrimsonPro}
	\let\oldnormalfont\normalfont
	\def\normalfont{\oldnormalfont\mdseries}
}{
\ifthenelse{\equal{\fontdocument}{cyklop}}{
	\usepackage{cyklop}
}{
\ifthenelse{\equal{\fontdocument}{dejavusans}}{
	\usepackage{DejaVuSans}
	\renewcommand*\familydefault{\sfdefault}
}{
\ifthenelse{\equal{\fontdocument}{dejavusanscondensed}}{
	\usepackage{DejaVuSansCondensed}
	\renewcommand*\familydefault{\sfdefault}
}{
\ifthenelse{\equal{\fontdocument}{domitian}}{
	\usepackage{mathpazo}
	\usepackage{domitian}
	\let\oldstylenums\oldstyle
}{
\ifthenelse{\equal{\fontdocument}{droidsans}}{
	\usepackage[defaultsans]{droidsans}
	\renewcommand*\familydefault{\sfdefault}
}{
\ifthenelse{\equal{\fontdocument}{electrum}}{
	\usepackage[lf]{electrum}
}{	
\ifthenelse{\equal{\fontdocument}{erewhon}}{
	\usepackage[proportional,scaled=1.064]{erewhon}
	\usepackage[erewhon,vvarbb,bigdelims]{newtxmath}
	\renewcommand*\oldstylenums[1]{\textosf{#1}}
}{
\ifthenelse{\equal{\fontdocument}{fbb}}{
	\usepackage{fbb}
}{
\ifthenelse{\equal{\fontdocument}{fetamont}}{
	\usepackage{fetamont}
	\renewcommand*\familydefault{\sfdefault}
}{
\ifthenelse{\equal{\fontdocument}{firasans}}{
	\usepackage[sfdefault]{FiraSans}
	\renewcommand*\familydefault{\sfdefault}
}{
\ifthenelse{\equal{\fontdocument}{firasansnewtxsf}}{
	\usepackage[sfdefault]{FiraSans}
	\usepackage{newtxsf}
}{
\ifthenelse{\equal{\fontdocument}{fourier}}{
	\usepackage{fourier}
}{
\ifthenelse{\equal{\fontdocument}{fouriernc}}{
	\usepackage{fouriernc}
}{
\ifthenelse{\equal{\fontdocument}{gfsartemisia}}{
	\let\textlozenge\relax
	\usepackage{gfsartemisia}
}{
\ifthenelse{\equal{\fontdocument}{gfsartemisiaeuler}}{
	\let\textlozenge\relax
	\let\Re\relax
	\let\Im\relax
	\usepackage{gfsartemisia-euler}
}{
\ifthenelse{\equal{\fontdocument}{heuristica}}{
	\usepackage{heuristica}
	\usepackage[heuristica,vvarbb,bigdelims]{newtxmath}
	\renewcommand*\oldstylenums[1]{\textosf{#1}}
}{
\ifthenelse{\equal{\fontdocument}{iwona}}{
	\usepackage[math]{iwona}
}{
\ifthenelse{\equal{\fontdocument}{iwonacondensed}}{
	\usepackage[condensed,math]{iwona}
}{
\ifthenelse{\equal{\fontdocument}{iwonalight}}{
	\usepackage[light,math]{iwona}
}{
\ifthenelse{\equal{\fontdocument}{iwonalightcondensed}}{
	\usepackage[light,condensed,math]{iwona}
}{
\ifthenelse{\equal{\fontdocument}{kerkis}}{
	\usepackage{kmath,kerkis}
}{
\ifthenelse{\equal{\fontdocument}{kurier}}{
	\usepackage[math]{kurier}
}{
\ifthenelse{\equal{\fontdocument}{kuriercondensed}}{
	\usepackage[condensed,math]{kurier}
}{
\ifthenelse{\equal{\fontdocument}{kurierlight}}{
	\usepackage[light,math]{kurier}
}{
\ifthenelse{\equal{\fontdocument}{kurierlightcondensed}}{
	\usepackage[light,condensed,math]{kurier}
}{
\ifthenelse{\equal{\fontdocument}{lato}}{
	\usepackage[default]{lato}
}{
\ifthenelse{\equal{\fontdocument}{libertinus}}{
	\usepackage{libertinus}
}{
\ifthenelse{\equal{\fontdocument}{librebaskerville}}{
	\usepackage{librebaskerville}
}{
\ifthenelse{\equal{\fontdocument}{librebodoni}}{
	\usepackage{LibreBodoni}
}{
\ifthenelse{\equal{\fontdocument}{librecaslon}}{
	\usepackage{librecaslon}
}{
\ifthenelse{\equal{\fontdocument}{libris}}{
	\usepackage{libris}
	\renewcommand*\familydefault{\sfdefault}
}{
\ifthenelse{\equal{\fontdocument}{lxfonts}}{
	\usepackage{lxfonts}
}{
\ifthenelse{\equal{\fontdocument}{merriweather}}{
	\usepackage[sfdefault]{merriweather}
}{
\ifthenelse{\equal{\fontdocument}{merriweatherlight}}{
	\usepackage[sfdefault,light]{merriweather}
}{
\ifthenelse{\equal{\fontdocument}{mintspirit}}{
	\usepackage[default]{mintspirit}
}{
\ifthenelse{\equal{\fontdocument}{mlmodern}}{
	\usepackage{mlmodern}
}{
\ifthenelse{\equal{\fontdocument}{montserratalternatesextralight}}{
	\usepackage[defaultfam,extralight,tabular,lining,alternates]{montserrat}
	\renewcommand*\oldstylenums[1]{{\fontfamily{Montserrat-TOsF}\selectfont #1}}
}{
\ifthenelse{\equal{\fontdocument}{montserratalternatesregular}}{
	\usepackage[defaultfam,tabular,lining,alternates]{montserrat}
	\renewcommand*\oldstylenums[1]{{\fontfamily{Montserrat-TOsF}\selectfont #1}}
}{
\ifthenelse{\equal{\fontdocument}{montserratalternatesthin}}{
	\usepackage[defaultfam,thin,tabular,lining,alternates]{montserrat}
	\renewcommand*\oldstylenums[1]{{\fontfamily{Montserrat-TOsF}\selectfont #1}}
}{
\ifthenelse{\equal{\fontdocument}{montserratextralight}}{
	\usepackage[defaultfam,extralight,tabular,lining]{montserrat}
	\renewcommand*\oldstylenums[1]{{\fontfamily{Montserrat-TOsF}\selectfont #1}}
}{
\ifthenelse{\equal{\fontdocument}{montserratlight}}{
	\usepackage[defaultfam,light,tabular,lining]{montserrat}
	\renewcommand*\oldstylenums[1]{{\fontfamily{Montserrat-TOsF}\selectfont #1}}
}{
\ifthenelse{\equal{\fontdocument}{montserratregular}}{
	\usepackage[defaultfam,tabular,lining]{montserrat}
	\renewcommand*\oldstylenums[1]{{\fontfamily{Montserrat-TOsF}\selectfont #1}}
}{
\ifthenelse{\equal{\fontdocument}{montserratthin}}{
	\usepackage[defaultfam,thin,tabular,lining]{montserrat}
	\renewcommand*\oldstylenums[1]{{\fontfamily{Montserrat-TOsF}\selectfont #1}}
}{
\ifthenelse{\equal{\fontdocument}{newpx}}{
	\usepackage{newpxtext,newpxmath}
}{
\ifthenelse{\equal{\fontdocument}{nimbussans}}{
	\usepackage{nimbussans}
	\renewcommand*\familydefault{\sfdefault}
}{
\ifthenelse{\equal{\fontdocument}{noto}}{
	\usepackage[sfdefault]{noto}
	\renewcommand*\familydefault{\sfdefault}
}{
\ifthenelse{\equal{\fontdocument}{notoserif}}{
	\usepackage{notomath}
}{
\ifthenelse{\equal{\fontdocument}{opensansserif}}{
	\usepackage[default,oldstyle,scale=0.95]{opensans}
}{
\ifthenelse{\equal{\fontdocument}{overlock}}{
	\usepackage[sfdefault]{overlock}
	\renewcommand*\familydefault{\sfdefault}
}{
\ifthenelse{\equal{\fontdocument}{paratype}}{
	\usepackage{paratype}
	\renewcommand*\familydefault{\sfdefault}
}{
\ifthenelse{\equal{\fontdocument}{paratypesanscaption}}{
	\usepackage{PTSansCaption}
	\renewcommand*\familydefault{\sfdefault}
}{
\ifthenelse{\equal{\fontdocument}{paratypesansnarrow}}{
	\usepackage{PTSansNarrow}
	\renewcommand*\familydefault{\sfdefault}
}{
\ifthenelse{\equal{\fontdocument}{pxfonts}}{
	\usepackage{pxfonts}
}{
\ifthenelse{\equal{\fontdocument}{quattrocento}}{
	\usepackage[sfdefault]{quattrocento}
}{
\ifthenelse{\equal{\fontdocument}{raleway}}{
	\usepackage[default]{raleway}
}{
\ifthenelse{\equal{\fontdocument}{roboto}}{
	\usepackage[sfdefault]{roboto}
}{
\ifthenelse{\equal{\fontdocument}{robotocondensed}}{
	\usepackage[sfdefault,condensed]{roboto}
}{
\ifthenelse{\equal{\fontdocument}{robotolight}}{
	\usepackage[sfdefault,light]{roboto}
}{
\ifthenelse{\equal{\fontdocument}{robotolightcondensed}}{
	\usepackage[sfdefault,light,condensed]{roboto}
}{
\ifthenelse{\equal{\fontdocument}{robotothin}}{
	\usepackage[sfdefault,thin]{roboto}
}{
\ifthenelse{\equal{\fontdocument}{rosario}}{
	\usepackage[familydefault]{Rosario}
}{
\ifthenelse{\equal{\fontdocument}{sourcesanspro}}{
	\usepackage[default]{sourcesanspro}
}{
\ifthenelse{\equal{\fontdocument}{step}}{
	\usepackage[notext]{stix}
	\usepackage{step}
}{
\ifthenelse{\equal{\fontdocument}{stickstoo}}{
	\usepackage{stickstootext}
	\usepackage[stickstoo,vvarbb]{newtxmath}
}{
\ifthenelse{\equal{\fontdocument}{texgyrebonum}}{
	\usepackage{tgbonum}
}{
\ifthenelse{\equal{\fontdocument}{txfonts}}{
	\usepackage{txfonts}
}{
\ifthenelse{\equal{\fontdocument}{uarial}}{
	\usepackage{uarial}
	\renewcommand*\familydefault{\sfdefault}
}{
\ifthenelse{\equal{\fontdocument}{ugq}}{
	\renewcommand*\sfdefault{ugq}
	\renewcommand*\familydefault{\sfdefault}
}{
\ifthenelse{\equal{\fontdocument}{universalis}}{
	\usepackage[sfdefault]{universalis}
}{
\ifthenelse{\equal{\fontdocument}{universaliscondensed}}{
	\usepackage[condensed,sfdefault]{universalis}
}{
\ifthenelse{\equal{\fontdocument}{venturis}}{
	\usepackage[lf]{venturis}
	\renewcommand*\familydefault{\sfdefault}
}{
	\throwbadconfig[nostop]{Fuente desconocida}{\fontdocument}{(Fuentes recomendadas) default,lmodern,carial,arial2,times,mathptmx,helvet,opensans,mathpazo,cambria,libertine,custom}
	\throwbadconfig[noheader-nostop]{Fuente desconocida}{\fontdocument}{(Fuentes adicionales) accanthis,alegreya,alegreyasans,algolrevived,almendra,antpolt,antpoltlight,anttor,anttorcondensed,anttorlight,anttorlightcondensed,arev,arimo,arvo,baskervald,baskervaldx,berasans,beraserif,biolinum,bitter,boisik,bookman,cabin,cabincondensed,cantarell,caladea,carlito,charterbt,chivolight,chivoregular,clara,clearsans,cochineal,coelacanth,coelacanthextralight,coelacanthlight,comfortaa,comicneue,comicneueangular,computerconcrete,computerconcreteeuler,computermodern,computermodernbright,crimson,crimsonpro,crimsonproextralight,crimsonprolight,crimsonpromedium,cyklop}
	\throwbadconfig[noheader-nostop]{Fuente desconocida}{\fontdocument}{dejavusans,dejavusanscondensed,domitian,droidsans,electrum,erewhon,fbb,fetamont,firasans,firasansnewtxsf,fourier,fouriernc,gfsartemisia,gfsartemisiaeuler,heuristica,iwona,iwonacondensed,iwonalight,iwonalightcondensed,kerkis,kurier,kuriercondensed,kurierlight,kurierlightcondensed,lato,libertinus,librebaskerville,librebodoni,librecaslon,libris,lxfonts}
	\throwbadconfig[noheader]{Fuente desconocida}{\fontdocument}{merriweather,merriweatherlight,mintspirit,mlmodern,montserratalternatesextralight,montserratalternatesregular,montserratalternatesthin,montserratextralight,montserratlight,montserratregular,montserratthin,newpx,nimbussans,noto,notoserif,opensansserif,overlock,paratype,paratypesanscaption,paratypesansnarrow,pxfonts,quattrocento,raleway,roboto,robotolight,robotolightcondensed,robotothin,rosario,sourcesanspro,step,stickstoo,uarial,texgyrebonum,txfonts,ugq,universalis,universaliscondensed,venturis}
	}}}}}}}}}}}}}}}}}}}}}}}}}}}}}}}}}}}}}}}}}}}}}}}}}}}}}}}}}}}}}}}}}}}}}}}}}}}}}}}}}}}}}}}}}}}}}}}}}}}}}}}}}}}}}}}}}}}}}}}}}}}}}}}}}}
}

% -----------------------------------------------------------------------------
% Tipografía typewriter
% -----------------------------------------------------------------------------
% https://tug.org/FontCatalogue/typewriterfonts.html
\ifthenelse{\equal{\fonttypewriter}{custom}}{
}{
\ifthenelse{\equal{\fonttypewriter}{tmodern}}{
	\renewcommand*\ttdefault{lmvtt}
}{
\ifthenelse{\equal{\fonttypewriter}{anonymouspro}}{
	\usepackage[ttdefault=true]{AnonymousPro}
}{
\ifthenelse{\equal{\fonttypewriter}{ascii}}{
	\usepackage{ascii}
	\let\SI\relax
}{
\ifthenelse{\equal{\fonttypewriter}{beramono}}{
	\usepackage[scaled]{beramono}
}{
\ifthenelse{\equal{\fonttypewriter}{cascadiacode}}{
	\usepackage{cascadia-code}
}{
\ifthenelse{\equal{\fonttypewriter}{cmpica}}{
	\usepackage{addfont}
	\addfont{OT1}{cmpica}{\pica}
	\addfont{OT1}{cmpicab}{\picab}
	\addfont{OT1}{cmpicati}{\picati}
	\renewcommand*\ttdefault{pica}
}{
\ifthenelse{\equal{\fonttypewriter}{cmodern}}{
}{
\ifthenelse{\equal{\fonttypewriter}{courier}}{
	\usepackage{courier}
}{
\ifthenelse{\equal{\fonttypewriter}{courier10}}{
	\usepackage{courierten}
}{
\ifthenelse{\equal{\fonttypewriter}{cmvtt}}{
	\renewcommand*\ttdefault{cmvtt}
}{
\ifthenelse{\equal{\fonttypewriter}{dejavusansmono}}{
	\usepackage[scaled]{DejaVuSansMono}
}{
\ifthenelse{\equal{\fonttypewriter}{droidsansmono}}{
	\usepackage[defaultmono]{droidsansmono}
}{
\ifthenelse{\equal{\fonttypewriter}{firamono}}{
	\usepackage[scale=0.85]{FiraMono}
}{
\ifthenelse{\equal{\fonttypewriter}{gomono}}{
	\usepackage[scale=0.85]{GoMono}
}{
\ifthenelse{\equal{\fonttypewriter}{inconsolata}}{
	\usepackage{inconsolata}
}{
\ifthenelse{\equal{\fonttypewriter}{nimbusmono}}{
	\usepackage{nimbusmono}
}{
\ifthenelse{\equal{\fonttypewriter}{newtxtt}}{
	\usepackage[zerostyle=d]{newtxtt}
}{
\ifthenelse{\equal{\fonttypewriter}{nimbusmono}}{
	\usepackage{nimbusmono}
}{
\ifthenelse{\equal{\fonttypewriter}{nimbusmononarrow}}{
	\usepackage{nimbusmononarrow}
}{
\ifthenelse{\equal{\fonttypewriter}{lcmtt}}{
	\renewcommand*\ttdefault{lcmtt}
}{
\ifthenelse{\equal{\fonttypewriter}{sourcecodepro}}{
	\usepackage[ttdefault=true,scale=0.85]{sourcecodepro}
}{
\ifthenelse{\equal{\fonttypewriter}{texgyrecursor}}{
	\usepackage{tgcursor}
}{
\ifthenelse{\equal{\fonttypewriter}{txtt}}{
	\renewcommand*\ttdefault{txtt}
}{
	\throwbadconfig{Fuente desconocida}{\fonttypewriter}{custom,anonymouspro,ascii,beramono,cascadiacode,cmpica,cmodern,courier,courier10,cvmtt,dejavusansmono,droidsansmono,firamono,gomono,inconsolata,kpmonospaced,lcmtt,newtxtt,nimbusmono,nimbusmononarrow,texgyrecursor,tmodern,txtt}
	}}}}}}}}}}}}}}}}}}}}}}}
}

% -----------------------------------------------------------------------------
% Finales
% -----------------------------------------------------------------------------
\usepackage[T1]{fontenc} % Caracteres acentuados
\usepackage{csquotes} % Citas y comillas
\ifthenelse{\equal{\compilertype}{pdf2latex}}{
	\inputencoding{utf8}}{
}

% Definición de variables globales
\global\def\GLOBALhasinstitution {false}
\global\def\GLOBALheaderfaimport {false}
\global\def\GLOBALheaderlineitem {false}
\global\def\GLOBALheaderseparatorsticky {false}

\def\LOCALpercentchar#1{}
\edef\LOCALpercentchar{\expandafter\LOCALpercentchar\string\%}

% Archivo que guarda el código del header
\newwrite\fileheaderitems
\immediate\openout\fileheaderitems=\jobname.hitems
\AtBeginDocument{\immediate\closeout\fileheaderitems}                               

% Lanza un mensaje de error
% 	#1	Función del error
%	#2	Mensaje
\newcommand{\throwerror}[2]{%
	\errmessage{LaTeX Error: \noexpand#1 #2 (linea \the\inputlineno)}%
	\stop%
}

% Lanza un mensaje de advertencia
%	#1	Mensaje
\newcommand{\throwwarning}[1]{%
	\errmessage{LaTeX Warning: #1 (linea \the\inputlineno)}%
}

% Lanza un mensaje de error indicando mala configuración
%	#1	Mensaje de error
% 	#2	Configuración usada
%	#3	Valores esperados
\newcommand{\throwbadconfig}[3]{%
	\errmessage{LaTeX Warning: #1 \noexpand #2=#2. Valores esperados: #3}%
	\stop%
}

% Lanza un mensaje de error indicando mala configuración dentro de begin{document}
%	#1	Mensaje de error
% 	#2	Configuración usada
%	#3	Valores esperados
\newcommand{\throwbadconfigondoc}[3]{%
	\errmessage{#1 \noexpand #2=#2. Valores esperados: #3}%
	\stop%
}

% Comprueba si una variable está definida
%	#1	Variable
\newcommand{\checkvardefined}[1]{%
	\ifthenelse{\isundefined{#1}}{%
		\errmessage{LaTeX Warning: Variable \noexpand#1 no definida}%
		\stop%
	}{}%
}

% Comprueba si una variable está definida
%	#1	Variable
%	#2	Mensaje
\newcommand{\checkextravarexist}[2]{%
	\ifthenelse{\isundefined{#1}}{%
		\errmessage{LaTeX Warning: Variable \noexpand#1 no definida}%
		\ifx\hfuzz#2\hfuzz%
			\errmessage{LaTeX Warning: Defina la variable en el bloque de INFORMACION DEL USUARIO al comienzo del archivo principal del Template}%
		\else%
			\errmessage{LaTeX Warning: #2}%
		\fi%
	}{}%
}

% Lanza un mensaje de error si una variable no ha sido definida
% 	#1	Función del error
%	#2	Variable
%	#3	Mensaje
\newcommand{\emptyvarerr}[3]{
	\ifx\hfuzz#2\hfuzz%
		\errmessage{LaTeX Warning: \noexpand#1 #3 (linea \the\inputlineno)}%
	\fi%
}

% Cambia márgenes de las páginas [cm]
% 	#1	Margen izquierdo
%	#2	Margen superior
%	#3	Margen derecho
%	#4	Margen inferior
\newcommand{\setpagemargincm}[4]{
	\ifthenelse{\equal{\compilertype}{lualatex}}{
		% Geometry no válido en lualatex
	}{
		\newgeometry{left=#1cm, top=#2cm, right=#3cm, bottom=#4cm}
	}
}

% Cambia los márgenes del documento
%	#1	Margen izquierdo
%	#2	Margen derecho
\newcommand{\changemargin}[2]{%
	\emptyvarerr{\changemargin}{#1}{Margen izquierdo no definido}%
	\emptyvarerr{\changemargin}{#2}{Margen derecho no definido}%
	\list{}{\rightmargin#2\leftmargin#1}\item[]%
}
\let\endchangemargin=\endlist

% Agrega el separador entre items del header
\newcommand{\addheaderseparator}{%
	\ifthenelse{\equal{\GLOBALheaderlineitem}{true}}{%
		\ifthenelse{\equal{\GLOBALheaderseparatorsticky}{false}}{%
			\immediate\write\fileheaderitems{\unexpanded{\textcolor{\headerseparatorcolor}{\headerseparator}}\LOCALpercentchar}%
		}{}%
	}{}%
	\global\def\GLOBALheaderlineitem {true}%
}

% Agrega un item de texto al header
%	#1	Texto
\newcommand{\addheadertext}[1]{%
	\global\def\GLOBALheaderseparatorsticky {false}%
	\addheaderseparator%
	\immediate\write\fileheaderitems{\unexpanded{\textcolor{\headertextcolor}{#1}}\LOCALpercentchar}%
}

% Agrega un correo al header
%	#1	Correo
\newcommand{\addheadermail}[1]{%
	\global\def\GLOBALheaderseparatorsticky {false}%
	\addheaderseparator%
	\immediate\write\fileheaderitems{\unexpanded{\href{mailto:#1}{#1}}\LOCALpercentchar}%
}

% Agrega un enlace al header
%	#1	Ícono de fontawesome, si está vacío se usa url normal
%	#2	Enlace
\newcommand{\addheaderlink}[2][]{%
	\ifthenelse{\equal{#1}{}}{ % Sin ícono
		\global\def\GLOBALheaderseparatorsticky {false}%
		\addheaderseparator%
		\immediate\write\fileheaderitems{\unexpanded{\url{#2}}\LOCALpercentchar}%
	}{%
		\addheaderseparator%
		\ifthenelse{\equal{\GLOBALheaderfaimport}{false}}{%
			\global\def\GLOBALheaderfaimport {true}%
			\usepackage{fontawesome5}%
		}{}%
		\ifthenelse{\equal{\GLOBALheaderseparatorsticky}{true}}{%
			\immediate\write\fileheaderitems{\unexpanded{\hspace{0.25cm}}\LOCALpercentchar}%
		}{}%
		\immediate\write\fileheaderitems{\unexpanded{\href{#2}{\faIcon{#1}}}\LOCALpercentchar}%
		\global\def\GLOBALheaderseparatorsticky {true}%
	}%
}

% Agrega una nueva línea al header
\newcommand{\addheadernewline}{%
	\global\def\GLOBALheaderlineitem {false}%
	\global\def\GLOBALheaderseparatorsticky {false}%
	\immediate\write\fileheaderitems{\unexpanded{\vspace{\headeritemslinespace cm}\\}\LOCALpercentchar}%
}

% Agrega un ícono
%	#1	Nombre del ícono
\newcommand{\addheadericon}[1]{%
	\ifthenelse{\equal{\GLOBALheaderfaimport}{false}}{%
		\global\def\GLOBALheaderfaimport {true}%
		\usepackage{fontawesome5}%
	}{}%
	\immediate\write\fileheaderitems{\unexpanded{\faIcon{#1}}\LOCALpercentchar}%
}

% Agrega un espacio vertical
%	#1	Tamaño del espacio
\newcommand{\addheaderspace}[1]{%
	\immediate\write\fileheaderitems{\unexpanded{\hspace{#1}}\LOCALpercentchar}%
}

% Agrega una imagen
%	#1	Dirección de la imagen
%	#2	Tamaño de la imagen {small, normal, large}
\newcommand{\addheaderimage}[2]{%
	\ifthenelse{\equal{#2}{small}}{%
		\immediate\write\fileheaderitems{\unexpanded{\raisebox{-0.01\baselineskip}{\includegraphics[height=0.65\baselineskip]{#1}}}\LOCALpercentchar}%
	}{%
	\ifthenelse{\equal{#2}{normal}}{%
		\immediate\write\fileheaderitems{\unexpanded{\raisebox{-0.15\baselineskip}{\includegraphics[height=1\baselineskip]{#1}}}\LOCALpercentchar}%
	}{%
	\ifthenelse{\equal{#2}{large}}{%
		\immediate\write\fileheaderitems{\unexpanded{\raisebox{-0.25\baselineskip}{\includegraphics[height=1.25\baselineskip]{#1}}}\LOCALpercentchar}%
	}{%
		\throwerror{\addheaderimage}{Tamano de la imagen debe ser: small, normal o large}}}%
	}%
}

% Escribe el encabezado del CV
\newcommand{\writeheader}{%
	% Escribe título
	\ifthenelse{\equal{\writetitleheader}{true}}{%
		\begin{center}%
			\fontsize{16}{12} \selectfont \vspace{0.3cm} \textcolor{\titlecolor}{\headertitle}%
		\end{center}
		\vspace{0.1cm}%
	}{}%
	
	% Nombre
	\noindent {\fontsizemaintitle \stylemaintitle \textcolor{\titlecolor}{\name}}%
	\ifthenelse{\equal{\writebirthdayheader}{true}}{%
		\quad \emph{\textcolor{\headertextcolor}{\birthmonth \birthday, \birthyear}}%
	}{}%
	\ifthenelse{\equal{\writelastchangeheader}{true}}{%
		\hfill \quad {\scriptsize \textcolor{\lastchangeheadercolor}{\nomlastchange\ \today}}%
	}{}%
	\\ \vspace{-1em}%
	
	% Escribe items del header
	\hypersetup{urlcolor=\headerurlcolor}%
	\vspace{\headeritemsmargin cm}%
	{%
		\noindent%
		\input{\jobname.hitems}%
	}
	\vspace{0.55em}%
	\hypersetup{urlcolor=\urlcolor}%
}

% Título de las secciones principales
%	#1	Título
\newcommand*{\roottitle}[1]{%
	\emptyvarerr{\roottitle}{#1}{Titulo no definido}%
	\section*{#1}\vspace{-0.3em}\nopagebreak[4]%
	\addcontentsline{toc}{section}{#1}%
}

% Crea un acrónimo de forma fácil
%	#1	Acrónimo
\newcommand*{\acr}[1]{%
	\textscale{.85}{#1}%
}

% Crea un campo de texto
% 	#1	Texto
\newcommand{\bodytext}[1]{%
	\nopagebreak[4]%
	\begin{indentsection}%
		\item[] #1%
	\end{indentsection}
	\pagebreak[2]%
}

% Crea una entrada en la tabla de datos personales
%	#1	Nombre entrada
%	#2	Datos entrada
\newcommand{\personaltableentry}[2]{%
	\emptyvarerr{\personaltableentry}{#1}{Nombre entrada no definido}%
	\emptyvarerr{\personaltableentry}{#2}{Datos entrada no definidos}%
	\textcolor{\personaltblentcolor}{\textbf{#1:}} & #2 \\%
}

% Inserta un objeto en un elemento institución
%	#1	Cargo
%	#2	Indica si escribe fechas o no
%	#3	Fecha inicial
%	#4	Fecha final
%	#5	Ancho título
%	#6	Ancho fecha
%	#7	Descripción
\newcommand{\newinstitutionentry}[7]{%
	\ifthenelse{\equal{\GLOBALinstitutionenabled}{true}}{}{%
		\throwwarning{Funciones \noexpand\newinstitutionentry, \noexpand\institutionentry o \noexpand\institutionentryshort no pueden usarse fuera del entorno \noexpand\institution}\stop%
	}%
	\emptyvarerr{\newinstitutionentry}{#1}{Cargo o posición no definido}%
	\emptyvarerr{\newinstitutionentry}{#5}{Dimensiones de titulo no definido}%
	\emptyvarerr{\newinstitutionentry}{#6}{Dimensiones de fecha no definido}%
	\nopagebreak[4]%
	\begin{indentsectiondouble}%
		\item []%
		\noindent%
		\begin{minipage}{1\linewidth}%
			\begin{minipage}[t][][t]{#5}%
				{\textcolor{\instentrytitlecolor}{\textbf{#1}} ~ \\ \vspace{0.7em}}%
			\end{minipage}
			\hfill%
			\begin{minipage}[t][][t]{#6}%
				\begin{flushright}%
					\noindent \textcolor{\datecolor}{\emph{#2#4#3}}%
				\end{flushright}
			\end{minipage}
		\end{minipage}
		\vspace{-1.1\baselineskip}%
		\break%
		\ifx\hfuzz#7\hfuzz%
			\vspace{-0.75\baselineskip}%
			\global\def\GLOBALinstitutionshortend {true}%
		\else%
			\begin{minipage}{1.0\linewidth}%
				\bodytext{#7}%
				\par%
			\end{minipage}\vspace{0.1\baselineskip}%
			\global\def\GLOBALinstitutionshortend {false}%
		\fi%
	\end{indentsectiondouble}
}

% Entrada en institución
%	#1	Cargo
%	#2	Fecha inicial
%	#3	Fecha final
%	#4	Descripción
\newcommand{\institutionentry}[4]{%
	\newinstitutionentry{#1}{#2 }{ #3}{\dateseparator}{.687\linewidth}{.31\linewidth}{#4}%
}

% Entrada en institución con fecha corta
%	#1	Cargo
%	#2	Fecha inicial
%	#3	Fecha final
%	#4	Descripción
\newcommand{\institutionentryshort}[4]{%
	\newinstitutionentry{#1}{#2 }{ #3}{\dateseparator}{.833\linewidth}{.163\linewidth}{#4}%
}

% Entrada en institución sin fecha
%	#1	Cargo
%	#2	Descripción
\newcommand{\institutionentrynodate}[2]{%
	\newinstitutionentry{#1}{}{}{}{\linewidth}{0\linewidth}{#2}%
}

% Otro tipo de entrada en cvblock
%	#1	Título
%	#2	Contenido
\newcommand{\otherentry}[2]{%
	\begin{basedescript}%
		{\setlength{\leftmargin}{\doubleparindent}}%
		\item[\hspace{\newparindent}\textcolor{\otherentrytitlecolor}{\textbf{#1}}] #2%
	\end{basedescript}
}

% Inserta un elemento en el bloque de firma
%	#1	Elemento en la firma
\newcommand{\signatureentry}[1]{%
	\texttt{\MakeUppercase{#1}} \\%
}

% Crea un bull
\newcommand{\sbullet}{%
	\ \ \raisebox{0.11em}[-1em][-1em]{\tiny $\bullet$} \ \ %
}

% Variaciones de vspace
\newcommand{\breakvspace}[1]{%
	\pagebreak[2] \vspace{#1} \pagebreak[2]%
}
\newcommand{\nobreakvspace}[1]{%
	\nopagebreak[4] \vspace{#1} \nopagebreak[4]%
}

% Apóstrofe
\newcommand{\apo}{%
	\raisebox{-.18ex}{'}{\hspace{-.1em}}%
}

% Inserta un texto entre comillas
\newcommand{\quotes}[1]{%
	\enquote*{#1}%
}

% Inserta un texto entre comillas y negrita
%	#1 	Texto
\newcommand{\quotesbf}[1]{%
	\quotes{\textbf{#1}}%
}

% Inserta un texto entre comillas e itálico
%	#1 	Texto
\newcommand{\quotesit}[1]{%
	\quotes{\textit{#1}}%
}

% Inserta un texto entre comillas y typewriter
%	#1 	Texto
\newcommand{\quotesttt}[1]{%
	\quotes{\texttt{#1}}%
}

% Inserta un texto entre comillas dobles
%	#1 	Texto
\newcommand{\doublequotes}[1]{%
	\enquote{#1}%
}

% Inserta un texto con el formato de enlace
% 	#1 	Enlace
\newcommand{\hreftext}[1]{%
	\ifthenelse{\equal{\fonturl}{same}}{%
		#1%
	}{%
	\ifthenelse{\equal{\fonturl}{tt}}{%
		\texttt{#1}%
	}{%
	\ifthenelse{\equal{\fonturl}{rm}}{%
		\textrm{#1}%
	}{%
	\ifthenelse{\equal{\fonturl}{sf}}{%
		\textsf{#1}%
	}{}}}}%
}

% Inserta un email con un link cliqueable
\newcommand{\insertemail}[1]{%
	\href{mailto:#1}{\hreftext{#1}}%
}

% Inserta un teléfono celular
\newcommand{\insertphone}[1]{%
	\href{tel:#1}{\hreftext{#1}}%
}

% Actualiza el padding de las celdas de las tablas
%	#1	Padding horizontal (em)
%	#2	Padding vertical (em)
\newcommand{\settablecellpadding}[2]{%
	\emptyvarerr{\settablecellpadding}{#1}{Padding horizontal no definido}%
	\emptyvarerr{\settablecellpadding}{#2}{Padding vertical no definido}%
	\setlength{\tabcolsep}{#1 em} % Horizontal
	\def\arraystretch {#2} % Vertical
}

% Definición de colores
\definecolor{backcolour}{rgb}{0.95, 0.95, 0.92}
\definecolor{codegray}{rgb}{0.5, 0.5, 0.5}
\definecolor{codegreen}{rgb}{0, 0.6, 0}
\definecolor{codepurple}{rgb}{0.58, 0, 0.82}
\definecolor{dark-blue}{rgb}{0.15, 0.15, 0.40}
\definecolor{dgray}{RGB}{104, 108, 113}
\definecolor{dkgreen}{rgb}{0, 0.6, 0}
\definecolor{gray}{rgb}{0.5, 0.5, 0.5}
\definecolor{lbrown}{RGB}{255, 252, 249}
\definecolor{lgray}{RGB}{180, 180, 180}
\definecolor{lyellow}{rgb}{1.0, 1.0, 0.88}
\definecolor{mauve}{rgb}{0.58, 0, 0.82}
\definecolor{mygray}{rgb}{0.5, 0.5, 0.5}
\definecolor{mygreen}{rgb}{0, 0.6, 0}
\definecolor{mylilas}{RGB}{170, 55, 241}
\definecolor{ocre}{RGB}{243, 102, 25}

% Definición de variables globales
\global\def\GLOBALinstitutionenabled {false}
\global\def\GLOBALinstitutionicon {}
\global\def\GLOBALinstitutionshortend {false}

% Crea una sección identada
\newenvironment{indentsection}{%
	\begin{list}{}{%
		\setlength{\leftmargin}{0.75\newparindent}%
		\setlength{\parsep}{0pt}%
		\setlength{\parskip}{0pt}%
		\setlength{\itemsep}{0pt}%
		\setlength{\topsep}{0pt}}%
	}{%
	\end{list}
}

% Crea una sección identada doble
\newenvironment{indentsectiondouble}{%
	\begin{list}{}{%
			\setlength{\leftmargin}{1.5\newparindent}%
			\setlength{\parsep}{0pt}%
			\setlength{\parskip}{0pt}%
			\setlength{\itemsep}{0pt}%
			\setlength{\topsep}{0pt}}%
	}{%
	\end{list}
}

% Crea un bloque de contenido
%	#1	Título
%	#2	Estilo de línea
\newenvironment{cvblock}[2][]{%
	\needspace{1\baselineskip}%
	\textcolor{\cvblocklinecolor}{%
		\ifx\hfuzz#1\hfuzz%
			\hrule%
		\else%
			\ifthenelse{\equal{#1}{..}}{%
				\noindent%
				\hspace*{-1.65em}%
				\makebox[1.01\textwidth][l]{\dotfill}%
				\vspace{-2em}
			}{%
			\ifthenelse{\equal{#1}{none}}{%
				\vspace{-1em}%
			}{%
				\throwwarning{Estilo de linea invalido}%
				\stop%
			}}%
		\fi%
	}%
	\ifx\hfuzz#2\hfuzz%
		\throwwarning{Titulo del bloque no definido}%
		\stop%
	\else%
		\roottitle{#2}%
	\fi%
	\vspace{0.15em}%
	\global\def\GLOBALhasinstitution {false}%
	\global\def\GLOBALinstitutionshortend {false}%
	}{%
	\ifthenelse{\equal{\GLOBALhasinstitution}{false}}{}{%
		\ifthenelse{\equal{\GLOBALinstitutionshortend}{false}}{%
			\vspace{0.45\baselineskip}%
		}{%
			\vspace{-0.7\baselineskip}%
		}%
	}%
	\par%
}

% Párrafo de descripción
%	#1	Estilo de línea
\newenvironment{summary}[1][]{%
	\begin{cvblock}[#1]{\nomsummary}%
		\begingroup%
		\vspace{-1em}%
		\begin{multicols}{2}%
			\noindent%
		}{%
		\end{multicols}
		\vspace{-0.75\baselineskip}%
		\par%
		\endgroup%
	\end{cvblock}
	\vspace{0.49\baselineskip}%
}

% Párrafo de descripción con foto de perfil
%	#1	Estilo de línea
\newenvironment{photosummary}[1][]{%
	\begin{cvblock}[#1]{\nomsummary}%
		\vspace{0.1em}%
		\noindent%
		\begin{minipage}[t]{\userphotosize cm}%
			\begin{flushleft}%
				\begingroup%
				\setlength{\fboxsep}{0pt}%
				~ \\ \vspace{-0.75em}%
				\ifthenelse{\equal{\showuserphotoborder}{true}}{%
					\noindent \textcolor{\userphotobordercolor}{\fbox{\includegraphics[width=0.99\linewidth]{\photo}}}%
				}{%
					\noindent \includegraphics[width=\userphotosize cm,height=\userphotosize cm]{\photo}%
				}%
				\vspace{-0.7em}%
				\endgroup%
			\end{flushleft}
		\end{minipage}
		\hspace*{0.54cm}%
		\begingroup%
		\def\arraystretch{1.3}%
		\begin{minipage}[t]{\dimexpr\linewidth-\userphotosize cm-0.57cm}%
		}{%
		\end{minipage}
		\endgroup%
		\par%
	\end{cvblock}
	\vspace{1.6em}%
}

% Tabla de datos personales
%	#1	Estilo de línea
\newenvironment{personaltabledata}[1][]{%
	\begin{cvblock}[#1]{\nompersonaldata}%
		\vspace{0.1em}%
		\noindent%
		\ifthenelse{\equal{\personaltabledatapic}{true}}{%
			\begin{minipage}[t]{\userphotosize cm}%
				\begin{flushleft}%
					\begingroup%
					\setlength{\fboxsep}{0pt}%
					~ \\ \vspace{-0.95em}%
					\ifthenelse{\equal{\showuserphotoborder}{true}}{%
						\noindent \textcolor{\userphotobordercolor}{\fbox{\includegraphics[width=0.99\linewidth]{\photo}}}%
					}{%
						\noindent \includegraphics[width=\userphotosize cm,height=\userphotosize cm]{\photo}%
					}%
					\endgroup%
				\end{flushleft}
			\end{minipage}
		}{}%
		\hspace*{0.34cm}%
		\begingroup%
		\def\arraystretch{1.3}%
		\begin{minipage}[t]{\linewidth-\userphotosize cm-0.37cm}%
			\vspace{-0.95cm}%
			\begin{table}[H]%
				\begin{tabular}{ll}%
				}{%
				\end{tabular}
			\end{table}
		\end{minipage}
		\endgroup%
		\par%
	\end{cvblock}
	\vspace{1.15em}%
}

% Nueva institución
%	#1	Opcional: url o link institución
%	#2	Nombre institución
%	#3	Ubicación institución
\newenvironment{institution}[3][]{%
	\needspace{2\baselineskip}%
	\ifthenelse{\equal{\institutiontitlebold}{true}}{%
		\def\LOCALinstitutiontitle {\textbf{#2}}%
	}{%
		\def\LOCALinstitutiontitle {#2}%
	}%
	\nopagebreak[4]%
	\hypersetup{urlcolor=\instnamecolor}%
	\begin{indentsection}%
		\item []%
		\noindent%
		\ifx\hfuzz#3\hfuzz%
			\begin{minipage}{\linewidth}%
				\GLOBALinstitutionicon%
				\ifx\hfuzz#1\hfuzz%
					\textscale{1.1}{\textcolor{\instnamecolor}{\LOCALinstitutiontitle}}%
				\else%
					\textscale{1.1}{\href{#1}{\LOCALinstitutiontitle}}%
				\fi%
			\end{minipage}
		\else%
			\begin{minipage}[t][][t]{.558\linewidth}%
				\GLOBALinstitutionicon%
				\ifx\hfuzz#1\hfuzz%
					\textscale{1.1}{\textcolor{\instnamecolor}{\LOCALinstitutiontitle}}%
				\else%
					\textscale{1.1}{\href{#1}{\LOCALinstitutiontitle}}%
				\fi%
			\end{minipage}
			\begin{minipage}[t][][t]{.44\linewidth}%
				\begin{flushright}%
					\noindent \textcolor{\instregioncolor}{\textsc{#3}}%
				\end{flushright}
			\end{minipage}
		\fi%
		\vspace{-\baselineskip}%
		\break%
	\end{indentsection}
	\hypersetup{urlcolor=\urlcolor}%
	\vspace{-1.5\baselineskip}%
	\noindent%
	\global\def\GLOBALinstitutionicon {}%
	\global\def\GLOBALinstitutionenabled {true}
	}{%
	\global\def\GLOBALinstitutionenabled {false}%
	\global\def\GLOBALhasinstitution {true}%
	\par%
	\vspace{0.2\baselineskip}%
	\nopagebreak[4]%
}

% Nueva institución con ícono
%	#1	Opcional: url o link institución
%	#2	Nombre institución
%	#3	Ubicación institución
%	#4	Ícono de la institución
\newenvironment{institutionicon}[4][]{%
	\ifthenelse{\equal{\institutioniconsize}{small}}{%
		\global\def\GLOBALinstitutionicon {\raisebox{-0.01\baselineskip}{\includegraphics[height=0.65\baselineskip,width=0.65\baselineskip]{#4}}\hspace{0.5em}}%
	}{%
	\ifthenelse{\equal{\institutioniconsize}{normal}}{%
		\global\def\GLOBALinstitutionicon {\raisebox{-0.15\baselineskip}{\includegraphics[height=1\baselineskip,width=1\baselineskip]{#4}}\hspace{0.65em}}%
	}{%
	\ifthenelse{\equal{\institutioniconsize}{large}}{%
		\global\def\GLOBALinstitutionicon {\raisebox{-0.25\baselineskip}{\includegraphics[height=1.25\baselineskip,width=1.25\baselineskip]{#4}}\hspace{0.75em}}%
	}{%
		\throwbadconfigondoc{Configuracion tamano icono incorrecto}{\institutioniconsize}{small,normal,large}}}%
	}%
	\begin{institution}[#1]{#2}{#3}%
}{%
	\end{institution}
}

% Crea la firma del usuario
\newenvironment{signature}{%
	\vfill%
	\begin{flushright}%
		\begin{tabular}{c}%
			\includegraphics[width=\usersignaturesize cm]{\sign} \\%
		}{%
		\end{tabular}
	\end{flushright}
	\vspace{-1em}%
}

% Itemize en negrita
%	#1	Parámetros opcionales
\newenvironment{itemizebf}[1][]{%
	\begin{itemize}[font=\bfseries,#1]%
	}{%
	\end{itemize}
}

% Enumerate en negrita
%	#1	Parámetros opcionales
\newenvironment{enumeratebf}[1][]{%
	\begin{enumerate}[font=\bfseries,#1]%
	}{%
	\end{enumerate}
}

% -----------------------------------------------------------------------------
% CONFIGURACIÓN INICIAL DEL DOCUMENTO
% -----------------------------------------------------------------------------
% Se revisa si las variables no han sido borradas
\checkvardefined{\birthday}
\checkvardefined{\birthmonth}
\checkvardefined{\birthyear}
\checkvardefined{\email}
\checkvardefined{\name}
\checkvardefined{\phonenumber}

% Se añade \xspace a las variables
\makeatletter
	\g@addto@macro\birthday\xspace
	\g@addto@macro\birthmonth\xspace
	\g@addto@macro\birthyear\xspace
	\g@addto@macro\name\xspace
	\g@addto@macro\phonenumber\xspace
\makeatother

% Se define metadata del pdf
\ifthenelse{\equal{\cfgshowbookmarkmenu}{true}}{
	\def\cdfpagemodepdf {UseOutlines}
}{
	\def\cdfpagemodepdf {UseNone}
}
\hypersetup{
	keeppdfinfo,
	bookmarksopen={\cfgpdfbookmarkopen},
	bookmarksopenlevel={\cfgbookmarksopenlevel},
	bookmarkstype={toc},
	pdfauthor={\name},
	pdfcenterwindow={\cfgpdfcenterwindow},
	pdfcopyright={\headertitle, \name. Email: \email. Phone: \phonenumber},
	pdfcreator={LaTeX},
	pdfdisplaydoctitle={\cfgpdfdisplaydoctitle},
	pdffitwindow={\cfgpdffitwindow},
	pdfinfo={
		Author.Email={\email},
		Author.Name={\name},
		Author.Phone={\phonenumber},
		Template.Author.Alias={ppizarror},
		Template.Author.Email={pablo@ppizarror.com},
		Template.Author.Name={Pablo Pizarro R.},
		Template.Author.Web={https://ppizarror.com/},
		Template.Date={09/11/2021},
		Template.Encoding={UTF-8},
		Template.License.Type={MIT},
		Template.License.Web={https://opensource.org/licenses/MIT/},
		Template.Name={Professional-CV},
		Template.Type={Normal},
		Template.Version.Dev={3.2.1-2},
		Template.Version.Hash={D9F4D83349A527A51FCBF3EDC39E9968},
		Template.Version.Release={3.2.1},
		Template.Web.Dev={https://github.com/Template-Latex/Professional-CV/},
		Template.Web.Manual={https://latex.ppizarror.com/Professional-CV/}
	},
	pdfkeywords={CV, \name, \headertitle, \email},
	pdfmenubar={\cfgpdfmenubar},
	pdfpagelayout={\cfgpdfpagemode},
	pdfpagemode={\cdfpagemodepdf},
	pdfproducer={Professional-CV v3.2.1 | (Pablo Pizarro R.) ppizarror.com},
	pdfremotestartview={Fit},
	pdfstartpage={1},
	pdfstartview={\cfgpdfpageview},
	pdfsubject={CV \name},
	pdftitle={\headertitle~ \name},
	pdftoolbar={\cfgpdftoolbar},
	pdftype={Text}
}

% Establece la carpeta de imágenes por defecto
\graphicspath{{./img}}

% Ajuste del entrelineado
\renewcommand{\baselinestretch}{\documentinterline}

% Ajuste de tablas
\settablecellpadding{0.5}{1}

% Configuración de los colores
\color{\maintextcolor} % Color principal
\arrayrulecolor{\tablelinecolor} % Color de las líneas de las tablas
\sethlcolor{\highlightcolor} % Color del subrayado por defecto
\ifthenelse{\equal{\showborderonlinks}{true}}{
	\hypersetup{
		% Color de links con borde
		citebordercolor=\numcitecolor,
		linkbordercolor=\urlcolor,
		urlbordercolor=\urlcolor
	}
}{
	\hypersetup{
		% Color de links sin borde
		hidelinks,
		colorlinks=true,
		citecolor=\numcitecolor,
		linkcolor=\urlcolor,
		urlcolor=\urlcolor
	}
}
\ifthenelse{\equal{\colorpage}{white}}{}{
	\pagecolor{\colorpage}
}

% Configuración de la identación
\newlength{\newparindent}
\addtolength{\newparindent}{\parindent}
\newlength{\doubleparindent}
\addtolength{\doubleparindent}{\parindent}
\addtolength{\doubleparindent}{\parindent}

% Configuración de hbox y vbox
\hfuzz=200pt
\vfuzz=200pt
\hbadness=\maxdimen
\vbadness=\maxdimen

% Se activa el word-wrap para textos con \texttt{}
\ttfamily \hyphenchar\the\font=`\-

% Se define el tipo de texto de los url
\urlstyle{\fonturl}

% Se define la versión menor a compilar
\ifthenelse{\equal{\compilertype}{pdf2latex}}{
	% Nivel de compresión
	\pdfcompresslevel=9
	
	% El óptimo es 2, según
	% https://texdoc.org/serve/pdftex-a.pdf/0 p.20
	\pdfdecimaldigits=2
	
	% Inclusión de PDF
	\pdfinclusionerrorlevel=0
	
	% Versión
	\pdfminorversion=\pdfcompileversion
	
	% Compresión de objetos
	\pdfobjcompresslevel=2
}{}

% Configura items
\AfterEndEnvironment{itemize}{\vspace{-0.4\baselineskip}}
\AfterEndEnvironment{enumerate}{\vspace{-0.4\baselineskip}}

% -----------------------------------------------------------------------------
% DECLARACIÓN DE CARACTERES ADICIONALES
% -----------------------------------------------------------------------------
% Definición letras griegas
\def\Alpha{A}
\def\Beta{B}
\def\Chi{X}
\def\Epsilon{E}
\def\Eta{H}
\def\Iota{I}
\def\Kappa{K}
\def\Mu{M}
\def\Nu{N}
\def\Omicron{O}
\def\omicron{o}
\def\Rho{P}
\def\Tau{T}
\def\Zeta{Z}

\def\LOCALunknownchar {\ensuremath{\mathrm{UNKNOWN\;CHAR}}}

% Definición de símbolos
\makeatletter
\newsavebox{\@brxanglelr}
\newcommand{\llangle}[1][]{\savebox{\@brxanglelr}{\(\m@th{#1\langle}\)}%
	\mathopen{\copy\@brxanglelr\kern-0.5\wd\@brxanglelr\usebox{\@brxanglelr}}}
\newcommand{\rrangle}[1][]{\savebox{\@brxanglelr}{\(\m@th{#1\rangle}\)}%
	\mathclose{\copy\@brxanglelr\kern-0.5\wd\@brxanglelr\usebox{\@brxanglelr}}}
\makeatother

% Creación de comandos si no existen
\ifx\DeclareUnicodeCharacter\undefined
	\def\DeclareUnicodeCharacter#1#2{%
		\def\tmp{#2}\uccode`\~="#1 \catcode"#1 \active%
		\uppercase{\global\let~\tmp}%
		\uccode`\~=0%
	}
\fi
\ifx\mapsfrom\undefined
	\newcommand\mapsfrom{\mathrel{\reflectbox{\ensuremath{\mapsto}}}}
\fi

% Agrega caracteres unicode
\ifdefined\DeclareUnicodeCharacter
\DeclareUnicodeCharacter{00A0}{~}
\DeclareUnicodeCharacter{00A1}{\textexclamdown}
\DeclareUnicodeCharacter{00A2}{\textcent}
\DeclareUnicodeCharacter{00A3}{\pounds}
\DeclareUnicodeCharacter{00A4}{\textcurrency}
\DeclareUnicodeCharacter{00A5}{\textyen}
\DeclareUnicodeCharacter{00A6}{\textbrokenbar}
\DeclareUnicodeCharacter{00A7}{{\mathhexbox 278}}
\DeclareUnicodeCharacter{00A8}{\"{ }}
\DeclareUnicodeCharacter{00A9}{\copyright}
\DeclareUnicodeCharacter{00AA}{\textordfeminine}
\DeclareUnicodeCharacter{00AB}{\guillemotleft}
\DeclareUnicodeCharacter{00AC}{\ensuremath{\neg}}
\DeclareUnicodeCharacter{00AE}{\textregistered}
\DeclareUnicodeCharacter{00AF}{\textasciimacron}
\DeclareUnicodeCharacter{00B0}{\textsuperscript{o}}
\DeclareUnicodeCharacter{00B1}{\ensuremath{\pm}}
\DeclareUnicodeCharacter{00B2}{\textsuperscript{2}}
\DeclareUnicodeCharacter{00B3}{\textsuperscript{3}}
\DeclareUnicodeCharacter{00B5}{\textmu}
\DeclareUnicodeCharacter{00B6}{{\mathhexbox 27B}}
\DeclareUnicodeCharacter{00B7}{\ensuremath{\cdot}}
\DeclareUnicodeCharacter{00B9}{\textsuperscript{1}}
\DeclareUnicodeCharacter{00BA}{\textordmasculine}
\DeclareUnicodeCharacter{00BB}{\guillemotright}
\DeclareUnicodeCharacter{00BC}{\ensuremath{\frac{1}{4}}}
\DeclareUnicodeCharacter{00BD}{\ensuremath{\frac{1}{2}}}
\DeclareUnicodeCharacter{00BE}{\ensuremath{\frac{3}{4}}}
\DeclareUnicodeCharacter{00BF}{\textquestiondown}
\DeclareUnicodeCharacter{00D7}{\ensuremath{\times}}
\DeclareUnicodeCharacter{00F7}{\ensuremath{\div}}
\DeclareUnicodeCharacter{0131}{\ensuremath{\imath}}
\DeclareUnicodeCharacter{02102}{\ensuremath{\mathbb{C}}}
\DeclareUnicodeCharacter{0210D}{\ensuremath{\mathbb{H}}}
\DeclareUnicodeCharacter{02115}{\ensuremath{\mathbb{N}}}
\DeclareUnicodeCharacter{02119}{\ensuremath{\mathbb{P}}}
\DeclareUnicodeCharacter{0211A}{\ensuremath{\mathbb{Q}}}
\DeclareUnicodeCharacter{0211D}{\ensuremath{\mathbb{R}}}
\DeclareUnicodeCharacter{02124}{\ensuremath{\mathbb{Z}}}
\DeclareUnicodeCharacter{0237}{\ensuremath{\jmath}}
\DeclareUnicodeCharacter{02B0}{\ensuremath{^h}}
\DeclareUnicodeCharacter{02B2}{\ensuremath{^j}}
\DeclareUnicodeCharacter{02B3}{\ensuremath{^r}}
\DeclareUnicodeCharacter{02B7}{\ensuremath{^w}}
\DeclareUnicodeCharacter{02B8}{\ensuremath{^y}}
\DeclareUnicodeCharacter{02E1}{\ensuremath{^l}}
\DeclareUnicodeCharacter{02E2}{\ensuremath{^s}}
\DeclareUnicodeCharacter{02E3}{\ensuremath{^x}}
\DeclareUnicodeCharacter{0302}{\ensuremath{\hat{\phantom{x}}}}
\DeclareUnicodeCharacter{0308}{\ensuremath{\ddot{\phantom{x}}}}
\DeclareUnicodeCharacter{0332}{\ensuremath{\underline{\phantom{x}}}}
\DeclareUnicodeCharacter{0391}{\ensuremath{\Alpha}}
\DeclareUnicodeCharacter{0392}{\ensuremath{\Beta}}
\DeclareUnicodeCharacter{0393}{\ensuremath{\Gamma}}
\DeclareUnicodeCharacter{0394}{\ensuremath{\Delta}}
\DeclareUnicodeCharacter{0395}{\ensuremath{\Epsilon}}
\DeclareUnicodeCharacter{0396}{\ensuremath{\Zeta}}
\DeclareUnicodeCharacter{0397}{\ensuremath{\Eta}}
\DeclareUnicodeCharacter{0398}{\ensuremath{\Theta}}
\DeclareUnicodeCharacter{0399}{\Iota}
\DeclareUnicodeCharacter{039A}{\Kappa}
\DeclareUnicodeCharacter{039B}{\ensuremath{\Lambda}}
\DeclareUnicodeCharacter{039C}{\Mu}
\DeclareUnicodeCharacter{039D}{\Nu}
\DeclareUnicodeCharacter{039E}{\ensuremath{\Xi}}
\DeclareUnicodeCharacter{039F}{\Omicron}
\DeclareUnicodeCharacter{03A0}{\ensuremath{\Pi}}
\DeclareUnicodeCharacter{03A1}{\Rho}
\DeclareUnicodeCharacter{03A3}{\ensuremath{\Sigma}}
\DeclareUnicodeCharacter{03A4}{\Tau}
\DeclareUnicodeCharacter{03A5}{\ensuremath{\Upsilon}}
\DeclareUnicodeCharacter{03A6}{\ensuremath{\Phi}}
\DeclareUnicodeCharacter{03A7}{\Chi}
\DeclareUnicodeCharacter{03A8}{\ensuremath{\Psi}}
\DeclareUnicodeCharacter{03A9}{\ensuremath{\Omega}}
\DeclareUnicodeCharacter{03B1}{\ensuremath{\alpha}}
\DeclareUnicodeCharacter{03B2}{\ensuremath{\beta}}
\DeclareUnicodeCharacter{03B3}{\ensuremath{\gamma}}
\DeclareUnicodeCharacter{03B4}{\ensuremath{\delta}}
\DeclareUnicodeCharacter{03B5}{\ensuremath{\varepsilon}}
\DeclareUnicodeCharacter{03B6}{\ensuremath{\zeta}}
\DeclareUnicodeCharacter{03B7}{\ensuremath{\eta}}
\DeclareUnicodeCharacter{03B8}{\ensuremath{\theta}}
\DeclareUnicodeCharacter{03B9}{\ensuremath{\iota}}
\DeclareUnicodeCharacter{03BA}{\ensuremath{\kappa}}
\DeclareUnicodeCharacter{03BB}{\ensuremath{\lambda}}
\DeclareUnicodeCharacter{03BC}{\ensuremath{\mu}}
\DeclareUnicodeCharacter{03BD}{\ensuremath{\nu}}
\DeclareUnicodeCharacter{03BE}{\ensuremath{\xi}}
\DeclareUnicodeCharacter{03BF}{\omicron}
\DeclareUnicodeCharacter{03C0}{\ensuremath{\pi}}
\DeclareUnicodeCharacter{03C1}{\ensuremath{\rho}}
\DeclareUnicodeCharacter{03C2}{\ensuremath{\varsigma}}
\DeclareUnicodeCharacter{03C3}{\ensuremath{\sigma}}
\DeclareUnicodeCharacter{03C4}{\ensuremath{\tau}}
\DeclareUnicodeCharacter{03C5}{\ensuremath{\upsilon}}
\DeclareUnicodeCharacter{03C6}{\ensuremath{\phi}}
\DeclareUnicodeCharacter{03C7}{\ensuremath{\chi}}
\DeclareUnicodeCharacter{03C8}{\ensuremath{\psi}}
\DeclareUnicodeCharacter{03C9}{\ensuremath{\omega}}
\DeclareUnicodeCharacter{03D0}{\ensuremath{\beta}}
\DeclareUnicodeCharacter{03D1}{\ensuremath{\theta}}
\DeclareUnicodeCharacter{03D5}{\ensuremath{\phi}}
\DeclareUnicodeCharacter{03D6}{\ensuremath{\pi}}
\DeclareUnicodeCharacter{03D8}{\ensuremath{Q}}
\DeclareUnicodeCharacter{03D9}{\ensuremath{q}}
\DeclareUnicodeCharacter{03DA}{\ensuremath{S}}
\DeclareUnicodeCharacter{03DB}{\ensuremath{s}}
\DeclareUnicodeCharacter{03DC}{\ensuremath{D}}
\DeclareUnicodeCharacter{03DD}{\ensuremath{d}}
\DeclareUnicodeCharacter{03DE}{\ensuremath{K}}
\DeclareUnicodeCharacter{03DF}{\ensuremath{k}}
\DeclareUnicodeCharacter{03E0}{\ensuremath{S}}
\DeclareUnicodeCharacter{03E1}{\ensuremath{s}}
\DeclareUnicodeCharacter{03F0}{\ensuremath{\varkappa}}
\DeclareUnicodeCharacter{03F1}{\ensuremath{\rho}}
\DeclareUnicodeCharacter{03F5}{\ensuremath{\epsilon}}
\DeclareUnicodeCharacter{03F6}{\ensuremath{\backepsilon}}
\DeclareUnicodeCharacter{041F}{\LOCALunknownchar}
\DeclareUnicodeCharacter{0432}{\LOCALunknownchar}
\DeclareUnicodeCharacter{0435}{\LOCALunknownchar}
\DeclareUnicodeCharacter{0438}{\LOCALunknownchar}
\DeclareUnicodeCharacter{043C}{\LOCALunknownchar}
\DeclareUnicodeCharacter{0440}{\LOCALunknownchar}
\DeclareUnicodeCharacter{0442}{\LOCALunknownchar}
\DeclareUnicodeCharacter{0BA8}{\LOCALunknownchar}
\DeclareUnicodeCharacter{0BBF}{\LOCALunknownchar}
\DeclareUnicodeCharacter{1100}{\LOCALunknownchar}
\DeclareUnicodeCharacter{11F9}{\LOCALunknownchar}
\DeclareUnicodeCharacter{1D2C}{\ensuremath{^A}}
\DeclareUnicodeCharacter{1D2E}{\ensuremath{^B}}
\DeclareUnicodeCharacter{1D30}{\ensuremath{^D}}
\DeclareUnicodeCharacter{1D31}{\ensuremath{^E}}
\DeclareUnicodeCharacter{1D33}{\ensuremath{^G}}
\DeclareUnicodeCharacter{1D34}{\ensuremath{^H}}
\DeclareUnicodeCharacter{1D35}{\ensuremath{^I}}
\DeclareUnicodeCharacter{1D36}{\ensuremath{^J}}
\DeclareUnicodeCharacter{1D37}{\ensuremath{^K}}
\DeclareUnicodeCharacter{1D38}{\ensuremath{^L}}
\DeclareUnicodeCharacter{1D39}{\ensuremath{^M}}
\DeclareUnicodeCharacter{1D3A}{\ensuremath{^N}}
\DeclareUnicodeCharacter{1D3C}{\ensuremath{^O}}
\DeclareUnicodeCharacter{1D3E}{\ensuremath{^P}}
\DeclareUnicodeCharacter{1D3F}{\ensuremath{^R}}
\DeclareUnicodeCharacter{1D40}{\ensuremath{^T}}
\DeclareUnicodeCharacter{1D41}{\ensuremath{^U}}
\DeclareUnicodeCharacter{1D42}{\ensuremath{^W}}
\DeclareUnicodeCharacter{1D43}{\ensuremath{^a}}
\DeclareUnicodeCharacter{1D47}{\ensuremath{^b}}
\DeclareUnicodeCharacter{1D48}{\ensuremath{^d}}
\DeclareUnicodeCharacter{1D49}{\ensuremath{^e}}
\DeclareUnicodeCharacter{1D4D}{\ensuremath{^g}}
\DeclareUnicodeCharacter{1D4F}{\ensuremath{^k}}
\DeclareUnicodeCharacter{1D50}{\ensuremath{^m}}
\DeclareUnicodeCharacter{1D52}{\ensuremath{^o}}
\DeclareUnicodeCharacter{1D538}{\ensuremath{\mathbb{A}}}
\DeclareUnicodeCharacter{1D539}{\ensuremath{\mathbb{B}}}
\DeclareUnicodeCharacter{1D53B}{\ensuremath{\mathbb{D}}}
\DeclareUnicodeCharacter{1D53C}{\ensuremath{\mathbb{E}}}
\DeclareUnicodeCharacter{1D53D}{\ensuremath{\mathbb{F}}}
\DeclareUnicodeCharacter{1D53E}{\ensuremath{\mathbb{G}}}
\DeclareUnicodeCharacter{1D540}{\ensuremath{\mathbb{I}}}
\DeclareUnicodeCharacter{1D541}{\ensuremath{\mathbb{J}}}
\DeclareUnicodeCharacter{1D542}{\ensuremath{\mathbb{K}}}
\DeclareUnicodeCharacter{1D543}{\ensuremath{\mathbb{L}}}
\DeclareUnicodeCharacter{1D544}{\ensuremath{\mathbb{M}}}
\DeclareUnicodeCharacter{1D546}{\ensuremath{\mathbb{O}}}
\DeclareUnicodeCharacter{1D54A}{\ensuremath{\mathbb{S}}}
\DeclareUnicodeCharacter{1D54B}{\ensuremath{\mathbb{T}}}
\DeclareUnicodeCharacter{1D54C}{\ensuremath{\mathbb{U}}}
\DeclareUnicodeCharacter{1D54D}{\ensuremath{\mathbb{V}}}
\DeclareUnicodeCharacter{1D54E}{\ensuremath{\mathbb{W}}}
\DeclareUnicodeCharacter{1D54F}{\ensuremath{\mathbb{X}}}
\DeclareUnicodeCharacter{1D550}{\ensuremath{\mathbb{Y}}}
\DeclareUnicodeCharacter{1D552}{\ensuremath{\mathbb{a}}}
\DeclareUnicodeCharacter{1D553}{\ensuremath{\mathbb{b}}}
\DeclareUnicodeCharacter{1D554}{\ensuremath{\mathbb{c}}}
\DeclareUnicodeCharacter{1D555}{\ensuremath{\mathbb{d}}}
\DeclareUnicodeCharacter{1D556}{\ensuremath{\mathbb{e}}}
\DeclareUnicodeCharacter{1D557}{\ensuremath{\mathbb{f}}}
\DeclareUnicodeCharacter{1D558}{\ensuremath{\mathbb{g}}}
\DeclareUnicodeCharacter{1D559}{\ensuremath{\mathbb{h}}}
\DeclareUnicodeCharacter{1D55A}{\ensuremath{\mathbb{i}}}
\DeclareUnicodeCharacter{1D55B}{\ensuremath{\mathbb{j}}}
\DeclareUnicodeCharacter{1D55C}{\ensuremath{\mathbb{k}}}
\DeclareUnicodeCharacter{1D55D}{\ensuremath{\mathbb{l}}}
\DeclareUnicodeCharacter{1D55E}{\ensuremath{\mathbb{m}}}
\DeclareUnicodeCharacter{1D55F}{\ensuremath{\mathbb{n}}}
\DeclareUnicodeCharacter{1D56}{\ensuremath{^p}}
\DeclareUnicodeCharacter{1D560}{\ensuremath{\mathbb{o}}}
\DeclareUnicodeCharacter{1D561}{\ensuremath{\mathbb{p}}}
\DeclareUnicodeCharacter{1D562}{\ensuremath{\mathbb{q}}}
\DeclareUnicodeCharacter{1D563}{\ensuremath{\mathbb{r}}}
\DeclareUnicodeCharacter{1D564}{\ensuremath{\mathbb{s}}}
\DeclareUnicodeCharacter{1D565}{\ensuremath{\mathbb{t}}}
\DeclareUnicodeCharacter{1D566}{\ensuremath{\mathbb{u}}}
\DeclareUnicodeCharacter{1D567}{\ensuremath{\mathbb{v}}}
\DeclareUnicodeCharacter{1D568}{\ensuremath{\mathbb{w}}}
\DeclareUnicodeCharacter{1D569}{\ensuremath{\mathbb{x}}}
\DeclareUnicodeCharacter{1D56A}{\ensuremath{\mathbb{y}}}
\DeclareUnicodeCharacter{1D56B}{\ensuremath{\mathbb{z}}}
\DeclareUnicodeCharacter{1D57}{\ensuremath{^t}}
\DeclareUnicodeCharacter{1D58}{\ensuremath{^u}}
\DeclareUnicodeCharacter{1D5B}{\ensuremath{^v}}
\DeclareUnicodeCharacter{1D62}{\ensuremath{_i}}
\DeclareUnicodeCharacter{1D63}{\ensuremath{_r}}
\DeclareUnicodeCharacter{1D64}{\ensuremath{_u}}
\DeclareUnicodeCharacter{1D65}{\ensuremath{_v}}
\DeclareUnicodeCharacter{1D7D8}{\ensuremath{\mathbb{0}}}
\DeclareUnicodeCharacter{1D7D9}{\ensuremath{\mathbb{1}}}
\DeclareUnicodeCharacter{1D7DA}{\ensuremath{\mathbb{2}}}
\DeclareUnicodeCharacter{1D7DB}{\ensuremath{\mathbb{3}}}
\DeclareUnicodeCharacter{1D7DC}{\ensuremath{\mathbb{4}}}
\DeclareUnicodeCharacter{1D7DD}{\ensuremath{\mathbb{5}}}
\DeclareUnicodeCharacter{1D7DE}{\ensuremath{\mathbb{6}}}
\DeclareUnicodeCharacter{1D7DF}{\ensuremath{\mathbb{7}}}
\DeclareUnicodeCharacter{1D7E0}{\ensuremath{\mathbb{8}}}
\DeclareUnicodeCharacter{1D7E1}{\ensuremath{\mathbb{9}}}
\DeclareUnicodeCharacter{1D9C}{\ensuremath{^c}}
\DeclareUnicodeCharacter{1DA0}{\ensuremath{^f}}
\DeclareUnicodeCharacter{1DBB}{\ensuremath{^z}}
\DeclareUnicodeCharacter{1F329}{\ensuremath{\lightning}}
\DeclareUnicodeCharacter{2013}{--}
\DeclareUnicodeCharacter{2014}{---}
\DeclareUnicodeCharacter{2016}{\textbardbl}
\DeclareUnicodeCharacter{2018}{\textquoteleft}
\DeclareUnicodeCharacter{2019}{\textquoteright}
\DeclareUnicodeCharacter{201A}{\quotesinglbase}
\DeclareUnicodeCharacter{201C}{\textquotedblleft}
\DeclareUnicodeCharacter{201D}{\textquotedblright}
\DeclareUnicodeCharacter{201E}{\quotedblbase}
\DeclareUnicodeCharacter{2020}{\ensuremath{\dagger}}
\DeclareUnicodeCharacter{2021}{\ddag}
\DeclareUnicodeCharacter{2022}{\ensuremath{\bullet}}
\DeclareUnicodeCharacter{2023}{\ensuremath{\RHD}}
\DeclareUnicodeCharacter{2026}{\ensuremath{\ldots}}
\DeclareUnicodeCharacter{202F}{\,}
\DeclareUnicodeCharacter{2030}{\textperthousand}
\DeclareUnicodeCharacter{2031}{\textpertenthousand}
\DeclareUnicodeCharacter{2032}{\ensuremath{\prime}}
\DeclareUnicodeCharacter{2033}{\ensuremath{''}}
\DeclareUnicodeCharacter{2034}{\ensuremath{'''}}
\DeclareUnicodeCharacter{2035}{\ensuremath{\backprime}}
\DeclareUnicodeCharacter{2038}{\ifmmode\widehat{}\else\textasciicircum\fi}
\DeclareUnicodeCharacter{2039}{\guilsinglleft}
\DeclareUnicodeCharacter{203A}{\guilsinglright}
\DeclareUnicodeCharacter{203B}{\textreferencemark}
\DeclareUnicodeCharacter{203C}{{!\kern -.5ex!}}
\DeclareUnicodeCharacter{203D}{\textinterrobang}
\DeclareUnicodeCharacter{203E}{\ensuremath{\overline{0}}}
\DeclareUnicodeCharacter{2042}{\LOCALunknownchar}
\DeclareUnicodeCharacter{2045}{\textlquill}
\DeclareUnicodeCharacter{2046}{\textrquill}
\DeclareUnicodeCharacter{2047}{{?\kern -.5ex?}}
\DeclareUnicodeCharacter{2048}{{?\kern -.5ex!}}
\DeclareUnicodeCharacter{2049}{{!\kern -.5ex?}}
\DeclareUnicodeCharacter{2052}{\textdiscount}
\DeclareUnicodeCharacter{2062}{{}}
\DeclareUnicodeCharacter{2070}{\ensuremath{^0}}
\DeclareUnicodeCharacter{2071}{\ensuremath{^i}}
\DeclareUnicodeCharacter{2074}{\ensuremath{^4}}
\DeclareUnicodeCharacter{2075}{\ensuremath{^5}}
\DeclareUnicodeCharacter{2076}{\ensuremath{^6}}
\DeclareUnicodeCharacter{2077}{\ensuremath{^7}}
\DeclareUnicodeCharacter{2078}{\ensuremath{^8}}
\DeclareUnicodeCharacter{2079}{\ensuremath{^9}}
\DeclareUnicodeCharacter{207A}{\ensuremath{^+}}
\DeclareUnicodeCharacter{207B}{\ensuremath{^-}}
\DeclareUnicodeCharacter{207C}{\ensuremath{^=}}
\DeclareUnicodeCharacter{207D}{\ensuremath{^(}}
\DeclareUnicodeCharacter{207E}{\ensuremath{^)}}
\DeclareUnicodeCharacter{207F}{\ensuremath{^n}}
\DeclareUnicodeCharacter{2080}{\ensuremath{_0}}
\DeclareUnicodeCharacter{2081}{\ensuremath{_1}}
\DeclareUnicodeCharacter{2082}{\ensuremath{_2}}
\DeclareUnicodeCharacter{2083}{\ensuremath{_3}}
\DeclareUnicodeCharacter{2084}{\ensuremath{_4}}
\DeclareUnicodeCharacter{2085}{\ensuremath{_5}}
\DeclareUnicodeCharacter{2086}{\ensuremath{_6}}
\DeclareUnicodeCharacter{2087}{\ensuremath{_7}}
\DeclareUnicodeCharacter{2088}{\ensuremath{_8}}
\DeclareUnicodeCharacter{2089}{\ensuremath{_9}}
\DeclareUnicodeCharacter{208A}{\ensuremath{_+}}
\DeclareUnicodeCharacter{208B}{\ensuremath{_-}}
\DeclareUnicodeCharacter{208C}{\ensuremath{_=}}
\DeclareUnicodeCharacter{208D}{\ensuremath{_(}}
\DeclareUnicodeCharacter{208E}{\ensuremath{_)}}
\DeclareUnicodeCharacter{2090}{\ensuremath{_a}}
\DeclareUnicodeCharacter{2091}{\ensuremath{_e}}
\DeclareUnicodeCharacter{2092}{\ensuremath{_o}}
\DeclareUnicodeCharacter{2093}{\ensuremath{_x}}
\DeclareUnicodeCharacter{2095}{\ensuremath{_h}}
\DeclareUnicodeCharacter{2096}{\ensuremath{_k}}
\DeclareUnicodeCharacter{2097}{\ensuremath{_l}}
\DeclareUnicodeCharacter{2098}{\ensuremath{_m}}
\DeclareUnicodeCharacter{2099}{\ensuremath{_n}}
\DeclareUnicodeCharacter{209A}{\ensuremath{_p}}
\DeclareUnicodeCharacter{209B}{\ensuremath{_s}}
\DeclareUnicodeCharacter{209C}{\ensuremath{_t}}
\DeclareUnicodeCharacter{20AC}{\euro}
\DeclareUnicodeCharacter{2102}{\ensuremath{\mathbb{C}}}
\DeclareUnicodeCharacter{2107}{\ensuremath{\mathbb{E}}}
\DeclareUnicodeCharacter{210A}{\ensuremath{\mathcal g}}
\DeclareUnicodeCharacter{210B}{\ensuremath{\mathcal H}}
\DeclareUnicodeCharacter{210C}{\ensuremath{\mathfrak H}}
\DeclareUnicodeCharacter{210D}{\ensuremath{\mathbb{H}}}
\DeclareUnicodeCharacter{210F}{\ensuremath{\hbar}}
\DeclareUnicodeCharacter{2110}{\ensuremath{\mathcal I}}
\DeclareUnicodeCharacter{2111}{\ensuremath{\IM}}
\DeclareUnicodeCharacter{2112}{\ensuremath{\mathcal L}}
\DeclareUnicodeCharacter{2113}{\ensuremath{\ell}}
\DeclareUnicodeCharacter{2115}{\ensuremath{\mathbb{N}}}
\DeclareUnicodeCharacter{2118}{\ensuremath{\wp}}
\DeclareUnicodeCharacter{2119}{\ensuremath{\mathbb{P}}}
\DeclareUnicodeCharacter{211A}{\ensuremath{\mathbb{Q}}}
\DeclareUnicodeCharacter{211C}{\ensuremath{\Re}}
\DeclareUnicodeCharacter{211D}{\ensuremath{\mathbb{R}}}
\DeclareUnicodeCharacter{2122}{\texttrademark}
\DeclareUnicodeCharacter{2124}{\ensuremath{\mathbb{Z}}}
\DeclareUnicodeCharacter{2126}{\ensuremath{\Omega}}
\DeclareUnicodeCharacter{2127}{\ensuremath{\mho}}
\DeclareUnicodeCharacter{2128}{\ensuremath{\mathfrak Z}}
\DeclareUnicodeCharacter{212A}{\ensuremath{\mathrm K}}
\DeclareUnicodeCharacter{212B}{\ensuremath{\mathring{\mathrm A}}}
\DeclareUnicodeCharacter{212C}{\ensuremath{\mathcal B}}
\DeclareUnicodeCharacter{212D}{\ensuremath{\mathfrak C}}
\DeclareUnicodeCharacter{212E}{\textestimated}
\DeclareUnicodeCharacter{212F}{\ensuremath{\mathcal e}}
\DeclareUnicodeCharacter{2130}{\ensuremath{\mathcal E}}
\DeclareUnicodeCharacter{2131}{\ensuremath{\mathcal F}}
\DeclareUnicodeCharacter{2132}{\ensuremath{\Finv}}
\DeclareUnicodeCharacter{2135}{\ensuremath{\aleph}}
\DeclareUnicodeCharacter{2136}{\ensuremath{\beth}}
\DeclareUnicodeCharacter{2137}{\ensuremath{\gimel}}
\DeclareUnicodeCharacter{2138}{\ensuremath{\daleth}}
\DeclareUnicodeCharacter{213C}{\ensuremath{\mathbb{\pi}}}
\DeclareUnicodeCharacter{213D}{\ensuremath{\mathbb{\gamma}}}
\DeclareUnicodeCharacter{213E}{\ensuremath{\mathbb{\Pi}}}
\DeclareUnicodeCharacter{213F}{\ensuremath{\mathbb{\Gamma}}}
\DeclareUnicodeCharacter{2140}{\ensuremath{\mathbb{\Sigma}}}
\DeclareUnicodeCharacter{2141}{\ensuremath{\Game}}
\DeclareUnicodeCharacter{2144}{\ensuremath{Y}}
\DeclareUnicodeCharacter{2146}{\ensuremath{\mathrm{d}}}
\DeclareUnicodeCharacter{2148}{\ensuremath{\imath}}
\DeclareUnicodeCharacter{2149}{\ensuremath{\jmath}}
\DeclareUnicodeCharacter{214B}{\LOCALunknownchar}
\DeclareUnicodeCharacter{2153}{\ensuremath{\frac{1}{3}}}
\DeclareUnicodeCharacter{2154}{\ensuremath{\frac{2}{3}}}
\DeclareUnicodeCharacter{2155}{\ensuremath{\frac{1}{5}}}
\DeclareUnicodeCharacter{2156}{\ensuremath{\frac{2}{5}}}
\DeclareUnicodeCharacter{2157}{\ensuremath{\frac{3}{5}}}
\DeclareUnicodeCharacter{2158}{\ensuremath{\frac{4}{5}}}
\DeclareUnicodeCharacter{2159}{\ensuremath{\frac{1}{6}}}
\DeclareUnicodeCharacter{215A}{\ensuremath{\frac{5}{6}}}
\DeclareUnicodeCharacter{215B}{\ensuremath{\frac{1}{8}}}
\DeclareUnicodeCharacter{215D}{\ensuremath{\frac{5}{8}}}
\DeclareUnicodeCharacter{215E}{\ensuremath{\frac{7}{8}}}
\DeclareUnicodeCharacter{2190}{\ensuremath{\leftarrow}}
\DeclareUnicodeCharacter{2191}{\ensuremath{\uparrow}}
\DeclareUnicodeCharacter{2192}{\ensuremath{\rightarrow}}
\DeclareUnicodeCharacter{2193}{\ensuremath{\downarrow}}
\DeclareUnicodeCharacter{2194}{\ensuremath{\leftrightarrow}}
\DeclareUnicodeCharacter{2195}{\ensuremath{\updownarrow}}
\DeclareUnicodeCharacter{2196}{\ensuremath{\nwarrow}}
\DeclareUnicodeCharacter{2197}{\ensuremath{\nearrow}}
\DeclareUnicodeCharacter{2198}{\ensuremath{\searrow}}
\DeclareUnicodeCharacter{2199}{\ensuremath{\swarrow}}
\DeclareUnicodeCharacter{219A}{\ensuremath{\nleftarrow}}
\DeclareUnicodeCharacter{219B}{\ensuremath{\nrightarrow}}
\DeclareUnicodeCharacter{219E}{\ensuremath{\twoheadleftarrow}}
\DeclareUnicodeCharacter{21A0}{\ensuremath{\twoheadrightarrow}}
\DeclareUnicodeCharacter{21A2}{\ensuremath{\leftarrowtail}}
\DeclareUnicodeCharacter{21A3}{\ensuremath{\rightarrowtail}}
\DeclareUnicodeCharacter{21A4}{\ensuremath{\mapsfrom}}
\DeclareUnicodeCharacter{21A6}{\ensuremath{\mapsto}}
\DeclareUnicodeCharacter{21A9}{\ensuremath{\hookleftarrow}}
\DeclareUnicodeCharacter{21AA}{\ensuremath{\hookrightarrow}}
\DeclareUnicodeCharacter{21AB}{\ensuremath{\looparrowleft}}
\DeclareUnicodeCharacter{21AC}{\ensuremath{\looparrowright}}
\DeclareUnicodeCharacter{21AD}{\ensuremath{\leftrightsquigarrow}}
\DeclareUnicodeCharacter{21AE}{\ensuremath{\nleftrightarrow}}
\DeclareUnicodeCharacter{21AF}{\ensuremath{\lightning}}
\DeclareUnicodeCharacter{21B0}{\ensuremath{\Lsh}}
\DeclareUnicodeCharacter{21B1}{\ensuremath{\Rsh}}
\DeclareUnicodeCharacter{21B6}{\ensuremath{\curvearrowleft}}
\DeclareUnicodeCharacter{21B7}{\ensuremath{\curvearrowright}}
\DeclareUnicodeCharacter{21BA}{\ensuremath{\circlearrowleft}}
\DeclareUnicodeCharacter{21BB}{\ensuremath{\circlearrowright}}
\DeclareUnicodeCharacter{21BC}{\ensuremath{\leftharpoonup}}
\DeclareUnicodeCharacter{21BD}{\ensuremath{\leftharpoondown}}
\DeclareUnicodeCharacter{21BE}{\ensuremath{\upharpoonright}}
\DeclareUnicodeCharacter{21BF}{\ensuremath{\upharpoonleft}}
\DeclareUnicodeCharacter{21C0}{\ensuremath{\rightharpoonup}}
\DeclareUnicodeCharacter{21C1}{\ensuremath{\rightharpoondown}}
\DeclareUnicodeCharacter{21C2}{\ensuremath{\downharpoonright}}
\DeclareUnicodeCharacter{21C3}{\ensuremath{\downharpoonleft}}
\DeclareUnicodeCharacter{21C4}{\ensuremath{\rightleftarrows}}
\DeclareUnicodeCharacter{21C5}{\LOCALunknownchar}
\DeclareUnicodeCharacter{21C6}{\ensuremath{\leftrightarrows}}
\DeclareUnicodeCharacter{21C7}{\ensuremath{\leftleftarrows}}
\DeclareUnicodeCharacter{21C8}{\ensuremath{\upuparrows}}
\DeclareUnicodeCharacter{21C9}{\ensuremath{\rightrightarrows}}
\DeclareUnicodeCharacter{21CA}{\ensuremath{\downdownarrows}}
\DeclareUnicodeCharacter{21CB}{\ensuremath{\leftrightharpoons}}
\DeclareUnicodeCharacter{21CC}{\ensuremath{\rightleftharpoons}}
\DeclareUnicodeCharacter{21CD}{\ensuremath{\nLeftarrow}}
\DeclareUnicodeCharacter{21CE}{\ensuremath{\nLeftrightarrow}}
\DeclareUnicodeCharacter{21CF}{\ensuremath{\nRightarrow}}
\DeclareUnicodeCharacter{21D0}{\ensuremath{\Leftarrow}}
\DeclareUnicodeCharacter{21D1}{\ensuremath{\Uparrow}}
\DeclareUnicodeCharacter{21D2}{\ensuremath{\Rightarrow}}
\DeclareUnicodeCharacter{21D3}{\ensuremath{\Downarrow}}
\DeclareUnicodeCharacter{21D4}{\ensuremath{\Leftrightarrow}}
\DeclareUnicodeCharacter{21D5}{\ensuremath{\Updownarrow}}
\DeclareUnicodeCharacter{21D6}{\ensuremath{\nwarrow}}
\DeclareUnicodeCharacter{21D7}{\ensuremath{\nearrow}}
\DeclareUnicodeCharacter{21D8}{\ensuremath{\searrow}}
\DeclareUnicodeCharacter{21D9}{\ensuremath{\swarrow}}
\DeclareUnicodeCharacter{21DA}{\ensuremath{\Lleftarrow}}
\DeclareUnicodeCharacter{21DB}{\ensuremath{\Rrightarrow}}
\DeclareUnicodeCharacter{21DC}{\LOCALunknownchar}
\DeclareUnicodeCharacter{21DD}{\ensuremath{\rightsquigarrow}}
\DeclareUnicodeCharacter{21E0}{\ensuremath{\dashleftarrow}}
\DeclareUnicodeCharacter{21E2}{\ensuremath{\dashrightarrow}}
\DeclareUnicodeCharacter{21E4}{\ensuremath{\Leftarrow}}
\DeclareUnicodeCharacter{21E5}{\ensuremath{\Rightarrow}}
\DeclareUnicodeCharacter{21F0}{\ensuremath{\mapsto}}
\DeclareUnicodeCharacter{21FD}{\ensuremath{\leftarrow}}
\DeclareUnicodeCharacter{21FE}{\ensuremath{\rightarrow}}
\DeclareUnicodeCharacter{21FF}{\ensuremath{\leftrightarrow}}
\DeclareUnicodeCharacter{2200}{\ensuremath{\forall}}
\DeclareUnicodeCharacter{2201}{\ensuremath{\complement}}
\DeclareUnicodeCharacter{2202}{\ensuremath{\partial}}
\DeclareUnicodeCharacter{2203}{\ensuremath{\exists}}
\DeclareUnicodeCharacter{2204}{\ensuremath{\not\exists}}
\DeclareUnicodeCharacter{2205}{\ensuremath{\varnothing}}
\DeclareUnicodeCharacter{2207}{\ensuremath{\nabla}}
\DeclareUnicodeCharacter{2208}{\ensuremath{\in}}
\DeclareUnicodeCharacter{2209}{\ensuremath{\notin}}
\DeclareUnicodeCharacter{220B}{\ensuremath{\ni}}
\DeclareUnicodeCharacter{220C}{\ensuremath{!\ni}}
\DeclareUnicodeCharacter{220D}{\ensuremath{\bullet}}
\DeclareUnicodeCharacter{220E}{{\tiny \ensuremath{\blacksquare}}}
\DeclareUnicodeCharacter{220F}{\ensuremath{\prod}}
\DeclareUnicodeCharacter{2210}{\ensuremath{\coprod}}
\DeclareUnicodeCharacter{2211}{\ensuremath{\sum}}
\DeclareUnicodeCharacter{2212}{-}
\DeclareUnicodeCharacter{2213}{\ensuremath{\mp}}
\DeclareUnicodeCharacter{2214}{\ensuremath{\dotplus}}
\DeclareUnicodeCharacter{2215}{\ensuremath{/}}
\DeclareUnicodeCharacter{2216}{\ensuremath{\smallsetminus}}
\DeclareUnicodeCharacter{2217}{\ensuremath{\star}}
\DeclareUnicodeCharacter{2218}{\ensuremath{\circ}}
\DeclareUnicodeCharacter{2219}{\ensuremath{\bullet}}
\DeclareUnicodeCharacter{221A}{\ensuremath{\sqrt{}}}
\DeclareUnicodeCharacter{221B}{\ensuremath{\sqrt[3]{}}}
\DeclareUnicodeCharacter{221C}{\ensuremath{\sqrt[4]{}}}
\DeclareUnicodeCharacter{221D}{\ensuremath{\propto}}
\DeclareUnicodeCharacter{221E}{\ensuremath{\infty}}
\DeclareUnicodeCharacter{2220}{\ensuremath{\angle}}
\DeclareUnicodeCharacter{2221}{\ensuremath{\measuredangle}}
\DeclareUnicodeCharacter{2222}{\ensuremath{\sphericalangle}}
\DeclareUnicodeCharacter{2223}{\ensuremath{\mid}}
\DeclareUnicodeCharacter{2224}{\ensuremath{\nmid}}
\DeclareUnicodeCharacter{2225}{\ensuremath{\parallel}}
\DeclareUnicodeCharacter{2226}{\ensuremath{\nparallel}}
\DeclareUnicodeCharacter{2227}{\ensuremath{\wedge}}
\DeclareUnicodeCharacter{2228}{\ensuremath{\vee}}
\DeclareUnicodeCharacter{2229}{\ensuremath{\cap}}
\DeclareUnicodeCharacter{222A}{\ensuremath{\cup}}
\DeclareUnicodeCharacter{222B}{\ensuremath{\int}}
\DeclareUnicodeCharacter{222C}{\ensuremath{\iint}}
\DeclareUnicodeCharacter{222D}{\ensuremath{\iiint}}
\DeclareUnicodeCharacter{222E}{\ensuremath{\oint}}
\DeclareUnicodeCharacter{222F}{\LOCALunknownchar}
\DeclareUnicodeCharacter{2230}{\LOCALunknownchar}
\DeclareUnicodeCharacter{2232}{\LOCALunknownchar}
\DeclareUnicodeCharacter{2233}{\LOCALunknownchar}
\DeclareUnicodeCharacter{2234}{\ensuremath{\therefore}}
\DeclareUnicodeCharacter{2235}{\ensuremath{\because}}
\DeclareUnicodeCharacter{2236}{:}
\DeclareUnicodeCharacter{2237}{\LOCALunknownchar}
\DeclareUnicodeCharacter{2238}{\LOCALunknownchar}
\DeclareUnicodeCharacter{2239}{\ensuremath{\eqcolon}}
\DeclareUnicodeCharacter{223C}{\ensuremath{\sim}}
\DeclareUnicodeCharacter{223D}{\ensuremath{\backsim}}
\DeclareUnicodeCharacter{223F}{\AC}
\DeclareUnicodeCharacter{2240}{\ensuremath{\wr}}
\DeclareUnicodeCharacter{2241}{\ensuremath{\nsim}}
\DeclareUnicodeCharacter{2243}{\ensuremath{\simeq}}
\DeclareUnicodeCharacter{2244}{\ensuremath{\not\simeq}}
\DeclareUnicodeCharacter{2245}{\ensuremath{\cong}}
\DeclareUnicodeCharacter{2247}{\ensuremath{\ncong}}
\DeclareUnicodeCharacter{2248}{\ensuremath{\approx}}
\DeclareUnicodeCharacter{2249}{\ensuremath{\not\approx}}
\DeclareUnicodeCharacter{224A}{\ensuremath{\approxeq}}
\DeclareUnicodeCharacter{224D}{\ensuremath{\asymp}}
\DeclareUnicodeCharacter{224E}{\ensuremath{\Bumpeq}}
\DeclareUnicodeCharacter{224F}{\ensuremath{\bumpeq}}
\DeclareUnicodeCharacter{2250}{\ensuremath{\doteq}}
\DeclareUnicodeCharacter{2251}{\ensuremath{\doteqdot}}
\DeclareUnicodeCharacter{2252}{\ensuremath{\fallingdotseq}}
\DeclareUnicodeCharacter{2253}{\ensuremath{\risingdotseq}}
\DeclareUnicodeCharacter{2254}{\ensuremath{\coloneqq}}
\DeclareUnicodeCharacter{2255}{\ensuremath{\eqqcolon}}
\DeclareUnicodeCharacter{2256}{\ensuremath{\eqcirc}}
\DeclareUnicodeCharacter{2257}{\ensuremath{\circeq}}
\DeclareUnicodeCharacter{2258}{\ensuremath{\stackrel{\frown}{=}}}
\DeclareUnicodeCharacter{2259}{\ensuremath{\stackrel{\wedge}{=}}}
\DeclareUnicodeCharacter{225A}{\ensuremath{\stackrel{\vee}{=}}}
\DeclareUnicodeCharacter{225B}{\ensuremath{\stackrel{\star}{=}}}
\DeclareUnicodeCharacter{225C}{\ensuremath{\triangleq}}
\DeclareUnicodeCharacter{225D}{\ensuremath{\stackrel{\text{\tiny def}}{=}}}
\DeclareUnicodeCharacter{225F}{\ensuremath{\stackrel{\text{\tiny ?}}{=}}}
\DeclareUnicodeCharacter{2260}{\ensuremath{\ne}}
\DeclareUnicodeCharacter{2261}{\ensuremath{\equiv}}
\DeclareUnicodeCharacter{2262}{\ensuremath{\not\equiv}}
\DeclareUnicodeCharacter{2263}{\ensuremath{\stackrel{=}{=}}}
\DeclareUnicodeCharacter{2264}{\ensuremath{\le}}
\DeclareUnicodeCharacter{2265}{\ensuremath{\ge}}
\DeclareUnicodeCharacter{2266}{\ensuremath{\leqq}}
\DeclareUnicodeCharacter{2267}{\ensuremath{\geqq}}
\DeclareUnicodeCharacter{2268}{\ensuremath{\lneqq}}
\DeclareUnicodeCharacter{2269}{\ensuremath{\gneqq}}
\DeclareUnicodeCharacter{226A}{\ensuremath{\ll}}
\DeclareUnicodeCharacter{226B}{\ensuremath{\gg}}
\DeclareUnicodeCharacter{226C}{\ensuremath{\between}}
\DeclareUnicodeCharacter{226D}{\ensuremath{\not\asymp}}
\DeclareUnicodeCharacter{226E}{\ensuremath{\nless}}
\DeclareUnicodeCharacter{226F}{\ensuremath{\ngtr}}
\DeclareUnicodeCharacter{2270}{\ensuremath{\nleq}}
\DeclareUnicodeCharacter{2271}{\ensuremath{\ngeq}}
\DeclareUnicodeCharacter{2272}{\ensuremath{\lesssim}}
\DeclareUnicodeCharacter{2273}{\ensuremath{\gtrsim}}
\DeclareUnicodeCharacter{2274}{\ensuremath{\not\lesssim}}
\DeclareUnicodeCharacter{2275}{\ensuremath{\not\gtrsim}}
\DeclareUnicodeCharacter{2276}{\ensuremath{\lessgtr}}
\DeclareUnicodeCharacter{2277}{\ensuremath{\gtrless}}
\DeclareUnicodeCharacter{2278}{\ensuremath{\not\lessgtr}}
\DeclareUnicodeCharacter{2279}{\ensuremath{\not\gtrless}}
\DeclareUnicodeCharacter{227A}{\ensuremath{\prec}}
\DeclareUnicodeCharacter{227B}{\ensuremath{\succ}}
\DeclareUnicodeCharacter{227C}{\ensuremath{\preccurlyeq}}
\DeclareUnicodeCharacter{227D}{\LOCALunknownchar}
\DeclareUnicodeCharacter{227E}{\ensuremath{\precsim}}
\DeclareUnicodeCharacter{227F}{\ensuremath{\succsim}}
\DeclareUnicodeCharacter{2280}{\ensuremath{\nprec}}
\DeclareUnicodeCharacter{2281}{\ensuremath{\nsucc}}
\DeclareUnicodeCharacter{2282}{\ensuremath{\subset}}
\DeclareUnicodeCharacter{2283}{\ensuremath{\supset}}
\DeclareUnicodeCharacter{2284}{\ensuremath{\not\subset}}
\DeclareUnicodeCharacter{2285}{\ensuremath{\not\supset}}
\DeclareUnicodeCharacter{2286}{\ensuremath{\subseteq}}
\DeclareUnicodeCharacter{2287}{\ensuremath{\supseteq}}
\DeclareUnicodeCharacter{2288}{\ensuremath{\nsubseteq}}
\DeclareUnicodeCharacter{2289}{\ensuremath{\nsupseteq}}
\DeclareUnicodeCharacter{228A}{\ensuremath{\subsetneq}}
\DeclareUnicodeCharacter{228B}{\ensuremath{\supsetneq}}
\DeclareUnicodeCharacter{228E}{\ensuremath{\uplus}}
\DeclareUnicodeCharacter{228F}{\ensuremath{\sqsubset}}
\DeclareUnicodeCharacter{2290}{\ensuremath{\sqsupset}}
\DeclareUnicodeCharacter{2291}{\ensuremath{\sqsubseteq}}
\DeclareUnicodeCharacter{2292}{\ensuremath{\ensuremath{\to}}}
\DeclareUnicodeCharacter{2293}{\ensuremath{\sqcap}}
\DeclareUnicodeCharacter{2294}{\ensuremath{\sqcup}}
\DeclareUnicodeCharacter{2295}{\ensuremath{\oplus}}
\DeclareUnicodeCharacter{2296}{\ensuremath{\ominus}}
\DeclareUnicodeCharacter{2297}{\ensuremath{\otimes}}
\DeclareUnicodeCharacter{2298}{\ensuremath{\oslash}}
\DeclareUnicodeCharacter{2299}{\ensuremath{\odot}}
\DeclareUnicodeCharacter{229A}{\ensuremath{\circledcirc}}
\DeclareUnicodeCharacter{229B}{\ensuremath{\circledast}}
\DeclareUnicodeCharacter{229D}{\ensuremath{\circleddash}}
\DeclareUnicodeCharacter{229E}{\ensuremath{\boxplus}}
\DeclareUnicodeCharacter{229F}{\ensuremath{\boxminus}}
\DeclareUnicodeCharacter{22A0}{\ensuremath{\boxtimes}}
\DeclareUnicodeCharacter{22A1}{\ensuremath{\boxdot}}
\DeclareUnicodeCharacter{22A2}{\ensuremath{\vdash}}
\DeclareUnicodeCharacter{22A3}{\ensuremath{\dashv}}
\DeclareUnicodeCharacter{22A4}{\ensuremath{\top}}
\DeclareUnicodeCharacter{22A5}{\ensuremath{\bot}}
\DeclareUnicodeCharacter{22A6}{\ensuremath{\vdash}}
\DeclareUnicodeCharacter{22A7}{\ensuremath{\models}}
\DeclareUnicodeCharacter{22A9}{\ensuremath{\Vdash}}
\DeclareUnicodeCharacter{22AA}{\ensuremath{\Vvdash}}
\DeclareUnicodeCharacter{22AB}{\LOCALunknownchar}
\DeclareUnicodeCharacter{22AC}{\ensuremath{\not\vdash}}
\DeclareUnicodeCharacter{22AD}{\ensuremath{\not\vDash}}
\DeclareUnicodeCharacter{22AE}{\ensuremath{\not\Vdash}}
\DeclareUnicodeCharacter{22AF}{\LOCALunknownchar}
\DeclareUnicodeCharacter{22B2}{\ensuremath{\triangleleft}}
\DeclareUnicodeCharacter{22B3}{\ensuremath{\triangleright}}
\DeclareUnicodeCharacter{22B4}{\ensuremath{\unlhd}}
\DeclareUnicodeCharacter{22B5}{\ensuremath{\unrhd}}
\DeclareUnicodeCharacter{22B8}{\ensuremath{\multimap}}
\DeclareUnicodeCharacter{22BA}{\ensuremath{\intercal}}
\DeclareUnicodeCharacter{22BB}{\ensuremath{\veebar}}
\DeclareUnicodeCharacter{22BC}{\ensuremath{\barwedge}}
\DeclareUnicodeCharacter{22C0}{\ensuremath{\bigwedge}}
\DeclareUnicodeCharacter{22C1}{\ensuremath{\bigvee}}
\DeclareUnicodeCharacter{22C2}{\ensuremath{\bigcap}}
\DeclareUnicodeCharacter{22C3}{\ensuremath{\bigcup}}
\DeclareUnicodeCharacter{22C4}{\ensuremath{\diamond}}
\DeclareUnicodeCharacter{22C5}{\ensuremath{\cdot}}
\DeclareUnicodeCharacter{22C6}{\ensuremath{\star}}
\DeclareUnicodeCharacter{22C7}{\ensuremath{\divideontimes}}
\DeclareUnicodeCharacter{22C8}{\ensuremath{\bowtie}}
\DeclareUnicodeCharacter{22C9}{\ensuremath{\ltimes}}
\DeclareUnicodeCharacter{22CA}{\ensuremath{\rtimes}}
\DeclareUnicodeCharacter{22CB}{\ensuremath{\leftthreetimes}}
\DeclareUnicodeCharacter{22CC}{\ensuremath{\rightthreetimes}}
\DeclareUnicodeCharacter{22CD}{\ensuremath{\backsimeq}}
\DeclareUnicodeCharacter{22CE}{\ensuremath{\curlyvee}}
\DeclareUnicodeCharacter{22CF}{\ensuremath{\curlywedge}}
\DeclareUnicodeCharacter{22D0}{\ensuremath{\Subset}}
\DeclareUnicodeCharacter{22D1}{\ensuremath{\Supset}}
\DeclareUnicodeCharacter{22D2}{\ensuremath{\Cap}}
\DeclareUnicodeCharacter{22D3}{\ensuremath{\Cup}}
\DeclareUnicodeCharacter{22D4}{\ensuremath{\pitchfork}}
\DeclareUnicodeCharacter{22D6}{\ensuremath{\lessdot}}
\DeclareUnicodeCharacter{22D7}{\ensuremath{\gtrdot}}
\DeclareUnicodeCharacter{22D8}{\ensuremath{\lll}}
\DeclareUnicodeCharacter{22D9}{\ensuremath{\ggg}}
\DeclareUnicodeCharacter{22DA}{\ensuremath{\lesseqgtr}}
\DeclareUnicodeCharacter{22DB}{\ensuremath{\gtreqless}}
\DeclareUnicodeCharacter{22DE}{\ensuremath{\curlyeqprec}}
\DeclareUnicodeCharacter{22DF}{\ensuremath{\curlyeqsucc}}
\DeclareUnicodeCharacter{22E0}{\ensuremath{\not\preceq}}
\DeclareUnicodeCharacter{22E1}{\ensuremath{\not\succeq}}
\DeclareUnicodeCharacter{22E2}{\ensuremath{\not\sqsubseteq}}
\DeclareUnicodeCharacter{22E3}{\ensuremath{\not\sqsupseteq}}
\DeclareUnicodeCharacter{22E4}{\LOCALunknownchar}
\DeclareUnicodeCharacter{22E5}{\LOCALunknownchar}
\DeclareUnicodeCharacter{22E6}{\ensuremath{\lnsim}}
\DeclareUnicodeCharacter{22E7}{\ensuremath{\gnsim}}
\DeclareUnicodeCharacter{22E8}{\ensuremath{\precnsim}}
\DeclareUnicodeCharacter{22E9}{\ensuremath{\succnsim}}
\DeclareUnicodeCharacter{22EA}{\ensuremath{\not\triangleleft}}
\DeclareUnicodeCharacter{22EB}{\ensuremath{\not\triangleright}}
\DeclareUnicodeCharacter{22EC}{\ensuremath{\not\trianglelefteq}}
\DeclareUnicodeCharacter{22ED}{\ensuremath{\not\trianglerighteq}}
\DeclareUnicodeCharacter{22EE}{\ensuremath{\vdots}}
\DeclareUnicodeCharacter{22EF}{\ensuremath{\cdots}}
\DeclareUnicodeCharacter{22F0}{\LOCALunknownchar}
\DeclareUnicodeCharacter{22F1}{\ensuremath{\ddots}}
\DeclareUnicodeCharacter{2300}{\ensuremath{\diameter}}
\DeclareUnicodeCharacter{2308}{\ensuremath{\lceil}}
\DeclareUnicodeCharacter{2309}{\ensuremath{\rceil}}
\DeclareUnicodeCharacter{230A}{\ensuremath{\lfloor}}
\DeclareUnicodeCharacter{230B}{\ensuremath{\rfloor}}
\DeclareUnicodeCharacter{2322}{\ensuremath{\frown}}
\DeclareUnicodeCharacter{2323}{\ensuremath{\smile}}
\DeclareUnicodeCharacter{2329}{\ensuremath{\langle}}
\DeclareUnicodeCharacter{232A}{\ensuremath{\rangle}}
\DeclareUnicodeCharacter{23CE}{\ensuremath{\hookleftarrow}}
\DeclareUnicodeCharacter{2460}{\ensuremath{\text{1}}}
\DeclareUnicodeCharacter{2461}{\ensuremath{\text{2}}}
\DeclareUnicodeCharacter{2462}{\ensuremath{\text{3}}}
\DeclareUnicodeCharacter{2463}{\ensuremath{\text{4}}}
\DeclareUnicodeCharacter{2464}{\ensuremath{\text{5}}}
\DeclareUnicodeCharacter{2465}{\ensuremath{\text{6}}}
\DeclareUnicodeCharacter{2466}{\ensuremath{\text{7}}}
\DeclareUnicodeCharacter{2467}{\ensuremath{\text{8}}}
\DeclareUnicodeCharacter{2468}{\ensuremath{\text{9}}}
\DeclareUnicodeCharacter{25A1}{\ensuremath{\square}}
\DeclareUnicodeCharacter{25B3}{\ensuremath{\triangle}}
\DeclareUnicodeCharacter{25C5}{\ensuremath{\triangleleft}}
\DeclareUnicodeCharacter{2610}{\fbox{\ensuremath{\phantom{{\checkmark}}}}}
\DeclareUnicodeCharacter{2611}{\fbox{\ensuremath{\checkmark}}}
\DeclareUnicodeCharacter{2615}{\LOCALunknownchar}
\DeclareUnicodeCharacter{2621}{\LOCALunknownchar}
\DeclareUnicodeCharacter{2627}{\LOCALunknownchar}
\DeclareUnicodeCharacter{2639}{\ensuremath{\frownie}}
\DeclareUnicodeCharacter{263A}{\ensuremath{\smiley}}
\DeclareUnicodeCharacter{2660}{\ensuremath{\spadesuit}}
\DeclareUnicodeCharacter{2661}{\ensuremath{\heartsuit}}
\DeclareUnicodeCharacter{2662}{\ensuremath{\diamondsuit}}
\DeclareUnicodeCharacter{2663}{\ensuremath{\clubsuit}}
\DeclareUnicodeCharacter{266D}{\ensuremath{\flat}}
\DeclareUnicodeCharacter{266E}{\ensuremath{\natural}}
\DeclareUnicodeCharacter{266F}{\ensuremath{\sharp}}
\DeclareUnicodeCharacter{26A0}{\ensuremath{\lower .25ex\hbox{\Large $\triangle$\hskip -1.25ex}!\;\,}}
\DeclareUnicodeCharacter{2713}{\ensuremath{\checkmark}}
\DeclareUnicodeCharacter{27C2}{\ensuremath{\perp}}
\DeclareUnicodeCharacter{27E6}{\ensuremath{[}}
\DeclareUnicodeCharacter{27E7}{\ensuremath{]}}
\DeclareUnicodeCharacter{27E8}{\ensuremath{\langle}}
\DeclareUnicodeCharacter{27E9}{\ensuremath{\rangle}}
\DeclareUnicodeCharacter{27EA}{\ensuremath{\llangle}}
\DeclareUnicodeCharacter{27EB}{\ensuremath{\rrangle}}
\DeclareUnicodeCharacter{27F5}{\ensuremath{\longleftarrow}}
\DeclareUnicodeCharacter{27F6}{\ensuremath{\longrightarrow}}
\DeclareUnicodeCharacter{2983}{\LOCALunknownchar}
\DeclareUnicodeCharacter{2984}{\LOCALunknownchar}
\DeclareUnicodeCharacter{2985}{\LOCALunknownchar}
\DeclareUnicodeCharacter{2986}{\LOCALunknownchar}
\DeclareUnicodeCharacter{2987}{\ensuremath{(}}
\DeclareUnicodeCharacter{2988}{\ensuremath{)}}
\DeclareUnicodeCharacter{29F5}{\ensuremath{\setminus}}
\DeclareUnicodeCharacter{2A00}{\ensuremath{\bigodot}}
\DeclareUnicodeCharacter{2A01}{\ensuremath{\bigoplus}}
\DeclareUnicodeCharacter{2A02}{\ensuremath{\bigotimes}}
\DeclareUnicodeCharacter{2A05}{\LOCALunknownchar}
\DeclareUnicodeCharacter{2A06}{\ensuremath{\bigsqcup}}
\DeclareUnicodeCharacter{2A0C}{\ensuremath{\iiiint}}
\DeclareUnicodeCharacter{2A1D}{\ensuremath{\Join}}
\DeclareUnicodeCharacter{2A3F}{\ensuremath{\amalg}}
\DeclareUnicodeCharacter{2A7D}{\ensuremath{\leqslant}}
\DeclareUnicodeCharacter{2A7E}{\ensuremath{\geqslant}}
\DeclareUnicodeCharacter{2AA8}{\LOCALunknownchar}
\DeclareUnicodeCharacter{2AA9}{\LOCALunknownchar}
\DeclareUnicodeCharacter{2AAF}{\ensuremath{\preceq}}
\DeclareUnicodeCharacter{2AB0}{\ensuremath{\succeq}}
\DeclareUnicodeCharacter{2C7C}{\ensuremath{_j}}
\DeclareUnicodeCharacter{2E18}{\textinterrobangdown}
\DeclareUnicodeCharacter{301A}{\ensuremath{[}}
\DeclareUnicodeCharacter{301B}{\ensuremath{]}}
\DeclareUnicodeCharacter{33D1}{\ensuremath{\ln}}
\DeclareUnicodeCharacter{33D2}{\ensuremath{\log}}
\DeclareUnicodeCharacter{D7B0}{\LOCALunknownchar}
\fi

% -----------------------------------------------------------------------------
% CONFIGURACIONES DE PÁGINA
% -----------------------------------------------------------------------------
\newcommand{\templatePagecfg}{
	
	% Se define el punto decimal
	\ifthenelse{\equal{\pointdecimal}{true}}{
		\decimalpoint}{
	}
	
	% Configuración de estilo de página
	\setpagemargincm{\pagemarginleft}{\pagemargintop}{\pagemarginright}{\pagemarginbottom}
	\ifthenelse{\equal{\hfstyle}{style1}}{
		\pagestyle{empty}
	}{
	\ifthenelse{\equal{\hfstyle}{style2}}{
		\pagestyle{fancy} \fancyhf{}
		\fancyfoot[L]{\textcolor{\hftextcolor}{\small \headertitle}}
		\fancyfoot[C]{\textcolor{\hftextcolor}{\small \name}}
		\fancyfoot[R]{\textcolor{\hftextcolor}{\small \thepage}}
		\renewcommand{\headrulewidth}{0pt}
		\renewcommand{\footrulewidth}{0pt}
	}{
	\ifthenelse{\equal{\hfstyle}{style3}}{
		\pagestyle{fancy} \fancyhf{}
		\fancyfoot[L]{\textcolor{\hftextcolor}{\small \thepage}}
		\renewcommand{\headrulewidth}{0pt}
		\renewcommand{\footrulewidth}{0pt}
	}{
	\ifthenelse{\equal{\hfstyle}{style4}}{
		\pagestyle{fancy} \fancyhf{}
		\fancyfoot[C]{\textcolor{\hftextcolor}{\small \thepage}}
		\renewcommand{\headrulewidth}{0pt}
		\renewcommand{\footrulewidth}{0pt}
	}{
	\ifthenelse{\equal{\hfstyle}{style5}}{
		\pagestyle{fancy} \fancyhf{}
		\fancyfoot[R]{\textcolor{\hftextcolor}{\small \thepage}}
		\renewcommand{\headrulewidth}{0pt}
		\renewcommand{\footrulewidth}{0pt}
	}{
	\ifthenelse{\equal{\hfstyle}{style6}}{
		\pagestyle{fancy} \fancyhf{}
		\fancyfoot[L]{\textcolor{\hftextcolor}{\small \thepage}}
		\renewcommand{\headrulewidth}{0pt}
		\renewcommand{\footrulewidth}{0.5pt}
	}{
	\ifthenelse{\equal{\hfstyle}{style7}}{
		\pagestyle{fancy} \fancyhf{}
		\fancyfoot[C]{\textcolor{\hftextcolor}{\small \thepage}}
		\renewcommand{\headrulewidth}{0pt}
		\renewcommand{\footrulewidth}{0.5pt}
	}{
	\ifthenelse{\equal{\hfstyle}{style8}}{
		\pagestyle{fancy} \fancyhf{}
		\fancyfoot[R]{\textcolor{\hftextcolor}{\small \thepage}}
		\renewcommand{\headrulewidth}{0pt}
		\renewcommand{\footrulewidth}{0.5pt}
	}{
	\ifthenelse{\equal{\hfstyle}{style9}}{
		\pagestyle{fancy} \fancyhf{}
		\fancyfoot[L]{\textcolor{\hftextcolor}{\small \name}}
		\fancyfoot[R]{\textcolor{\hftextcolor}{\small \thepage}}
		\renewcommand{\headrulewidth}{0pt}
		\renewcommand{\footrulewidth}{0.5pt}
	}{
	\ifthenelse{\equal{\hfstyle}{style10}}{
		\pagestyle{fancy} \fancyhf{}
		\fancyfoot[L]{\textcolor{\hftextcolor}{\small \name}}
		\fancyfoot[R]{\textcolor{\hftextcolor}{\small \thepage}}
		\renewcommand{\headrulewidth}{0pt}
		\renewcommand{\footrulewidth}{0pt}
	}{
	\ifthenelse{\equal{\hfstyle}{style11}}{
		\pagestyle{fancy} \fancyhf{}
		\fancyfoot[L]{\textcolor{\hftextcolor}{\small \name}}
		\fancyfoot[C]{\textcolor{\hftextcolor}{\small \headertitle}}
		\fancyfoot[R]{\textcolor{\hftextcolor}{\small \thepage}}
		\renewcommand{\headrulewidth}{0pt}
		\renewcommand{\footrulewidth}{0pt}
	}{
		\throwbadconfigondoc{Estilo de header-footer incorrecto}{\hfstyle}{style1..style11}}}}}}}}}}}
	}
	
	% Escribe el encabezado
	\writeheader
	
	% Estilo de título de secciones
	\sectionfont{\color{\titlecolor} \fontsizetitle \styletitle \selectfont}
	
}
