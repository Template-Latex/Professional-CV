% Template:     Informe LaTeX
% Documento:    Núcleo del template
% Versión:      2.1.5 (20/06/2021)
% Codificación: UTF-8
%
% Autor: Pablo Pizarro R.
%        Facultad de Ciencias Físicas y Matemáticas
%        Universidad de Chile
%        pablo@ppizarror.com
%
% Sitio web:    [https://latex.ppizarror.com/professional-cv]
% Licencia MIT: [https://opensource.org/licenses/MIT]

% CONFIGURACIONES
% Template:     Professional-CV
% Documento:    Configuraciones del template
% Versión:      3.3.9 (30/10/2023)
% Codificación: UTF-8
%
% Autor: Pablo Pizarro R.
%        pablo@ppizarror.com
%
% Manual template: [https://latex.ppizarror.com/professional-cv]
% Licencia MIT:    [https://opensource.org/licenses/MIT]

% CONFIGURACIONES GENERALES
\def\documentfontsize {11}        % Tamaño de la fuente del documento [pt]
\def\documentinterline {1.1}      % Interlineado del documento defecto [factor]
\def\fontdocument {default}       % Tipografía base, ver soportadas en manual
\def\fonttypewriter {cmodern}     % Tipografía de \texttt, ver manual
\def\fonturl {same}               % Tipo de fuente url {tt,sf,rm,same}
\def\pointdecimal {false}         % Decimales con punto en vez de coma

% CONFIGURACIÓN DEL ESTILO DEL TEMPLATE
\def\dateseparator {--}           % Carácter que separa las fechas
\def\headeritemsmargin {0.3}      % Margen vertical de los ítems del header [cm]
\def\headeritemslinespace {0.1}   % Distancia entre líneas del header [cm]
\def\headerseparator {\sbullet}   % Separador entre elementos del encabezado
\def\headertitlecentered {false}  % Header centrado
\def\headertitlemargin {0.25}     % Margen de \headertitle [cm]
\def\hfstyle {style5}             % Estilos de encabezado y pie de pag (11 disp.)
\def\institutioniconleft {false}  % Íconos a la izquierda o derecha
\def\institutioniconsize {normal} % Tamaño ícono {small,normal,large,huge}
\def\institutiontitlebold {false} % Título institución en negrita
\def\personaltabledatapic {true}  % Muestra/Oculta la foto en tabla antecedentes
\def\showuserphotoborder {true}   % Muestra un recuadro en la foto de perfil
\def\userphotosize {3.5}          % Dimensiones de la foto del usuario [cm]
\def\usersignaturesize {5.0}      % Ancho de la foto de la firma [cm]
\def\writebirthdayheader {true}   % Escribe año nacimiento encabezado
\def\writelastchangeheader{false} % Escribe fecha último cambio en el encabezado
\def\writetitleheader {false}     % Escribe el \headertitle en el encabezado

% CONFIGURACIÓN DE LOS COLORES DEL DOCUMENTO
\def\colorpage {white}            % Color de la página
\def\cvblocklinecolor {gray!30}   % Color de las líneas divisor. de los bloques
\def\datecolor {gray!90}          % Color de fechas en entradas de instituciones
\def\headerseparatorcolor {black} % Color del separador del header
\def\headertextcolor {black}      % Color del texto en el header
\def\headerurlcolor {dark-blue}   % Color de los enlaces en el header
\def\hftextcolor {lgray}          % Color del texto en header-footer
\def\highlightcolor {yellow}      % Color del subrayado con \hl
\def\instentrytitlecolor {black}  % Color del título en entrada de instituciones
\def\instnamecolor {dark-blue}    % Color del nombre de las instituciones (+url)
\def\instregioncolor {black}      % Color de la región en las instituciones
\def\lastchangeheadercolor {gray} % Color de fecha último cambio en encabezado
\def\maintextcolor {black}        % Color principal del texto
\def\numcitecolor {black}         % Color del número de las referencias o citas
\def\otherentrytitlecolor {black} % Color título en entradas <otherentry>
\def\personaltblentcolor {black}  % Color de los títulos tabla de antecedentes
\def\showborderonlinks {false}    % Encierra los links por un recuadro de color
\def\tablelinecolor {black}       % Color de las líneas de las tablas
\def\titlecolor {black}           % Color de los títulos
\def\urlcolor {blue}              % Color enlaces web en el cuerpo del documento
\def\userphotobordercolor {black} % Color del borde de la foto del usuario

% ESTILO DE LOS TÍTULOS
\def\fontsizemaintitle {\huge}    % Tamaño título principal
\def\fontsizetitle {\Large}       % Tamaño título secciones
\def\stylemaintitle {\bfseries}   % Estilo título principal
\def\styletitle {\bfseries}       % Estilo título secciones

% MÁRGENES DE PÁGINA
\def\pagemarginbottom {1.9}       % Margen inferior página [cm]
\def\pagemarginleft {1.9}         % Margen izquierdo página [cm]
\def\pagemarginright {1.9}        % Margen derecho página [cm]
\def\pagemargintop {1.5}          % Margen superior página [cm]

% OPCIONES DEL PDF COMPILADO
\def\cfgbookmarksopenlevel {1}    % Nivel de los marcadores (1: títulos)
\def\cfgpdfbookmarkopen {false}   % Expande marcadores hasta el nivel config.
\def\cfgpdfcenterwindow {true}    % Centra la ventana del lector al abrir el pdf
\def\cfgpdfdisplaydoctitle {true} % Muestra nombre autor como título del pdf
\def\cfgpdffitwindow {false}      % Ajusta ventana del lector al tamaño del pdf
\def\cfgpdfmenubar {true}         % Muestra el menú del lector al abrir el pdf
\def\cfgpdfpagemode {OneColumn}   % Modo de página pdf (OneColumn,SinglePage)
\def\cfgpdfpageview {FitH}        % Fit,FitH,FitV,FitR,FitB,FitBH,FitBV
\def\cfgpdftoolbar {true}         % Muestra la barra de herramientas en lector
\def\cfgshowbookmarkmenu {false}  % Muestra menú de marcadores al abrir el pdf
\def\pdfcompileversion {7}        % Versión mínima del pdf a compilar (1.x)

% NOMBRES DE LOS OBJETOS
\def\headertitle {Curriculum Vitae}   % Título principal en el header
\def\nomlastchange {Último cambio el} % Nombre apartado último cambio
\def\nompersonaldata {Antecedentes Personales} % Título antecedentes personales
\def\nomsummary {Descripción}         % Título sección descripción
\def\present {presente}               % Fecha presente en <institutionitem>


% IMPORTACIÓN DE FUNCIONES Y LIBRERÍAS
% PARCHES DE LIBRERÍAS
\let\counterwithout\relax
\let\counterwithin\relax
\let\underbar\relax
\let\underline\relax

% Si se desactiva el idioma
\def\unaccentedoperators{}
\def\decimalpoint{}
\def\bibname{}

% Parche de sectsty.sty
\makeatletter
\def\underline#1{\relax\ifmmode\@@underline{#1}\else $\@@underline{\hbox{#1}}\m@th$\relax\fi}\def\underbar#1{\underline{\sbox\tw@{#1}\dp\tw@\z@\box\tw@}}
\makeatother

% Tamaño de la fuente del documento
\usepackage{scrextend}
\usepackage{anyfontsize}
\changefontsizes{\documentfontsize pt}

% LIBRERÍAS IMPORTANTES
\usepackage{ifthen} 		% Manejo de condicionales

% Idioma español, borrar esta línea si se desea usar otro idioma
\ifthenelse{\equal{\usespanishbabel}{true}}{
	\usepackage[spanish,es-nosectiondot,es-lcroman,es-noquoting]{babel}}{
}
\usepackage[T1]{fontenc}	% Caracteres acentuados
\usepackage[utf8]{inputenc} % Codificación UTF-8

% LIBRERÍAS INDEPENDIENTES
\usepackage{array}         % Nuevas características a las tablas
\usepackage{booktabs}      % Permite manejar elementos visuales en tablas
\usepackage{color}         % Colores
\usepackage{colortbl}      % Administración de color en tablas
\usepackage{csquotes}      % Citas y comillas
\usepackage{enumitem}      % Permite enumerar ítems
\usepackage{geometry}      % Dimensiones y geometría del documento
\usepackage{graphicx}      % Propiedades extra para los gráficos
\usepackage{lipsum}        % Permite crear textos dummy
\usepackage{mdwlist}       % Listas con encabezado
\usepackage{multicol}      % Múltiples columnas
\usepackage{relsize}       % Escalado avanzado
\usepackage{sectsty}       % Cambia el estilo de los títulos
\usepackage{selinput}      % Compatibilidad con acentos
\usepackage{setspace}      % Cambia el espacio entre líneas
\usepackage{soul}          % Permite subrayar texto
\usepackage{textcomp}      % Simbología común
\usepackage{url}           % Permite añadir enlaces
\usepackage{wasysym}       % Contiene caracteres misceláneos
\usepackage{wrapfig}       % Permite comprimir imágenes
\usepackage{xcolor}        % Administración de color avanzado
\usepackage{xspace}        % Adminsitra espacios en párrafos y líneas

% LIBRERÍAS CON PARÁMETROS
\usepackage[pdfencoding=auto,psdextra]{hyperref} % Enlaces, referencias
\usepackage[final]{pdfpages} % Permite administrar páginas en pdf

% LIBRERÍAS DEPENDIENTES
\usepackage{bookmark}      % Administración de marcadores en pdf
\usepackage{fancyhdr}      % Encabezados y pie de páginas
\usepackage{float}         % Administrador de posiciones de objetos
\usepackage{hyperxmp}      % Etiquetas opcionales para el pdf compilado
\usepackage{multirow}      % Agrega nuevas opciones a las tablas
% Lanza un mensaje de error
% 	#1	Función del error
%	#2	Mensaje
\newcommand{\throwerror}[2]{
	\errmessage{LaTeX Error: \noexpand#1 #2 (linea \the\inputlineno)}
	\stop
}

% Lanza un mensaje de advertencia
%	#1	Mensaje
\newcommand{\throwwarning}[1]{
	\errmessage{LaTeX Warning: #1 (linea \the\inputlineno)}
}

% Lanza un mensaje de error indicando mala configuración
%	#1	Mensaje de error
% 	#2	Configuración usada
%	#3	Valores esperados
\newcommand{\throwbadconfig}[3]{
	\errmessage{LaTeX Warning: #1 \noexpand #2=#2. Valores esperados: #3}
	\stop
}

% Lanza un mensaje de error indicando mala configuración dentro de begin{document}
%	#1	Mensaje de error
% 	#2	Configuración usada
%	#3	Valores esperados
\newcommand{\throwbadconfigondoc}[3]{
	\errmessage{#1 \noexpand #2=#2. Valores esperados: #3}
	\stop
}

% Comprueba si una variable está definida
%	#1	Variable
\newcommand{\checkvardefined}[1]{
	\ifthenelse{\isundefined{#1}}{
		\errmessage{LaTeX Warning: Variable \noexpand#1 no definida}
		\stop
	}{}
}

% Comprueba si una variable está definida
%	#1	Variable
%	#2	Mensaje
\newcommand{\checkextravarexist}[2]{
	\ifthenelse{\isundefined{#1}}{
		\errmessage{LaTeX Warning: Variable \noexpand#1 no definida}
		\ifx\hfuzz#2\hfuzz
			\errmessage{LaTeX Warning: Defina la variable en el bloque de INFORMACION DEL USUARIO al comienzo del archivo principal del Template}
		\else
			\errmessage{LaTeX Warning: #2}
		\fi
	}{}
}

% Lanza un mensaje de error si una variable no ha sido definida
% 	#1	Función del error
%	#2	Variable
%	#3	Mensaje
\newcommand{\emptyvarerr}[3]{
	\ifx\hfuzz#2\hfuzz
		\errmessage{LaTeX Warning: \noexpand#1 #3 (linea \the\inputlineno)}
	\fi
}

% Cambia márgenes de las páginas [cm]
% 	#1	Margen izquierdo
%	#2	Margen superior
%	#3	Margen derecho
%	#4	Margen inferior
\newcommand{\setpagemargincm}[4]{
	\newgeometry{left=#1cm, top=#2cm, right=#3cm, bottom=#4cm}
}

% Cambia los márgenes del documento
%	#1	Margen izquierdo
%	#2	Margen derecho
\newcommand{\changemargin}[2]{
	\emptyvarerr{\changemargin}{#1}{Margen izquierdo no definido}
	\emptyvarerr{\changemargin}{#2}{Margen derecho no definido}
	\list{}{\rightmargin#2\leftmargin#1}\item[]
}
\let\endchangemargin=\endlist

% Título principal con subtítulo y último cambio
%	#1	Nombre autor
%	#2	Edad corta autor
%	#3	Último cambio
\newcommand{\maintitle}[3]{
	\emptyvarerr{\maintitle}{#1}{Nombre autor no definido}
	\emptyvarerr{\maintitle}{#2}{Edad corta autor no definido}
	%\emptyvarerr{\maintitle}{#3}{Fecha de ultimo cambio no definido}
	\noindent {\fontsizemaintitle \stylemaintitle #1} \quad \emph{#2} \hfill \quad {\scriptsize #3} \\ \vspace{-1em}
}

% Escribe el encabezado del CV
\newcommand{\writeheader}{
	% Escribe título
	\ifthenelse{\equal{\writetitleheader}{true}}{
		\begin{center}
			\fontsize{16}{12} \selectfont \vspace{0.3cm} \textcolor{\titlecolor}{\headertitle}
		\end{center}
		\vspace{0.1cm}
	}{}
	\ifthenelse{\equal{\writelastchangeheader}{true}}{
		\maintitle{\textcolor{\titlecolor}{\nombreautor}}{\textcolor{\headertextcolor}{\mesnacimientoautor \dianacimientoautor, \anonacimientoautor}}{\textcolor{\lastchangeheadercolor}{\nomlastchange\ \today}}
	}{
		\maintitle{\textcolor{\titlecolor}{\nombreautor}}{\textcolor{\headertextcolor}{\mesnacimientoautor \dianacimientoautor, \anonacimientoautor}}{}
	}
	
	% Escribe datos autor
	{
		\noindent
		\ifthenelse{\equal{\correoautor}{}}{}{%
			\href{mailto:\correoautor}{\correoautor}%
		}
		\textcolor{\headertextcolor}{%
		\ifthenelse{\equal{\telefonoautor}{}}{}{%
			\headerseparator\telefonoautor}%
		\ifthenelse{\equal{\ciautor}{\xspace}}{}{%
			\headerseparator\nomciheader: \ciautor }%
		\ifthenelse{\equal{\linkredsocial}{}}{}{%
			\headerseparator\href{\linkredsocial}{\linkredsocial}} \\
		\ifthenelse{\equal{\direccionautor}{\xspace}}{%
			\ifthenelse{\equal{\regionautor}{\xspace}}{}{\regionautor\headerseparator}
			\ifthenelse{\equal{\paisautor}{\xspace}}{}{\paisautor}
		}{%
			\direccionautor%
			\ifthenelse{\equal{\regionautor}{\xspace}}{}{\headerseparator\regionautor}
			\ifthenelse{\equal{\paisautor}{\xspace}}{}{\headerseparator\paisautor}
		}%
		}
	}
}

% Título de las secciones principales
%	#1	Título
\newcommand*{\roottitle}[1]{
	\emptyvarerr{\roottitle}{#1}{Titulo no definido}
	\section*{#1}\vspace{-0.3em}\nopagebreak[4]
	\addcontentsline{toc}{section}{#1}
}

% Crea un acrónimo de forma fácil
%	#1	Acrónimo
\newcommand*{\acr}[1]{
	\textscale{.85}{#1}
}

% Crea un campo de texto
% 	#1	Texto
\newcommand{\bodytext}[1]{
	\nopagebreak[4]
	\begin{indentsection}
		\item[] #1
	\end{indentsection}
	\pagebreak[2]
}

% Crea una entrada en la tabla de datos personales
%	#1	Nombre entrada
%	#2	Datos entrada
\newcommand{\personaltableentry}[2]{
	\emptyvarerr{\personaltableentry}{#1}{Nombre entrada no definido}
	\emptyvarerr{\personaltableentry}{#2}{Datos entrada no definidos}
	\textcolor{\personaltblentcolor}{\textbf{#1:}} & #2 \\
}

% Inserta un objeto en un elemento institución
%	#1	Cargo
%	#2	Indica si escribe fechas o no
%	#3	Fecha inicial
%	#4	Fecha final
%	#5	Ancho título
%	#6	Ancho fecha
%	#7	Descripción
\newcommand{\newinstitutionentry}[7]{
	\emptyvarerr{\newinstitutionentry}{#1}{Cargo o posición no definido}
	\emptyvarerr{\newinstitutionentry}{#5}{Dimensiones de titulo no definido}
	\emptyvarerr{\newinstitutionentry}{#6}{Dimensiones de fecha no definido}
	\nopagebreak[4]
	\begin{indentsection}
		\item []
		\noindent
		\begin{minipage}{#5}
			{\textcolor{\instentrytitlecolor}{\textbf{#1}} ~ \\ \vspace{-1em}}
		\end{minipage}
		\begin{minipage}{#6}
			\begin{flushright}
				\noindent \textcolor{\datecolor}{\emph{#2#4#3}}
			\end{flushright}
		\end{minipage}
		\break
		\ifx\hfuzz#7\hfuzz
			\vspace*{-0.6em}
		\else
			\begin{minipage}{1.0\linewidth}
				\vspace{0.2cm}
				\bodytext{#7}
				\par
			\end{minipage}
		\fi
	\end{indentsection}
	\nopagebreak[4]
}

% Entrada en institución
%	#1	Cargo
%	#2	Fecha inicial
%	#3	Fecha final
%	#4	Descripción
\newcommand{\institutionentry}[4]{
	\newinstitutionentry{#1}{#2 }{ #3}{\dateseparator}{.68\linewidth}{.31\linewidth}{#4}
}

% Entrada en institución con fecha corta
%	#1	Cargo
%	#2	Fecha inicial
%	#3	Fecha final
%	#4	Descripción
\newcommand{\institutionentryshort}[4]{
	\newinstitutionentry{#1}{#2 }{ #3}{\dateseparator}{.85\linewidth}{.14\linewidth}{#4}
}

% Entrada en institución sin fecha
%	#1	Cargo
%	#2	Descripción
\newcommand{\institutionentrynodate}[2]{
	\newinstitutionentry{#1}{}{}{}{.99\linewidth}{.0\linewidth}{#2}
}

% Otro tipo de entrada en cvblock
%	#1	Título
%	#2	Contenido
\newcommand{\otherentry}[2]{
	\begin{basedescript}
		{\setlength{\leftmargin}{\doubleparindent}}
		\item[\hspace{\newparindent}\textcolor{\otherentrytitlecolor}{\textbf{#1}}] #2
	\end{basedescript}
	\vspace{-0.6cm}
}

% Inserta un elemento en el bloque de firma
%	#1	Elemento en la firma
\newcommand{\signatureentry}[1]{
	\texttt{\MakeUppercase{#1}} \\
}

% Crea un bull
\newcommand{\sbullet}{\ \ \raisebox{0.11em}[-1em][-1em]{\tiny $\bullet$} \ \ }

% Variaciones de vspace
\newcommand{\breakvspace}[1]{\pagebreak[2] \vspace{#1} \pagebreak[2]}
\newcommand{\nobreakvspace}[1]{\nopagebreak[4] \vspace{#1} \nopagebreak[4]}

% Apóstrofe
\newcommand{\apo}{\raisebox{-.18ex}{'}{\hspace{-.1em}}}

% Inserta un texto entre comillas
\newcommand{\quotes}[1]{\enquote*{#1}}

% Inserta un email con un link cliqueable
\newcommand{\insertemail}[1]{\href{mailto:#1}{\texttt{#1}}}

% Inserta un teléfono celular
\newcommand{\insertphone}[1]{\href{tel:#1}{\texttt{#1}}}

% Definición de colores
\definecolor{backcolour}{rgb}{0.95, 0.95, 0.92}
\definecolor{codegray}{rgb}{0.5, 0.5, 0.5}
\definecolor{codegreen}{rgb}{0, 0.6, 0}
\definecolor{codepurple}{rgb}{0.58, 0, 0.82}
\definecolor{dark-blue}{rgb}{0.15, 0.15, 0.40}
\definecolor{dgray}{RGB}{104, 108, 113}
\definecolor{dkgreen}{rgb}{0, 0.6, 0}
\definecolor{gray}{rgb}{0.5, 0.5, 0.5}
\definecolor{lbrown}{RGB}{255, 252, 249}
\definecolor{lgray}{RGB}{180, 180, 180}
\definecolor{lyellow}{rgb}{1.0, 1.0, 0.88}
\definecolor{mauve}{rgb}{0.58, 0, 0.82}
\definecolor{mygray}{rgb}{0.5, 0.5, 0.5}
\definecolor{mygreen}{rgb}{0, 0.6, 0}
\definecolor{mylilas}{RGB}{170, 55, 241}
\definecolor{ocre}{RGB}{243, 102, 25}

% Crea una sección identada
\newenvironment{indentsection}{
	\begin{list}{}{
		\setlength{\leftmargin}{\newparindent}
		\setlength{\parsep}{0pt}
		\setlength{\parskip}{0pt}
		\setlength{\itemsep}{0pt}
		\setlength{\topsep}{0pt}}
	}{
	\end{list}
}

% Crea un bloque de contenido
%	#1	Título
%	#2	Estilo de línea
\newenvironment{cvblock}[2][]{
	\textcolor{\cvblocklinecolor}{
		\ifx\hfuzz#1\hfuzz
			\breakvspace{1em} \hrule \nobreakvspace{0em}
		\else
			\ifthenelse{\equal{#1}{..}}{
				\noindent \makebox[1\linewidth]{\dotfill}
				\vspace{-2em}
			}{
				\ifthenelse{\equal{#1}{none}}{
			}{
				\throwwarning{Estilo de linea invalido}
				\stop
			}}
		\fi
	}
	\ifx\hfuzz#2\hfuzz
		\throwwarning{Titulo del bloque no definido}
		\stop
	\else
		\roottitle{#2}
	\fi
	\vspace{0.5em}
	}{
	\par
}

% Párrafo de descripción
\newenvironment{summary}[1][]{
	\begin{cvblock}[#1]{\nomsummary}
		\begingroup
		\vspace{-1em}
		\begin{multicols}{2}
			\noindent
		}{
		\end{multicols}
		\vspace{-0.1cm}
		\par
		\endgroup
	\end{cvblock}
}

% Párrafo de descripción con foto de perfil
\newenvironment{photosummary}[1][]{
	\begin{cvblock}[#1]{\nomsummary}
		\vspace{0.1em}
		\noindent
		\begin{minipage}[t]{\userphotosize cm}
			\begin{flushleft}
				\begingroup
				\setlength{\fboxsep}{0pt}
				~ \\ \vspace{-1em}
				\ifthenelse{\equal{\showuserphotoborder}{true}}{
					\noindent \textcolor{\userphotobordercolor}{\fbox{\includegraphics[width=0.99\linewidth]{\fotoautor}}}
				}{
					\noindent \includegraphics[width=\userphotosize cm,height=\userphotosize cm]{\fotoautor}
				}
				\vspace{-0.7em}
				\endgroup
			\end{flushleft}
		\end{minipage}
		\hspace*{0.37cm}
		\begingroup
		\def\arraystretch{1.3}
		\begin{minipage}[t]{0.79\linewidth}
		}{
		\end{minipage}
		\endgroup
		\par
	\end{cvblock}
}

% Tabla de datos personales
\newenvironment{personaltabledata}[1][]{
	\begin{cvblock}[#1]{\nompersonaldata}
		\vspace{0.1em}
		\noindent
		\ifthenelse{\equal{\personaltabledatapic}{true}}{
			\begin{minipage}[t]{\userphotosize cm}
				\begin{flushleft}
					\begingroup
					\setlength{\fboxsep}{0pt}
					~ \\ \vspace{-1em}
					\ifthenelse{\equal{\showuserphotoborder}{true}}{
						\noindent \textcolor{\userphotobordercolor}{\fbox{\includegraphics[width=0.99\linewidth]{\fotoautor}}}
					}{
						\noindent \includegraphics[width=\userphotosize cm,height=\userphotosize cm]{\fotoautor}
					}
					\vspace{-0.7em}
					\endgroup
				\end{flushleft}
			\end{minipage}
		}{}
		\hspace*{0cm}
		\begingroup
		\def\arraystretch{1.3}
		\begin{minipage}[t]{0.79\linewidth}
			\vspace*{-0.945cm}
			\begin{table}[H]
				\begin{tabular}{ll}
				}{
				\end{tabular}
			\end{table}
		\end{minipage}
		\endgroup
		\par
	\end{cvblock}
}

% Nueva institución
%	#1	Opcional: url o link institución
%	#2	Nombre institución
%	#3	Ubicación institución
\newenvironment{institution}[3][]{
	\nopagebreak[4]
	\begin{indentsection}
		\item []
		\noindent
		\ifx\hfuzz#3\hfuzz
			\begin{minipage}{0.995\linewidth}
				\ifx\hfuzz#1\hfuzz
				\textscale{1.1}{\textcolor{\linkcolor}{#2}}
				\else
				\textscale{1.1}{\href{#1}{#2}}
				\fi
			\end{minipage}
		\else
			\begin{minipage}{.55\linewidth}
				\ifx\hfuzz#1\hfuzz
				\textscale{1.1}{\textcolor{\linkcolor}{#2}}
				\else
				\textscale{1.1}{\href{#1}{#2}}
				\fi
			\end{minipage}
			\begin{minipage}{.44\linewidth}
				\begin{flushright}
					\noindent \textcolor{\instregioncolor}{\textsc{#3}}
				\end{flushright}
			\end{minipage}
		\fi
		\break
		\begin{minipage}{1.0\linewidth}
			\noindent
			}{
		\end{minipage}
	\end{indentsection}
	\par
	\nopagebreak[4]
}

% Crea la firma del usuario
%	#1	Título
\newenvironment{signature}{
	\vfill
	\begin{flushright}
		\begin{tabular}{c}
			\includegraphics[width=\usersignaturesize cm]{\firmaautor} \\
		}{
		\end{tabular}
	\end{flushright}
}

% Itemize en negrita
%	#1	Parámetros opcionales
\newenvironment{itemizebf}[1][]{
	\begin{itemize}[font=\bfseries,#1]
	}{
	\end{itemize}
}

% Enumerate en negrita
%	#1	Parámetros opcionales
\newenvironment{enumeratebf}[1][]{
	\begin{enumerate}[font=\bfseries,#1]
	}{
	\end{enumerate}
}

% CONFIGURACIÓN INICIAL DEL DOCUMENTO
% Se revisa si las variables no han sido borradas
\checkvardefined{\anonacimientoautor}
\checkvardefined{\ciautor}
\checkvardefined{\correoautor}
\checkvardefined{\dianacimientoautor}
\checkvardefined{\direccionautor}
\checkvardefined{\edadautor}
\checkvardefined{\linkredsocial}
\checkvardefined{\mesnacimientoautor}
\checkvardefined{\nombreautor}
\checkvardefined{\regionautor}
\checkvardefined{\telefonoautor}

% Se añade \xspace a las variables
\makeatletter
\g@addto@macro\anonacimientoautor\xspace
\g@addto@macro\ciautor\xspace
\g@addto@macro\dianacimientoautor\xspace
\g@addto@macro\direccionautor\xspace
\g@addto@macro\edadautor\xspace
\g@addto@macro\mesnacimientoautor\xspace
\g@addto@macro\nombreautor\xspace
\g@addto@macro\paisautor\xspace
\g@addto@macro\regionautor\xspace
\g@addto@macro\telefonoautor\xspace
\makeatother

% Se define metadata del pdf
\ifthenelse{\equal{\cfgshowbookmarkmenu}{true}}{
	\def\cdfpagemodepdf {UseOutlines}
}{
	\def\cdfpagemodepdf {UseNone}
}
\hypersetup{
	keeppdfinfo,
	bookmarksopen={\cfgpdfbookmarkopen},
	bookmarksopenlevel={\cfgbookmarksopenlevel},
	bookmarkstype={toc},
	pdfauthor={\nombreautor},
	pdfcenterwindow={\cfgpdfcenterwindow},
	pdfcopyright={Currículum Vítae, \nombreautor. Email: \correoautor. Teléfono: \telefonoautor},
	pdfcreator={LaTeX},
	pdfdisplaydoctitle={\cfgpdfdisplaydoctitle},
	pdffitwindow={\cfgpdffitwindow},
	pdfinfo={
		Autor.Email={\correoautor},
		Autor.Nombre={\nombreautor},
		Autor.Telefono={\telefonoautor},
		Autor.Ubicacion.Direccion={\direccionautor},
		Autor.Ubicacion.Pais={\paisautor},
		Autor.Ubicacion.Region={\regionautor},
		Template.Autor.Alias={ppizarror},
		Template.Autor.Email={pablo@ppizarror.com},
		Template.Autor.Nombre={Pablo Pizarro R.},
		Template.Autor.Web={https://ppizarror.com/},
		Template.Codificacion={UTF-8},
		Template.Fecha={20/06/2021},
		Template.Licencia.Tipo={MIT},
		Template.Licencia.Web={https://opensource.org/licenses/MIT/},
		Template.Nombre={Professional-CV},
		Template.Tipo={Normal},
		Template.Version.Dev={2.1.5},
		Template.Version.Hash={D18127C0B74831F34146AD9DC682D29A},
		Template.Version.Release={2.1.5},
		Template.Web.Dev={https://github.com/Template-Latex/Professional-CV/},
		Template.Web.Manual={https://latex.ppizarror.com/Professional-CV/}
	},
	pdfkeywords={CV, \nombreautor, Currículum, Vítae},
	pdflang={\documentlang},
	pdfmenubar={\cfgpdfmenubar},
	pdfpagelayout={\cfgpdfpagemode},
	pdfpagemode={\cdfpagemodepdf},
	pdfproducer={Professional-CV v2.1.5 | (Pablo Pizarro R.) ppizarror.com},
	pdfremotestartview={Fit},
	pdfstartpage={1},
	pdfstartview={\cfgpdfpageview},
	pdfsubject={C.V. \nombreautor},
	pdftitle={Currículum Vítae \nombreautor},
	pdftoolbar={\cfgpdftoolbar},
	pdftype={Text}
}

% Establece la carpeta de imágenes por defecto
\graphicspath{{./img}}

% Ajuste del entrelineado
\renewcommand{\baselinestretch}{\documentinterline}

% Configuración de los colores
\def\linkcolor{\urlcolor}
\color{\maintextcolor} % Color principal
\arrayrulecolor{\tablelinecolor} % Color de las líneas de las tablas
\sethlcolor{\highlightcolor} % Color del subrayado por defecto
\ifthenelse{\equal{\showborderonlinks}{true}}{
	\hypersetup{
		% Color de links con borde
		citebordercolor=\numcitecolor,
		linkbordercolor=\linkcolor,
		urlbordercolor=\urlcolor
	}
}{
	\hypersetup{
		% Color de links sin borde
		hidelinks,
		colorlinks=true,
		citecolor=\numcitecolor,
		linkcolor=\linkcolor,
		urlcolor=\urlcolor
	}
}
\ifthenelse{\equal{\colorpage}{white}}{}{
	\pagecolor{\colorpage}
}

% Definición de la fuente
\usepackage{ifxetex}
\ifxetex
	\usepackage{fontspec}
	\setmainfont
	[ExternalLocation,
	Mapping=tex-text,
	Numbers=OldStyle,
	Ligatures={Common,Contextual},
	BoldFont=texgyrepagella-bold.otf,
	ItalicFont=texgyrepagella-italic.otf,
	BoldItalicFont=texgyrepagella-bolditalic.otf]
	{texgyrepagella-regular.otf}
	\usepackage[protrusion]{microtype}
\else
	\usepackage{tgpagella}
	\usepackage[expansion,protrusion]{microtype}
\fi

% Configuración de la identación
\newlength{\newparindent}
\addtolength{\newparindent}{\parindent}
\newlength{\doubleparindent}
\addtolength{\doubleparindent}{\parindent}
\addtolength{\doubleparindent}{\parindent}

% Configuración de hbox y vbox
\hfuzz=200pt \vfuzz=200pt
\hbadness=\maxdimen \vbadness=\maxdimen

% Se activa el word-wrap para textos con \texttt{}
\ttfamily \hyphenchar\the\font=`\-

% Se define el tipo de texto de los url
\urlstyle{tt}

% Se define la versión menor a compilar
\pdfminorversion=\pdfcompileversion

% CONFIGURACIONES DE PÁGINA
\newcommand{\templatePagecfg}{
	
% Se define el punto decimal
\ifthenelse{\equal{\pointdecimal}{true}}{
	\decimalpoint}{
}

% Configuración de estilo de página
\setpagemargincm{\pagemarginleft}{\pagemargintop}{\pagemarginright}{\pagemarginbottom}
\ifthenelse{\equal{\hfstyle}{style1}}{
	\pagestyle{empty}
}{
\ifthenelse{\equal{\hfstyle}{style2}}{
	\pagestyle{fancy} \fancyhf{}
	\fancyhead[L]{}
	\fancyhead[C]{}
	\fancyhead[R]{}
	\fancyfoot[L]{\textcolor{lgray}{\fontsize{10}{12} \headertitle}}
	\fancyfoot[C]{\textcolor{lgray}{\small \nombreautor}}
	\fancyfoot[R]{\textcolor{lgray}{\small \thepage}}
	\renewcommand{\headrulewidth}{0pt}
	\renewcommand{\footrulewidth}{0pt}
}{
\ifthenelse{\equal{\hfstyle}{style3}}{
	\pagestyle{fancy} \fancyhf{}
	\fancyfoot[L]{\small \thepage}
	\renewcommand{\headrulewidth}{0pt}
	\renewcommand{\footrulewidth}{0pt}
}{
\ifthenelse{\equal{\hfstyle}{style4}}{
	\pagestyle{fancy} \fancyhf{}
	\fancyfoot[C]{\small \thepage}
	\renewcommand{\headrulewidth}{0pt}
	\renewcommand{\footrulewidth}{0pt}
}{
\ifthenelse{\equal{\hfstyle}{style5}}{
	\pagestyle{fancy} \fancyhf{}
	\fancyfoot[R]{\small \thepage}
	\renewcommand{\headrulewidth}{0pt}
	\renewcommand{\footrulewidth}{0pt}
}{
\ifthenelse{\equal{\hfstyle}{style6}}{
	\pagestyle{fancy} \fancyhf{}
	\fancyfoot[L]{\small \thepage}
	\renewcommand{\headrulewidth}{0pt}
	\renewcommand{\footrulewidth}{0.5pt}
}{
\ifthenelse{\equal{\hfstyle}{style7}}{
	\pagestyle{fancy} \fancyhf{}
	\fancyfoot[C]{\small \thepage}
	\renewcommand{\headrulewidth}{0pt}
	\renewcommand{\footrulewidth}{0.5pt}
}{
\ifthenelse{\equal{\hfstyle}{style8}}{
	\pagestyle{fancy} \fancyhf{}
	\fancyfoot[R]{\small \thepage}
	\renewcommand{\headrulewidth}{0pt}
	\renewcommand{\footrulewidth}{0.5pt}
}{
\ifthenelse{\equal{\hfstyle}{style9}}{
	\pagestyle{fancy} \fancyhf{}
	\fancyfoot[L]{\small \nombreautor}
	\fancyfoot[R]{\small \thepage}
	\renewcommand{\headrulewidth}{0pt}
	\renewcommand{\footrulewidth}{0.5pt}
}{
	\throwbadconfigondoc{Estilo de header-footer incorrecto}{\hfstyle}{style1..style9}}}}}}}}}
}

% Escribe el encabezado
\writeheader

% Estilo de título de secciones
\sectionfont{\color{\titlecolor} \fontsizetitle \styletitle \selectfont}

}
